\section{金属半导体接触及其能级图}

\subsection{金属和半导体的功函数}
在讨论金属和半导体的接触之前,我们要首先弄清金属的能级状况,如\xref{fig:金属和半导体的能级图}所示。
\begin{Figure}[金属和半导体的能级图]
    \begin{FigureSub}[金属的能级图]
        \includegraphics[scale=0.9]{build/Chapter07A_01.fig.pdf}
    \end{FigureSub}
    \hspace{1cm}
    \begin{FigureSub}[半导体的能级图]
        \includegraphics[scale=0.9]{build/Chapter07A_02.fig.pdf}
    \end{FigureSub}
\end{Figure}

金属的能级图如\xref{fig:金属的能级图}所示,在\xref{sec:半导体中的能带结构}中,我们曾提到过,金属作为导体,其价带和导带是重合的。因此,金属中并没有禁带、价带顶、导带底之类的概念,但是,金属中仍然可以适用费米能级$(E_\text{F})_\text{m}$的想法,在温度作用下,部分$(E_\text{F})_\text{m}$以下的电子有可能跃迁到$(E_\text{F})_\text{m}$以上的能级,但是,绝大部分电子仍然不能脱离金属逸出体外。这表明,金属内部的电子是在一个势阱中运动,势阱底部即是费米能级$(E_\text{F})_\text{m}$,势阱边沿的能级以$E_0$表示,称为\uwave{真空能级}。

金属的真空能级和其费米能级的差,称为金属的\uwave{功函数}(Work Function)。

\begin{BoxDefinition}[金属的功函数]
    金属的功函数$W_\text{m}$,定义为其真空能级和费米能级之差
    \begin{Equation}
        W_\text{m}=E_0-(E_\text{F})_\text{m}
    \end{Equation}
\end{BoxDefinition}

金属的功函数的意义是,一个起始能量为费米能级的电子,由金属内部逸出所需的最小能量。

金属的功函数通常在几个电子伏的数量级,功函数随原子序数递增具有周期性变化,其中
\begin{itemize}
    \item 铯的功函数最小,为$1.93\si{eV}$。
    \item 铂的功函数最大,为$5.36\si{eV}$。
\end{itemize}

半导体的能级图如\xref{fig:半导体的能级图},其也可以有真空能级的概念,当然也可以定义功函数。

\begin{BoxDefinition}[半导体的功函数]
    半导体的功函数$W_\text{s}$,定义为其真空能级和费米能级之差
    \begin{Equation}
        W_\text{s}=E_0-(E_\text{F})_\text{s}
    \end{Equation}
\end{BoxDefinition}

半导体中还定义有电子亲合能的概念
\begin{BoxDefinition}[半导体的电子亲合能]
    半导体的电子亲合能$\chi$,定义为其真空能级和导带底之差
    \begin{Equation}
        \chi=E_0-E_\text{c}
    \end{Equation}
\end{BoxDefinition}
\begin{BoxFormula}[功函数的电子亲合能表示]
    半导体的功函数$W_\text{s}$可以用电子亲合能表示为
    \begin{Equation}
        W_\text{s}=\chi+[E_\text{c}-(E_\text{F})_\text{s}]=\chi+E_\text{n}
    \end{Equation}
    其中$E_\text{n}$代表导带底和费米能级的间距。
\end{BoxFormula}

简而言之,这里共有$W_\text{s}, \chi, E_\text{n}$三个量,$W_\text{s}$是真空能级与费米能级之差,而$\chi, E_\text{n}$则分别是其中的两段:真空能级与导带底之差、导带底与费米能级之差。因此有$W_\text{s}=\chi+E_\text{n}$成立。

\subsection{金属和半导体的接触电势差}
现在,设想有一块金属和一块N型半导体,现在我们试着将它们逐渐接触,如\xref{tab:金属半导体接触能级图}所示。
\begin{Table}[金属半导体接触能级图]{|c|c|c|c|}
\xcell<c>[1em][0em]
{\includegraphics[height=4.25cm]{build/Chapter07A_03.fig.pdf}}&
\xcell<c>[1em][0em]
{\includegraphics[height=4.25cm]{build/Chapter07A_04.fig.pdf}}&
\xcell<c>[1em][0em]
{\includegraphics[height=4.25cm]{build/Chapter07A_05.fig.pdf}}&
\xcell<c>[1em][0em]
{\includegraphics[height=4.25cm]{build/Chapter07A_06.fig.pdf}}\\
接触前&间隙较大&间隙较小&紧密接触\\
\end{Table}

通常来说,金属和半导体的功函数$W_\text{m}, W_\text{s}$是不同的,可能$W_\text{m}>W_\text{s}$,可能$W_\text{m}<W_\text{s}$。这里我们以前者为例讨论,即$W_\text{m}>W_\text{s}$,金属的功函数大于半导体的功函数。相应的,金属的费米能级$(E_\text{F})_\text{m}$低于半导体的费米能级$(E_\text{F})_\text{s}$。这是金属和半导体接触前的情况,如果用一根导线将金属和半导体连接起来,它们就成为了一个统一的电子系统。由于原先$(E_\text{F})_\text{s}$比$(E_\text{F})_\text{m}$高,因此电子将由半导体向金属流动,金属表面带负电,半导体表面带正电,整个系统仍然保持电中性,此时,金属电势降为$V_\text{m}$,半导体电势升为$V_\text{s}$,这将导致半导体的能带下移,直至最终半导体的费米能级$(E_\text{F})_\text{s}$与金属的费米能级$(E_\text{F})_\text{m}$到达同一水平,此时电子不再流动。

金属和半导体的费米能级相差$W_\text{m}-W_\text{s}$,这将由两者间的电势差补偿,称为接触电势差。

在间隙较大时,接触电势差完全落在金属和半导体表面间的导线上,即
\begin{Equation}
    W_\text{m}-W_\text{s}=q(V_\text{s}-V_\text{m})
\end{Equation}

在间隙较小时,这种想法就不完全正确了,这是因为,半导体中自由电荷密度是有限制的,因此,半导体带的正电荷实际是分布在表面相当厚的一层表面层内的,称为空间电荷区\footnote{没错,就是PN结势垒的那个空间电荷区,后面我们会看到金半接触与PN结有许多相似之处,共用一些名词便不奇怪了。}。而空间电荷区内也存在电场,能带在表面附近将向上弯曲,因此,接触电势差实际上,一部分降落在金属和半导体间,一部分降落在空间电荷区。设半导体内部的电势比表面高$V_\text{D}$,则有
\begin{Equation}
    W_\text{m}-W_\text{s}=q(V_\text{s}-V_\text{m})+qV_\text{D}
\end{Equation}
这里值得说明的是,在这种情况下$W_\text{s}$到底是什么?如\xref{tab:金属半导体接触能级图}所示,此时$W_\text{s}=\chi+E_\text{n}$仍然是成立的,但是$W_\text{s}$被拆成了分离的两段,第一段$E_\text{n}$是内部的导带底和费米能级的间距,第二段$\chi$是表面的导带底和真空能级的间距,中间被$qV_\text{D}$隔开。而$W_\text{s}=E_0-(E_\text{F})_\text{s}$不再成立。

随着间隙的持续减小,在接触电势差$W_\text{m}-W_\text{s}$中,金属和半导体间$q(V_\text{s}-V_\text{m})$的部分会越来越少,空间电荷区$qV_\text{D}$的部分则会越来越多,直至最终两者直接接触。此时,金属和半导体的表面成为等电势面,即$q(V_\text{s}-V_\text{m})=0$,接触电势差弯曲落在空间电荷区上,即有
\begin{Equation}
    W_\text{m}-W_\text{s}=qV_\text{D}
\end{Equation}
我们之后将总是考虑这种直接接触的极限情形。

值得注意的是,在这种极限情形下,由于金属和半导体表面等势,真空能级将会重新统一,但同时,真空能级也将失去其实际意义,因为现在,电子由金属到达半导体,不再需要先逸出金属表面再通过导线到达半导体,电子的能级只要能超过半导体表面处的导带底所在的能级,其就可以进入半导体。故之后,我们将不再绘制真空能级,而是绘制到表面处的导带底为止。

这种情况下,金属一边的势垒高度(即金属费米能级至半导体表面的导带底)为
\begin{Equation}
    q\phi_\text{ns}=qV_\text{D}+E_\text{n}
\end{Equation}
这里$q\phi_\text{ns}$亦可以表示为
\begin{Equation}
    q\phi_\text{ns}=W_\text{m}-W_\text{s}+E_\text{n}
\end{Equation}
以上都是在$W_\text{m}>W_\text{s}$的假设下讨论,现将其与$W_\text{m}<W_\text{s}$的情况对比一下,如\xref{fig:半导体的阻挡层与反阻挡层}所示。
\begin{Figure}[半导体的阻挡层与反阻挡层]
    \begin{FigureSub}[阻挡层($W_\text{m}>W_\text{s}$);阻挡层]
        \includegraphics{build/Chapter07A_07.fig.pdf}
    \end{FigureSub}
    \hspace{0.5cm}
    \begin{FigureSub}[反阻挡层($W_\text{m}<W_\text{s}$);反阻挡层]
        \includegraphics{build/Chapter07A_10.fig.pdf}
    \end{FigureSub}
\end{Figure}
我们知道,能带的弯曲$qV_\text{D}=W_\text{m}-W_\text{s}$
\begin{itemize}
    \item 若$W_\text{m}>W_\text{s}$,则有$qV_\text{D}>0$,半导体的能带在表面向上弯曲,如\xref{fig:阻挡层}所示。
    \item 若$W_\text{m}<W_\text{s}$,则有$qV_\text{D}<0$,半导体的能带在表面向下弯曲,如\xref{fig:反阻挡层}所示。
\end{itemize}
这里\xref{fig:半导体的阻挡层与反阻挡层}中,对于那些符号变化的量,均采用单向键头的标识,其含义是,箭头由一个能级指向另一个能级,则代表该量从定义上是后者减前者,具体而言
\begin{itemize}
    \item 若箭头由下向上指,则为“上减下”,即该量的值为正。
    \item 若键头由上向下指,则为“下减上”,即该量的值为负。
\end{itemize}

我们注意到,当$W_\text{m}>W_\text{s}$能带向上弯曲时,很明显,越靠近表面,半导体的导带底距离费米能级的距离就越远,该区域的电子便很少(空间电荷区正是这些电子离开后形成的),是一个\textbf{高电阻的区域},阻碍电流通过,称为\uwave{阻挡层}。相反,当$W_\text{m}<W_\text{s}$能带向下弯曲时,此时半导体的表面,是一个\textbf{高电导的区域},对于半导体和金属的接触几乎没有影响,称为\uwave{反阻挡层}。

以上是对于N型半导体而言的,若对于P型半导体,阻挡层和反阻挡层的条件将恰好相反。