\section{半导体中的载流子}

\subsection{有效质量的引入}
晶体中电子的能量形成能带,而$E(k)$和$k$的关系如\xref{tab:能带结构的三种图示}所示,但事实是,我们仅仅是半定量的确定了$E(k)$是$k$的周期函数,仍然没有得出$E(k)$的具体形式,这确实是很繁琐的。

不过,对于半导体而言,由于起作用的常常是接近价带顶部或导带底部处的电子,因此,只要掌握其价带顶部和导带底部,也就是能带极值处$E(k)$与$k$的关系式就可以了,而这就给我们的近似计算提供了可能。近似技巧是应用泰勒级数,假设能带极值位于波矢$k=0$处,由于能带底部附近的$k$值必然很小,因此可以将$E(k)$在$k=0$处泰勒展开,并仅取至二次项,得到\setpeq{有效质量的引入}
\begin{Equation}&[1]
    E(k)=E(0)+\eval{\dv{E}{k}}_{k=0}k+\frac{1}{2}\eval{\dv[2]{E}{k}}_{k=0}k^2
\end{Equation}
这里,由于在能带极值处一阶导数为零,即$(\dv*{E}{k})|_{k=0}=0$,故
\begin{Equation}&[2]
    E(k)-E(0)=\frac{1}{2}\eval{\dv[2]{E}{k}}_{k=0}k^2
\end{Equation}
这里$E(0)$是该能带底部的能量,而对于给定半导体,这里$(\dv*[2]{E}{k})$应该是一个定值。

基于此,我们可以定义一个代换变量(我们马上会看到它为什么称为“质量”)
\begin{BoxDefinition}[有效质量]
    定义半导体中的电子,在能带极值处的\uwave{有效质量}(Effective Mass)$\mne$为
    \begin{Equation}
        \frac{1}{\mne}=\frac{1}{\hbar^2}\eval{\dv[2]{E}{k}}_{k=0}
    \end{Equation}
\end{BoxDefinition}
运用有效质量的概念,\xrefpeq[有效质量的引入]{2}就可以改写为
\begin{Equation}
    E(k)-E(0)=\frac{\hbar^2k^2}{2\mne}
\end{Equation}
这和先前自由电子的能量表达式非常接近
\begin{Equation}
    E(k)=\frac{\hbar^2k^2}{2m_0}
\end{Equation}
唯一的差异是,电子的质量$m_0$被替换了能带极值处电子的有效质量$\mne$,换言之,\empx{半导体中位于能带极值处的电子的行为,可以用特定质量的自由电子等效}。我们知道,半导体中的电子会受到大量原子核和大量其他电子的群体作用,而现在,我们将这种群体作用的结果用一个特定质量简单的自由电子来代替。从这样的观点看,半导体中的电子在应用上述能带极值处的等效后,实际上,是一种\uwave{准粒子}(Quasiparticles)或\uwave{集体激发}(Collective Excitations)。\cite{W3}

有效质量,依据上述讨论,其实也就是半导体中电子对应的“准粒子”的“准质量”了,简洁起见,我们后面不再区分“半导体中的电子”和“半导体中的电子对应的准粒子”的提法。

有效质量$\mne$可以是正的,也可以是负的
\begin{itemize}
    \item 价带顶部处$E(0)>E(k)$,因此,价带顶部的有效质量$\mne<0$。
    \item 导带底部处$E(0)<E(k)$,因此,导带底部的有效质量$\mne>0$。
\end{itemize}

接下来,我们来看有效质量是如何与我们认知中的质量相一致的。

\subsection{半导体中电子的速度}\setpeq{半导体中电子的速度}
根据量子力学的概念,电子的运动速度相当于其波包的群速
\begin{Equation}&[1]
    v=\dv{\omega}{k}
\end{Equation}
在该式中,代入德布罗意关系$E=\hbar\omega$
\begin{Equation}&[2]
    v=\frac{1}{\hbar}\dv{E}{k}
\end{Equation}
在上式中将能量$E$用有效质量表示
\begin{Equation}&[3]
    v=\frac{1}{\hbar}\dv{k}\qty(\frac{\hbar^2k^2}{2\mne}+E_0)=\frac{\hbar k}{\mne}    
\end{Equation}
由此,半导体中电子的速度就可以用其有效质量表示了
\begin{BoxFormula}[半导体中电子的速度]
    半导体中,电子的速度可以用有效质量表示为
    \begin{Equation}
        v=\frac{\hbar k}{\mne}
    \end{Equation}
\end{BoxFormula}

\subsection{半导体中电子的加速度}\setpeq{半导体中电子的加速度}
实际中,许多半导体器件都在一定的外加电压下工作,这样一来,半导体内部就产生了外加电场,这种情况下,电子还会受到外加电场力的作用,那此时,电子将会以何种规律运动?

我们要回答的其实就是,如果电子受力为$f$,那么电子的加速度是多少?

计算外力$f$对电子做的功
\begin{Equation}&[1]
    \dd{E}=f\dd{s}=fv\dd{t}
\end{Equation}
将\xrefpeq[半导体中电子的速度]{2}中$v=(\dv*{E}{k})/\hbar$代入上式
\begin{Equation}&[2]
    \dd{E}=\frac{f}{\hbar}\dv{E}{k}\dd{t}
\end{Equation}
但很显然的是
\begin{Equation}&[3]
    \dd{E}=\dv{E}{k}\dd{k}
\end{Equation}
这样,对比一下\xrefpeq{2}和\xrefpeq{3}
\begin{Equation}&[4]
    \frac{f}{\hbar}\dd{t}=\dd{k}
\end{Equation}
这样就可以确定$f$
\begin{Equation}&[5]
    f=\hbar\dv{k}{t}
\end{Equation}
这就表明,\empx{电子波矢的变化率正比于外力},而根据\fancyref{fml:半导体中电子的速度},既然此处电子的波矢在变化,那么电子的速度也必然在变化,这就意味着,电子将具有加速度。

在加速度定义的基础上应用\xrefpeq[半导体中电子的速度]{2},即$v=(\dv*{E}{k})/\hbar$
\begin{Equation}&[6]
    a=\dv{v}{t}=\frac{1}{\hbar}\dv{t}\qty(\dv{E}{k})=\frac{1}{\hbar}\dv[2]{E}{k}\dv{k}{t}
\end{Equation}
在上式中代入\xrefpeq{5}
\begin{Equation}
    a=\frac{f}{\hbar^2}\dv[2]{E}{k}
\end{Equation}
根据\fancyref{def:有效质量},即$(1/\mne)=(\dv*[2]{E}{k})/\hbar^2$
\begin{Equation}
    a=\frac{f}{\mne}
\end{Equation}

这就表明,电子所受外力与加速度的关系,在使用有效质量时符合牛顿第二定律。
\begin{BoxFormula}[半导体中电子的加速度]
    半导体中,电子的加速度和所受外力间满足以下关系
    \begin{Equation}
        a=\frac{f}{\mne}
    \end{Equation}
\end{BoxFormula}

\subsection{空穴的进一步讨论}
在\xref{subsec:能带与固体导电性}中曾提到,导带中是电子导电,价带中是空穴导电。实际上,空穴也可以视为一种准粒子,因为,空穴的实质是电子群体运动导致的在能带上移动的空量子态。既然空穴是准粒子,它也可以定义有效质量的概念,那么,空穴的有效质量和电子的有效质量有何种关系?

\begin{BoxFormula}[空穴和电子的有效质量的关系]
    空穴的效质量$\mpe$总是相应电子有效质量$\mne$的负值
    \begin{Equation}
        \mpe=-\mne
    \end{Equation}
\end{BoxFormula}
在这里,让我们来总结一下
\begin{itemize}
    \item 价带中空穴导电,空穴带正电,空穴质量$\mpe$是价带顶电子有效质量的负值。
    \item 导带中电子导电,电子带负电,电子质量$\mne$是导带底电子有效质量的正值。
\end{itemize}
价带顶电子有效质量为负,导带底电子有效质量为正,因此,\empx{电子和空穴的质量均为正}。






