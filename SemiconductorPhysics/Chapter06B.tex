\section{PN结的结构与能带图}\nopagebreak

\subsection{PN结的结构}
试想,在PN结中,N型部分电子很多而空穴很少,P型部分空穴很多而电子很少,这就是说,载流子在PN结中存在浓度梯度,而\xref{sec:载流子的扩散运动}指出,载流子会在浓度梯度下作扩散运动
\begin{itemize}
    \item P型区的多子是空穴,空穴将由P区向N区扩散。
    \item N型区的多子是电子,电子将由N区向P区扩散。
\end{itemize}
我们知道,当P型半导体和N型半导体各自独立存在时,两者都是电中性的,因为杂质电离产生了多少空穴和电子,就相应的有多少电荷量相反的电离受主杂质和电离施主杂质。然而在PN结中,空穴扩散流和电子扩散流从P区和N区带走了相当数量的空穴和电子,电离杂质却无法跟随其运动,将会留在原位,这就使得在PN结交界面附近出现了一些带电区
\begin{itemize}
    \item P型区一侧出现了由电离受主构成的\uwave{负电荷区}(与P区多子空穴带正电相反)。
    \item N型区一侧出现了由电离施主构成的\uwave{正电荷区}(与N区多子电子带负电相反)。
\end{itemize}
我们通常将PN结附近的电离施主和电离受主所带的电荷称为\uwave{空间电荷}(Space Charge),而它们所存在的区域则称为\uwave{空间电荷区}(Space Charge Region)。正如我们在\xref{sec:爱因斯坦关系式}中熟悉的那样,这在PN结中,形成了一个由N区指向P区的内建电场,而载流子在内建电场的作用下将作漂移运动,显然,电子和空穴的漂移运动方向与它们的扩散运动是相反的,换言之,空间电荷区形成的内建电场将会反过来限制其自身进一步扩大。在一段时间后,电子和空穴的扩散电流和漂移电流会达成平衡,大小相等,方向相反,相互抵消。此时,流过PN结的净电流为零,空间电荷区保持一定宽度不再继续扩展,空间电荷区中存在一定的内建电场。

\begin{Figure}[PN结的空间电荷区]
    \includegraphics[scale=0.9]{build/Chapter06A_06.fig.pdf}
\end{Figure}

我们将这种热平衡状态下的PN结,称为\uwave{平衡PN结}(PN Junction in Equilibrium)。


\subsection{PN结的能带图}
在理解PN结的能带图前,我们不妨先来回顾一些,对于两块独立的P型半导体和N型半导体,它们的能带图是什么样的?如\xref{fig:P型和N型半导体的能带图},两者应具有相同的价带底$E_\text{v}$和导带底$E_\text{c}$以及相同的禁带中线$E_\text{i}$\ \footnote{确切的说$E_\text{i}$是本征状态下的费米能级,根据\xref{chap:半导体中载流子的统计分布},尽管$E_\text{i}$随着温度升高会偏离中线,但偏离的不多,可以近似相等。}。然而,两者的费米能级$E_\text{Fp},E_\text{Fn}$是不同的,我们知道,费米能级与载流子的分布是密切相关的,P区空穴较多故$E_\text{Fp}$靠近价带,N区电子较多故$E_\text{Fn}$靠近导带。

\vspace{0.25cm}
\begin{Figure}[平衡PN结的能带图]
    \begin{FigureSub}[P型和N型半导体的能带图]
        \includegraphics[scale=0.75]{build/Chapter06A_07.fig.pdf}
    \end{FigureSub}
    \\ \vspace{1cm}
    \begin{FigureSub}[PN结的能带图]
        \includegraphics[scale=0.75]{build/Chapter06A_08.fig.pdf}
    \end{FigureSub}
\end{Figure}
现在的问题是,当P型半导体和N型半导体形成PN结后,在交界区域,这些能量将会以何种方式变化?很自然的一种想法是,原先相同的$E_\text{c},E_\text{v},E_\text{i}$仍然相同,以直线相连,原先不同的$E_\text{F}$在空间电荷区以某种平滑的方式由$E_\text{Fp}$抬升至$E_\text{Fc}$,但很遗憾的是,这并不正确。正如\xref{fig:PN结的能带图}所示的那样,恰相反!费米能级反而在PN结中变得相等!费米能级$E_\text{Fp}$和$E_\text{Fn}$在各自区域中相对$E_\text{cp},E_\text{vp}$和$E_\text{cn},E_\text{vp}$的位置亦没有变化,这就是说,能带发生了相对移动!\goodbreak

我们可能会对这个结果感到不解,毕竟,为何要假设能带相对移动呢?但其实细想一下,能带移动才是更为合理的假设。因为PN结中存在由N区指向P区的内建电场,换言之,PN结中的电势不是均等的,P区电势较低,N区电势较高。尽管此刻我们尚不知道电势$V(X)$具体的数学形式\footnote{稍后有一张PN结电势$V(x)$的图像(其实就是把$E_\text{c},E_\text{v}$的趋势倒过来),再晚些我们会推导$V(X)$的表达式。},但我们可以预见的是,若以P区电势作为电势$V(X)$的零参考点
\begin{itemize}
    \item 电势$V(x)$在负端,即远离空间电荷区的P区为$0$。
    \item 电势$V(x)$在正端,即远离空间电荷区的N区为某一定值,记作$V_\text{D}$。
    \item 电势$V(X)$在空间电荷区中,逐渐由$0$增长至$V_\text{D}$(姑且不关心这个函数的具体形式)。
\end{itemize}

而能带反映的是电子的能量,这当然要包含电子的电势能$-qV(X)$,而$-qV(X)$从负到正是逐渐减小的,这也就是为何在PN结中,N型区能带相对P型区能带会向下偏移。所以说,能带移动是合理的。但这仍然没有解释,能带移动后为何恰好对齐至费米能级一致$E_\text{Fp}=E_\text{Fn}$的位置了呢?实际上,这是平衡PN结中没有净电流的直接结果,下面我们来证明这一点。\setpeq{电流密度与费米能级的关系}

我们先来计算电子电流,根据\fancyref{fml:载流子的漂移扩散}
\begin{Equation}&[1]
    J_\text{n}=qn\mu_\text{n}\Emf+qD_\text{n}\dv{n}{x}
\end{Equation}
根据\fancyref{law:爱因斯坦关系式},将$D_\text{n}$代换
\begin{Equation}&[2]
    J_\text{n}=nq\mu_\text{n}\Emf+\kB T\mu_\text{n}\dv{n}{x}
\end{Equation}
考虑到
\begin{Equation}&[3]
    \dv{\ln n}{x}=\frac{1}{n}\dv{n}{x}
\end{Equation}
将\xrefpeq{3}代入\xrefpeq{2}
\begin{Equation}&[4]
    J_\text{n}=nq\mu_\text{n}\Emf+n\kB T\mu_\text{n}\dv{\ln n}{x}
\end{Equation}
试着提取一些公共项
\begin{Equation}&[5]
    J_\text{n}=nq\mu_\text{n}\qty[\Emf+\frac{\kB T}{q}\dv{\ln n}{x}]
\end{Equation}
根据\fancyref{fml:非平衡态下的电子浓度}
\begin{Equation}&[6]
    n=n_\text{i}\exp(\frac{E_\text{F}-E_\text{i}}{\kB T})
\end{Equation}
在\xrefpeq{6}两端取对数
\begin{Equation}&[7]
    \ln n=\ln n_i+\frac{E_\text{F}-E_\text{i}}{\kB T}
\end{Equation}
进而是
\begin{Equation}&[8]
    \dv{\ln n}{x}=\frac{1}{\kB T}\qty(\dv{E_\text{F}}{x}-\dv{E_\text{i}}{x})
\end{Equation}
将\xrefpeq{8}代入\xrefpeq{5},得到
\begin{Equation}&[9]
    J_\text{n}=nq\mu_\text{n}\qty[\Emf+\frac{1}{q}\qty(\dv{E_\text{F}}{x}-\dv{E_\text{i}}{x})]
\end{Equation}
而根据前面的讨论,$E_\text{i}$的变化与$-qV(x)$是一致的,故
\begin{Equation}&[10]
    \dv{E_\text{i}}{x}=-q\dv{V(x)}{x}=-q\Emf
\end{Equation}
将\xrefpeq{10}代入\xrefpeq{9}中
\begin{Equation}&[11]
    J_\text{n}=nq\mu_\text{n}\dv{E_\text{F}}{x}
\end{Equation}
类似的,我们亦可以证明
\begin{Equation}&[12]
    J_\text{p}=p\mu_\text{p}\dv{E_\text{F}}{x}
\end{Equation}
而在平衡PN结中,电子和空穴的净电流$J_\text{n},J_\text{p}$均为零
\begin{Equation}
    J_\text{n}=J_\text{p}=0
\end{Equation}
因而
\begin{Equation}
    \dv{E_\text{F}}{x}=0\qquad E_\text{F}=C
\end{Equation}
这就论证了在平衡PN结中,费米能级$E_\text{F}$必须是一个常数。

这里\xrefpeq{11}和\xrefpeq{12}亦有理论价值,其将电流密度与费米能级联系在了一起。
\begin{BoxFormula}[电流密度与费米能级]
    电子的电流密度,正比于电子浓度和费米能级的变化率的积
    \begin{Equation}
        J_\text{n}=nq\mu_\text{n}\dv{E_\text{F}}{x}
    \end{Equation}
    空穴的电流密度,正比于空穴浓度和费米能级的变化率的积
    \begin{Equation}
        J_\text{p}=pq\mu_\text{P}\dv{E_\text{F}}{x}
    \end{Equation}
\end{BoxFormula}

\subsection{PN结的接触电势差}
在本小节,我们解决一个问题,即先前的$V_D$到底是多少?我们将平衡PN结空间电荷区两端间的电势差$V_\text{D}$称为PN结的\uwave{内建电势差}或\uwave{接触电势差}(Contact Potential Difference)。

在\xref{fig:PN结的能带图}中,我们看到,PN结的接触电势差$V_\text{D}$使PN结在空间电荷区中的能带发生弯曲,而因能带弯曲,电子从势能低的N区向势能高的P区运动时,必须克服这一势能“壁垒”,故空间电荷区亦被形象的称为\uwave{势垒区}(Barrier Region),势垒的高$qV_D$则称为\uwave{势垒高度}(Barrier Height)。很明显,势垒高度$qV_\text{D}$正好补偿了N区和P区原有的费米能级之差,即有
\begin{BoxFormula}[势垒高度]
    势垒高度补偿了N区和P区的费米能级之差,即
    \begin{Equation}
        qV_\text{D}=E_\text{Fn}-E_\text{Fp}
    \end{Equation}
\end{BoxFormula}
那么,接触电势差$V_\text{D}$到底如何求呢?关键在于利用载流子浓度。\setpeq{接触电势差}

在PN结中,载流子浓度有四个,以下分别表示:N区空穴、N区电子、P区空穴、P区电子
\begin{Equation}&[1]
    p_\text{N0}\qquad n_\text{N0}\qquad
    p_\text{P0}\qquad n_\text{P0}
\end{Equation}
在这里我们要用的是N区和P区中的电子浓度$n_\text{N0},n_\text{P0}$,根据\fancyref{fml:导带电子浓度}
\begin{Equation}&[2]
    n_\text{N0}=n_\text{i}\exp(\frac{E_\text{Fn}-E_\text{i}}{\kB T})\qquad
    n_\text{P0}=n_\text{i}\exp(\frac{E_\text{Fp}-E_\text{i}}{\kB T})
\end{Equation}
两式相除,取对数
\begin{Equation}&[3]
    \ln\frac{n_\text{N0}}{n_\text{P0}}=\frac{1}{\kB T}\qty(E_\text{Fn}-E_\text{Fp})=\frac{qV_\text{D}}{\kB T}
\end{Equation}
关注到$n_\text{N0}$为N型区多子浓度,而$n_\text{P0}$为P型区少子浓度
\begin{Equation}&[4]
    n_\text{N0}=N_\text{D}\qquad
    n_\text{P0}=\frac{n_\text{i}^2}{N_\text{A}}
\end{Equation}
将\xrefpeq{4}代入\xrefpeq{3}左端
\begin{Equation}&[5]
    \ln\frac{n_\text{N0}}{n_\text{P0}}=\ln\frac{N_\text{D}N_\text{A}}{n_\text{i}^2}
\end{Equation}
联立\xrefpeq{3}和\xrefpeq{5}
\begin{BoxFormula}[接触电势差]
    PN结的接触电势差$V_\text{D}$满足
    \begin{Equation}
        V_\text{D}=\frac{\kB T}{q}\ln\frac{N_\text{D}N_\text{A}}{n_\text{i}^2}
    \end{Equation}
\end{BoxFormula}

\xref{fml:接触电势差}表明,PN结的接触电势差由PN结自身特性确定,掺杂越多,温度越高,禁带宽度越大($n_\text{i}$越大),接触电势差就越大。硅的禁带比锗宽,因此,硅PN结的$V_\text{D}$也比锗PN结要更大些,若$N_\text{A}=10^{17}\si{cm}^{-3},N_\text{D}=10^{15}\si{cm^{-3}}$,室温下,硅$V_\text{D}=0.70\si{V}$,锗$V_\text{D}=0.32\si{V}$。

\subsection{PN结的载流子分布}
在本小节,我们要研究在PN结中,形成空间电荷区后,载流子的浓度将会如何分布?
\begin{BoxFormula}[PN结的平衡载流子浓度]
    PN结在平衡状态下,电子的平衡载流子浓度为
    \begin{Equation}&[A]
        n(x)=n_\text{N0}\exp[\frac{qV(x)-qV_\text{D}}{\kB T}]
    \end{Equation}
    PN结在平衡状态下,空穴的平衡载流子浓度为
    \begin{Equation}&[B]
        p(x)=p_\text{P0}\exp[\frac{-qV(x)}{\kB T}]
    \end{Equation}
    这里$n_\text{N0}$和$p_\text{P0}$分别是P区和N区的多子浓度。
\end{BoxFormula}
\begin{Proof}
    根据\fancyref{fml:导带电子浓度},这里$E_\text{cn}$代表N区的导带底
    \begin{Equation}
        \qquad\qquad
        n(x)=N_\text{c}\exp[\frac{E_\text{F}-E_\text{c}(x)}{\kB T}]=N_\text{c}\exp[\frac{E_\text{F}-E_\text{cn}}{\kB T}]\exp[\frac{E_\text{cn}-E_\text{c}(x)}{\kB T}]
        \qquad\qquad
    \end{Equation}
    引入$n_\text{N0}$,如\xref{fig:PN结的能带图}所示,取$E_\text{cp}$为零参考点,则$E_\text{c}(x)=-qV(x)$,而$E_\text{cn}=-qV_\text{D}$
    \begin{Equation}
        n(x)=n_\text{N0}\exp[\frac{E_\text{cn}-E_\text{c}(x)}{\kB T}]=n_\text{N0}\exp[\frac{qV(x)-qV_\text{D}}{\kB T}]
    \end{Equation}
    根据\fancyref{fml:导带电子浓度},这里$E_\text{vp}$代表P区价带底
    \begin{Equation}
        \qquad\qquad
        p(x)=N_\text{v}\exp[\frac{E_\text{v}(x)-E_\text{F}}{\kB T}]=N_\text{v}\exp[\frac{E_\text{vp}-E_\text{F}}{\kB T}]\exp[\frac{E_\text{v}(x)-E_\text{vp}(x)}{\kB T}]
        \qquad\qquad
    \end{Equation}
    引入$p_\text{P0}$,如\xref{fig:PN结的能带图}所示,取$E_\text{vp}$为零参考点,则$E_\text{v}(x)=-qV(x)$,而$E_\text{vp}=0$
    \begin{Equation}
        p(x)=p_\text{P0}\exp[\frac{E_\text{v}(x)-E_\text{vp}}{\kB T}]=p_\text{P0}\exp[\frac{-qV(x)}{\kB T}]
    \end{Equation}
    由此,我们就得到了\xrefpeq{A}和\xrefpeq{B}。
\end{Proof}

若以$x_\text{p}$和$x_\text{n}$表示空间电荷区在P区和N区的边界,则
\begin{Equation}
    V(x_\text{p})=0\qquad V(x_\text{n})=V_\text{D}
\end{Equation}
而显然少子浓度分别满足$n_\text{P0}=n(x_\text{p})$和$p_\text{N0}=p(x_\text{n})$,将上式代入,即得以下结论。
\begin{BoxFormula}[PN结多子浓度与少子浓度的关系]*
    PN结中,电子在P区的少子浓度$n_\text{P0}$和其在N区的多子浓度$n_\text{N0}$的关系是
    \begin{Equation}
        n_\text{P0}=n(x_\text{p})=n_\text{N0}\exp[\frac{-qV(x)}{\kB T}]
    \end{Equation}
    PN结中,空穴在N区的少子浓度$p_\text{N0}$和其在P区的多子浓度$p_\text{P0}$的关系是
    \begin{Equation}
        p_\text{N0}=p(x_\text{n})=p_\text{P0}\exp[\frac{-qV(x)}{\kB T}]
    \end{Equation}
\end{BoxFormula}

