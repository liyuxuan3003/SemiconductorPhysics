\section{非平衡载流子的寿命}
光照停止后,非平衡载流子浓度$\delt{p}$的减小是一个过程,换言之,非平衡载流子在导带和价带中仍然有一定的生存时间,有些长一些,有些短一些,我们将这生存时间的平均值称为\uwave{非平衡载流子的寿命}(Lifetime of Non-Equilibrium Carriers),用$\tau$表示,很显然的,$1/\tau$表示的就是单位时间内非平衡载流子的\uwave{复合概率},$\delt{p}/\tau$则代表\uwave{复合率},即单位时间单位体积内净复合消失的电子--空穴对数。现在的问题是,非平衡载流子浓度$\delt{p}$随时间变化的规律是什么?

很显然,$\delt{p}$发生变化的原因在于复合,因而,$\delt{p}$随时间的负变化率即复合率
\begin{Equation}[非平衡载流子的寿命方程]
    \dv{\delt{p}}{t}=-\frac{\delt{p}}{\tau}
\end{Equation}
在小注入时,寿命$\tau$是不随时间变化的常数,故$\delt{p}$的通解为
\begin{Equation}
    \delt{p}(t)=C\e^{-t/\tau}
\end{Equation}
在$t=0$时记$\delt{p}(0)=(\delt{p})_0$,则
\begin{Equation}
    \delt{p}(t)=(\delt{p})_0\e^{-t/\tau}
\end{Equation}
这表明,\empx{非平衡载流子的浓度随时间以指数方式衰减}。

通过$\delt{p}(t)$的表达式可以反过来验证平均生存时间$\bar{t}$就是寿命$\tau$,推导如下
\begin{Equation}[寿命即平均生存时间]
    \qquad
    \bar{t}=\frac{\Int[0][\infty]t\delt{p}(t)\dd{t}}{\Int[0][\infty]\delt{p}(t)\dd{t}}=\frac{(\delt{p})_0\Int[0][\infty]t\e^{-t/\tau}\dd{t}}{(\delt{p})_0\Int[0][\infty]\e^{-t/\tau}\dd{t}}=\frac{(-\tau t\e^{-t/\tau}-\tau^2\e^{-t/\tau})|_0^{\infty}}{(-\tau\e^{-t/\tau})|_0^{\infty}}=\frac{\tau^2}{\tau}=\tau
    \qquad
\end{Equation}
通过$\delt{p}$的表达式亦可以看到,寿命$\tau$的意义是非平衡载流子下降到原先$1/\e$所经历的时间。\goodbreak

通常非平衡载流子的寿命是通过实验测量的,主要有两种方法
\begin{itemize}
    \item 光电导法,它的原理就是\xref{sec:非平衡载流子的产生与复合}中提到的,非平衡载流子会影响半导体的电导率,进而影响半导体上的电压降。在测量时,通过脉冲光照射半导体,通过示波器观察脉冲光照后电压降的指数衰减曲线(脉冲光使得指数衰减曲线周期性出现),从而确定寿命。
    \item 光电磁法,它利用的是半导体的光磁电效应,在此不做赘述。我们只需要知道,光电磁法尤为适合测量短的非平衡载流子寿命,在砷化镓等\Romnum{3}--\Romnum{5}族化合物中用的比较多。
\end{itemize}

通常来说,\empx{锗比硅的寿命高些,砷化镓的寿命则要短的多}
\begin{itemize}
    \item 较完整的锗单晶,寿命在$10^4\si{us}$以上。
    \item 较完整的硅单晶,寿命在$10^3\si{us}$以上。
    \item 砷化镓的寿命则是极短的,通常在$10^{-2}\si{us}$以下。
\end{itemize}
实际上,不仅是不同半导体材料的寿命很不相同,即便是同种半导体材料,不同条件下,寿命也可以在一个很大的范围内变化,以锗为例,其寿命可以在几十微秒到几百微秒的区间内。