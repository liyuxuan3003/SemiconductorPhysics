\section{本征半导体的载流子浓度}
\uwave{本征半导体}(Intrinsic Semiconductor)就是指一块没有杂质和缺陷的半导体,因此,本征半导体的载流子完全来自本征激发,由于本征激发中,电子从价带跃迁到导带,电子和空穴将会成对产生,因此,导带中的电子浓度应当等同于价带中的空穴浓度,这就是\uwave{电中性条件}。

\begin{BoxFormula}[本征半导体的电中性条件]
    本征半导体的电中性条件满足
    \begin{Equation}
        n_0=p_0
    \end{Equation}
\end{BoxFormula}

本征半导体的能带图,如\xref{fig:本征半导体的能带图}所示,其中,需要特别解释的是最右侧的图像,我们可能会困惑其中的$\dv*{n_0}{E}$和$\dv*{p_0}{E}$究竟是什么含义?介于$n_0$和$p_0$本身就是对能量$E$积分后的常数,又何谈对$E$的导数呢?实际上,在\xref{sec:载流子浓度}当我们通过积分求得$n_0$和$p_0$时,被积函数就是$f(E)g(E)/V$,它表征了载流子浓度的能量密度。原先的观点中$n_0,p_0$由能带底到能带顶的定积分,如果放宽一些,将$n_0,p_0$视为由能带顶到$E$的变上限积分$n_0(E),p_0(E)$,那这里的$\dv*{n_0}{E}$和$\dv*{p_0}{E}$就可以解释了,它们与$f(E)g(E)/V$是同一回事,都代表浓度的能量密度。这也是为何$f(E)g(E)/V$的曲线下面积就表示了电子浓度$n_0$和空穴浓度$p_0$的大小。
\begin{Figure}[本征半导体的能带图]
    \includegraphics[width=0.95\linewidth]{build/Chapter03D_03.fig.pdf}
\end{Figure}

这里也证实了先前的一个提法,我们注意到
\begin{itemize}
    \item 导带电子浓度的能量密度的最大值接近导带底,表明,\empx{导带电子主要位于导带底}。
    \item 价带电子浓度的能量密度的最大值接近价带顶,表明,\empx{价带空穴主要位于价带顶}。
\end{itemize}
这里关注到\xref{fig:本征半导体的能带图}中$f(E)$和$g(E)$的图像相较\xref{fig:状态密度函数}和\xref{fig:费米分布与玻尔兹曼分布},作为自变量的能量反而变为了纵轴,这其实是为了顺应简单能带图中,能量在纵向分布的习惯,也是半导体物理的通用作法。


至此,有一个疑惑,这一节我们到底要干嘛?本节题名“本征半导体的载流子浓度”,但问题是,载流子的浓度我们在\xref{sec:载流子浓度}中,已经由\fancyref{fml:导带电子浓度}和\fancyref{fml:价带空穴浓度}求出了,本节还有什么可做的?关键点在于,现有的$n_0$和$p_0$的公式中包含费米能级$E_\text{F}$,而根据固体物理的知识,\empx{费米能级是随温度变化的函数},而这个函数是我们目前未知的。因此,本节就是要求出本征半导体的费米能级的函数,进而求出$n_0$和$p_0$的具体表达。

\begin{BoxFormula}[本征半导体的费米能级]*
    本征半导体的费米能级满足
    \begin{Equation}
        E_\text{F}=\frac{E_\text{c}-E_\text{v}}{2}+\frac{\kB T}{2}\ln\frac{N_\text{v}}{N_\text{c}}
    \end{Equation}
    或
    \begin{Equation}
        E_\text{F}=E_\text{i}=\frac{E_\text{c}+E_\text{v}}{2}+\frac{3\kB T}{4}\ln\frac{\mpe}{\mne}
    \end{Equation}
    这里$E_\text{i}$是专属本征半导体的费米能级的记号,下标i表示本征(Intrinsic)。
\end{BoxFormula}
\begin{Proof}
    求解费米能级的关键,在于运用电中性条件,根据\fancyref{fml:本征半导体的电中性条件}
    \begin{Equation}&[1]
        n_0=p_0
    \end{Equation}
    根据\fancyref{fml:导带电子浓度}和\fancyref{fml:价带空穴浓度}
    \begin{Equation}&[2]
        N_\text{c}\exp(\frac{E_\text{F}-E_\text{c}}{\kB T})=N_\text{v}\exp(\frac{E_\text{v}-E_\text{F}}{\kB T})
    \end{Equation}
    在\xrefpeq{2}两端取对数
    \begin{Equation}&[3]
        \ln N_\text{c}+\qty(\frac{E_\text{F}-E_\text{c}}{\kB T})=\ln N_\text{V}+\qty(\frac{E_\text{v}}{\kB T})
    \end{Equation}
    移项整理
    \begin{Equation}&[4]
        \frac{2E_\text{F}-E_\text{c}-E_\text{v}}{\kB T}=\ln\frac{N_\text{v}}{N_\text{c}}
    \end{Equation}
    再整理
    \begin{Equation}&[5]
        \frac{2E_\text{F}}{\kB T}=\frac{E_\text{c}-E_\text{v}}{\kB T}+\ln\frac{N_\text{v}}{N_\text{c}}
    \end{Equation}
    由此即解得$E_\text{F}$
    \begin{Equation}&[6]
        E_\text{F}=\frac{E_\text{c}-E_\text{v}}{2}+\frac{\kB T}{2}\ln\frac{N_\text{v}}{N_\text{c}}
    \end{Equation}
    而我们知道$N_\text{v}\sim(\mpe)^{3/2}$和$N_\text{c}\sim(\mne)^{3/2}$
    \begin{Equation}&[7]
        E_\text{F}=\frac{E_\text{c}-E_\text{v}}{2}+\frac{3\kB T}{4}\ln\frac{\mpe}{\mne}
    \end{Equation}
    这里,我们愿意引入$E_\text{i}$作为本征半导体中费米能级$E_\text{F}$的特别记号。
\end{Proof}

\xref{fml:本征半导体的费米能级}指出,本征半导体的费米能级$E_\text{i}$由两部分组成,首先是$E_\text{c}+E_\text{v}/2$的常数部分,它表明费米能级在绝对零度时位于禁带中线上,其次是$(3\kB T/4)\ln(\mpe/\mne)$的线性项,它表明费米能级将随温度增大线性变化。不过事实上,对于大部分半导体材料,在室温范围内,线性项的影响其实都不大。例如,根据\xref{tab:半导体的状态密度有效质量}的数据,硅和锗的$\mpe/\mne$分别为$0.55$和$0.52$,砷化镓的$\mpe/\mne$则为$8.25$,故三者的$\ln(\mpe/\mne)$均大致在$\pm 2$的范围内,而$\kB T$在室温$T=300\si{K}$时为$\kB T=0.026\si{eV}$,因此整个线性项在室温下的影响小于$0.05\si{eV}$,根据\xref{tab:部分半导体材料的参数}的数据,其相较于硅、锗、砷化镓$1.14\si{eV}, 0.67\si{eV}, 1.52\si{eV}$在$1\si{eV}$量级的禁带宽度,完全可以忽略,因此我们可以近似认为,\empx{本征半导体的费米能级基本在禁带中线附近}。不过,也有例外,例如锑化铟的$\mpe/\mne$约为$36.7$,取对数后仍然有$3.60$,线性项的影响达到了近$0.07\si{eV}$,但是,锑化铟的禁带非常窄,仅有$0.17\si{eV}$,这时,线性项的影响就很显著了,费米能级已远在禁带中线上了。

现在,让我们来计算本征半导体的载流子浓度。

\begin{BoxFormula}[本征半导体的载流子浓度]
    本征半导体中,电子浓度$n_0$和空穴浓度$p_0$相等,统一记为$n_\text{i}$
    \begin{Equation}
        n_\text{i}=n_0=p_0
    \end{Equation}
    且$n_\text{i}$满足
    \begin{Equation}
        n_\text{i}=(N_\text{c}N_\text{v})^{1/2}\exp(-\frac{E_\text{g}}{2\kB T})
    \end{Equation}
\end{BoxFormula}

\begin{Proof}
    根据\fancyref{fml:导带电子浓度},并代入\fancyref{fml:本征半导体的费米能级}
    \begin{Equation}&[1]
        \qquad
        n_0=N_\text{c}\exp(\frac{E_\text{F}-E_\text{c}}{\kB T})=N_\text{c}\exp(\frac{E_\text{c}/2+E_\text{v}/2+(\kB T/2)\ln N_\text{v}/N_\text{c}-E_\text{c}}{\kB T})
        \qquad
    \end{Equation}
    化简整理得
    \begin{Equation}&[2]
        n_0=N_\text{c}\qty(-\frac{E_\text{c}-E_\text{v}}{2\kB T}+\frac{\ln N_\text{v}/N_\text{c}}{2})=N_\text{c}\sqrt{N_\text{v}/N_\text{c}}\exp(-\frac{E_\text{g}}{2\kB T})=(N_\text{c}N_\text{v})^{1/2}\exp(-\frac{E_\text{g}}{2\kB T})
    \end{Equation}
    根据\fancyref{fml:价带空穴浓度},并代入\fancyref{fml:本征半导体的费米能级}
    \begin{Equation}&[3]
        \qquad
        p_0=N_\text{v}\exp(\frac{E_\text{v}-E_\text{F}}{\kB T})=N_\text{c}\exp(\frac{E_\text{v}-E_\text{c}/2-E_\text{v}/2-(\kB T/2)\ln N_\text{v}/N_\text{c}}{\kB T})
        \qquad
    \end{Equation}
    化简整理得
    \begin{Equation}&[4]
        p_0=N_\text{v}\qty(-\frac{E_\text{c}-E_\text{v}}{2\kB T}-\frac{\ln N_\text{v}/N_\text{c}}{2})=N_\text{v}\sqrt{N_\text{c}/N_\text{v}}\exp(-\frac{E_\text{g}}{2\kB T})=(N_\text{v}N_\text{c})^{1/2}\exp(-\frac{E_\text{g}}{2\kB T})
    \end{Equation}
    比较\xrefpeq{3}和\xrefpeq{4}即得
    \begin{Equation}*
        n_0=p_0\qedhere
    \end{Equation}
\end{Proof}
由此可见,本征半导体的载流子浓度对温度非常敏感,其随温度增加而迅速增加。而在同一温度下,半导体材料的禁带宽度$E_\text{g}$越大,本征载流子的浓度$n_i$就相应越小,这从直观上想也是很合理的,因为禁带宽度越大,电子从价带跃迁到导带也越困难,载流子自然也就少了。

在\xref{fml:本征半导体的载流子浓度}中代入$N_\text{c}N_\text{v}$,并引入\fancyref{fml:禁带宽度和温度的关系}
\begin{Equation}
    n_i=2\frac{(2\pi \kB T)^{3/2}(\mpe\mne)^{3/4}}{\hbar^3}\exp[-\frac{E_\text{g}(0)}{2\kB T}]\exp[\frac{\alpha T}{2\kB (T+\beta)}]
\end{Equation}
该式可以在实验上用于测定禁带宽度。\goodbreak

正如我们所看到的,本征半导体的载流子浓度随温度增大而迅速增大,因此,使用本征半导体制造的器件热稳定性很差,所以,制造半导体器件往往使用的是恰当掺杂的杂质半导体。