\section{费米分布与玻尔兹曼分布}
在\xref{sec:状态密度}中,我们已经通过状态密度$g(E)$弄清了电子的量子态是如何分布的了,但是,并不是每一个电子可能存在的量子态都有电子存在,所以说,我们还需要找到一个电子关于能量的分布函数。事实上,尽管就单一电子而言,它所具有的能量时大时小经常变化,但是,电子按能量大小的分布具有统计规律性,这就是说,\empx{电子在不同能量上的概率密度分布是一定的}。

\subsection{费米分布}

量子统计理论指出,具有泡利不相容特性的费米子遵从费米分布,而电子作为一种费米子,其随能量$E$的概率密度分布$f(E)$也满足费米分布,这对于导带和价带(甚至禁带,只不过就算电子在禁带范围内有很高的概率密度,也没有允许电子存在的量子态)都是适用的,但是按照我们的习惯,导带讨论电子,价带讨论空穴,所以在价带中我们往往需要的是空穴的概率密度分布,空穴是电子不存在的情况,因此,空穴的统计分布其实就可以用$1-f(E)$表述。\goodbreak
\begin{BoxFormula}[费米分布]
    电子的概率密度分布遵从\uwave{费米分布}(Fermi Statistics)
    \begin{Equation}&[1]
        f(E)=\frac{1}{1+\exp(E-E_\text{F}/\kB T)}
    \end{Equation}
    空穴的概率密度分布相应遵从
    \begin{Equation}&[2]
        1-f(E)=\frac{1}{1+\exp(E_\text{F}-E/\kB T)}
    \end{Equation}
\end{BoxFormula}

在\fancyref{fml:费米分布}中的$E_\text{F}$是一个很重要的参数,称为\uwave{费米能级}(Fermi Level),它并不是一个真正的能级,只是一个具有能量量纲的参数,与温度、半导体材料、半导体的杂质类型和含量、能量零点的选取等因素有关。费米能级$E_\text{F}$作为费米分布$f(E)$的参数,具有分界线的意义,如\xref{fig:电子的统计分布}所示,\empx{费米能级是费米分布的半概率密度点},即有$f(E_\text{F})=1/2$成立。

更具体的说,电子的费米分布是这样以费米能级为界的
\begin{itemize}
    \item 在费米能级$E_\text{F}$以下,电子有较大概率出现,当$E<E_\text{F}$时有$f(E)>1/2$。
    \item 在费米能级$E_\text{F}$以上,电子有较小概率出现,当$E>E_\text{F}$时有$f(E)<1/2$。
\end{itemize}
而由\xref{fig:电子的统计分布},我们亦可以注意到温度对费米分布的影响,温度越低,费米分布的曲线在费米能级$E_\text{F}$处由$1.0$至$0.0$的跃变就越陡峭,事实上,温度为绝对零度时,此时,费米分布将转化为一个阶跃函数,换言之,电子将会完全分布在费米能级以内,严格的以最低能级填充。

而这里还有一个问题,费米能级$E_\text{F}$在哪里?事实是,\empx{费米能级通常会落在禁带中},即
\begin{Equation}
    E_\text{v}<E_\text{F}<E_\text{c}
\end{Equation}
直观上想这也是合理的,价带上有很多电子,导带上几乎没有什么电子。

\subsection{玻尔兹曼分布}
玻尔兹曼分布并不是全新的东西,它其实只不过是一定条件下对费米分布的近似。

\begin{BoxFormula}[玻尔兹曼分布]
    电子的概率密度分布在$E-E_\text{F}\gg\kB T$时遵从\uwave{玻尔兹曼分布}(Boltzmann Statistics)
    \begin{Equation}
        f(E)=\exp(\frac{E_\text{F}-E}{\kB T})
    \end{Equation}
    空穴的概率密度分布在$E-E_\text{F}\ll\kB T$时遵从
    \begin{Equation}
        1-f(E)=\exp(\frac{E-E_\text{F}}{\kB T})
    \end{Equation}
\end{BoxFormula}

\begin{Proof}
    根据\fancyref{fml:费米分布}的\xrefpeq[费米分布]{1},若$E-E_\text{F}\gg\kB T$
    \begin{Equation}*
        f(E)=\frac{1}{1+\exp(E-E_\text{F}/\kB T)}=\frac{1}{\exp(E-E_\text{F}/\kB T)}=
        \exp(\frac{E_\text{F}-E}{\kB T})
    \end{Equation}
    根据\fancyref{fml:费米分布}的\xrefpeq[费米分布]{2},若$E_\text{F}-E\gg\kB T$
    \begin{Equation}*
        1-f(E)=\frac{1}{1+\exp(E_\text{F}-E/\kB T)}=\frac{1}{\exp(E_\text{F}-E/\kB T)}=
        \exp(\frac{E-E_\text{F}}{\kB T})
    \end{Equation}
    由此可见,玻尔兹曼分布实质就是对费米分布作$1+\e^x=\e^x$近似的结果。
\end{Proof}

玻尔兹曼分布其实是非常实用的近似,因为我们研究电子的统计分布,最终目的就是要研究电子在价带和导带上的统计分布,前面我们提到过,对于大部分半导体,费米能级$E_\text{F}$往往落在禁带中,其距离导带底和价带顶都有一定的距离,因此,对于导带和价带中的能值
\begin{itemize}
    \item 在导带中$E-E_\text{F}\gg\kB T$,而导带中研究的是电子分布,符合玻尔兹曼分布的近似条件。
    \item 在价带中$E_\text{F}-E\gg\kB T$,而价带中研究的是空穴分布,符合玻尔兹曼分布的近似条件。
\end{itemize}
因此,通常我们在研究半导体时都可以应用近似的玻尔兹曼分布,适用玻尔兹曼分布的半导体称为\uwave{非简并半导体}(Non-degenerate Semiconductor),而有些情况下,费米能级$E_\text{F}$会比较靠近导带底或价带顶,这时$E-E_\text{F}\gg\kB T$或$E_\text{F}-E\gg\kB T$就不再成立了,仍然需要通过费米分布进行描述,我们将这类半导体称为\uwave{简并半导体}(Degenerate Semiconductor)。简而言之,\empx{适用费米分布的称为简并半导体,适用玻尔兹曼分布的称为非简并半导体}。我们或许会想,简并通常是描述多个量子态具有同一本征能量的,为何在又被用于描述半导体?实际上是这样的,简并一词的意义在应用中逐渐被拓宽,简并最初确实是指多个量子态具有同一本征能量,而事实是,简并态的数目越多,量子效应就会越明显,相反,非简并态可以较好的用经典物理解释,所以这样一来,简并亦被用于指称量子效应是否显著\cite{W10}。而费米分布到玻尔兹曼分布的近似从物理上看,其实就是忽略了泡利不相容原理,在\xref{fig:费米分布与玻尔兹曼分布}中,我们可以清楚的看到费米分布和玻尔兹曼分布间的关系,我们注意到,玻尔兹曼分布近似的比较好的部分其实都是费米分布值较接近$0$的部分,在这些位置,电子和空穴都很稀疏,因而不必考虑泡利不相容的。因此,\empx{作为量子效应的泡利不相容是否需要考虑,就决定了半导体是简并还是非简并的}。

玻尔兹曼分布中,还要强调的是,由于其$f(E)$和$1-f(E)$是分别由相应费米分布近似而来的,因此玻尔兹曼分布中$f(E)$和$1-f(E)$只是两个独立的记号罢了,相加并不等于$1$。

\begin{Figure}[费米分布与玻尔兹曼分布]
    \vspace{-0.15cm}
    \begin{FigureSub}[电子的统计分布]
        \includegraphics[scale=0.85]{build/Chapter03B_01.fig.pdf}\vspace{-0.15cm}
    \end{FigureSub}\vspace{0.3cm}
    \begin{FigureSub}[空穴的统计分布]
        \includegraphics[scale=0.85]{build/Chapter03B_02.fig.pdf}\vspace{-0.15cm}
    \end{FigureSub}
\end{Figure}

接下来的内容中,我们主要讨论的都是适用玻尔兹曼分布的非简并半导体。