\chapter{半导体中杂质和缺陷}
在实际应用的半导体材料的晶格中,总是存在着偏离理想情况的各种复杂现象。首先,原子并不是静止在具有严格周期性的晶格的格点位置上,而是其平衡位置附近振动;其次,半导体材料并不是纯净的,而是含有若干位置,即在半导体晶格中存在着与组成半导体材料的元素不同的其他化学元素的原子;再次,实际的半导体晶格结构并不是完整无缺的,而是存在着各种形式的缺陷。简而言之:振动、杂质、缺陷。本章,我们主要讨论半导体中的杂质和缺陷。

实践表明,极微量的杂质和缺陷,就能够对半导体材料的物理性质和化学性质阐述决定性的影响,当然,也严重地影响着半导体器件的质量。这种影响,既有消极的一面,例如提纯硅单晶时要求极高的纯度,又有积极的一面,例如我们会人为的像硅中掺入杂质原子,以构成P型硅和N型硅,两者构成的PN结是所有半导体器件的基础。那么,为什么存在于半导体中的杂质和缺陷,会起着这么重要的作用呢?理论分析认为,由于杂质和缺陷的存在,会使严格按周期性排列的原子所产生的周期性势场受到破坏,从而在原先的禁带中引入额外的能级。