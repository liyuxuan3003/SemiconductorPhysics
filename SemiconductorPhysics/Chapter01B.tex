\section{半导体中的能带结构}

\subsection{能级分裂与能带}
原子中的电子在原子核的势场和其他电子的作用下,它们分列在不同的能级桑,形成电子壳层,分别用1s, 1s, 2p, 3s, 3p, 3d等符号表示,每一电子壳层对应与一个确定的能级。原子相互接近形成晶体时,不同原子的电子壳层之间就有了一定程度的交叠,此时,电子不再局限于某一个原子上,而是可以由一个原子转移到相邻的原子上去,因而,电子将可以在整个晶体中运动,称为电子的共有化运动,但需要注意的是,电子只能在相同能级的壳层之间转移
\begin{itemize}
    \item 内层壳层的交叠较弱,因此内层壳层对应的共有化运动较弱。
    \item 外层壳层的交叠较强,因此外层壳层对应的共有化运动较强。
\end{itemize}
原子构成晶体后,电子做共有化运动时能量又是怎么样的呢?如\xref{fig:能级分裂与能带}所示,设想由$N$个原子构成的晶体(通常$N$是很大的数值),假设$N$个原子最初相距很远,尚未结合成晶体时,而此时,每个原子都可以视为孤立原子,每个原子的电子都具有相同的1s, 2s, 2p等能级,或者说,此时电子在1s, 2s, 2p上是$N$重简并的\footnote{所谓简并,就是指一系列特征函数具有相同的特征值,更具体的说,就是指一系列不同的电子波函数具有相同的能量。},而当$N$个原子逐渐靠近结合为晶体时,我们知道,电子作为一种费米子需要遵循泡利不相容准则,因此,电子的$N$重简并之间将产生逐渐增大的排斥,从而促使原先的每个能级分裂为$N$个彼此相距很近的能级,由于原子数$N$的值非常大,这$N$个能级组成一个准连续的\uwave{能带}(Energy Band)。即,原子在孤立状态下电子能量只能取一系列的分立值,原子在晶体状态下电子能量就可以在一系列的分立的区间中取值
\begin{itemize}
    \item 我们将晶体状态下,电子能量允许取值的那些区间,称为\uwave{允带}(Permitted Band)。
    \item 我们将晶体状态下,电子能量不能取值的那些区间,称为\uwave{禁带}(Forbidden Band)。
\end{itemize}
\begin{Figure}[能级分裂与能带]
    \includegraphics[scale=0.68]{build/Chapter01B_01.fig.pdf}
\end{Figure}
很明显,允带和禁带是相互交错的,它们互为彼此的空隙。

需要说明的是,如果原先的能级本身就具有简并,例如2p三重简并了2p$_x$,2p$_y$,2p$_z$三个不同取向的轨道,那么这种简并会在能级分裂时一并被释放,例如2p分裂后将产生$3N$个能级。

除此之外,能级分裂的程度并不是相同的,如\xref{fig:能级分裂与能带}
\begin{itemize}
    \item 内层电子原先处于低能级,共有化运动弱,其能级分裂很小,能带很窄。
    \item 外层电子原先处于高能级,共有化运动强,其能级分裂很大,能带很宽。
\end{itemize}

但是,由于杂化作用的影响,包括金刚石结构在内的硅和锗的能带与能级其实并不是一一对应的,如\xref{fig:能级分裂与能带}所示,首先2s的一个轨道与2p的三个轨道将杂化为四个2sp$^3$的轨道,每个轨道上各有一个电子,故$N$个原子共有$4N$个电子。然而,与我们的预期有些不符
\begin{itemize}
    \item 四重简并的2sp$^3$的轨道并未分裂为$1$个具有$4N$个能级的能带。
    \item 四重简并的2sp$^3$的轨道将会分裂为$2$个具有$2N$个能级的能带。
\end{itemize}
在这两个能带中,有一个能带能量低于2sp$^3$能级,有一个能带能量高于2sp$^3$能级,两者中的轨道分别称为\uwave{成键轨道}(Bonding Orbit)和\uwave{反键轨道}(Antibonding Orbit),显然$4N$个电子将完全填充在能量较低的$2N$个成键轨道,而能量较高的$2N$个反键轨道上则没有任何电子填充,将两者构成的能带,分别称为\uwave{价带}(Valence Band)和\uwave{导带}(Conduction band)。\cite{W2}

\begin{Figure}[能级分裂与能带]
    \includegraphics[scale=0.68]{build/Chapter01B_02.fig.pdf}
\end{Figure}

简而言之,价带和导带是允带的两个细分概念\footnote{允带给出只是电子允许存在的能量,但是这些能量未必真的填充有电子。},\empx{价带中填满了电子},\empx{导带中没有电子}。

因此,价带也称为\uwave{满带}(Filled Band),导带也称为\uwave{空带}(Empty Band)。
