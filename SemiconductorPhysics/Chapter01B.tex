\section{半导体中的能带结构}

\subsection{能级分裂与能带}
原子中的电子在原子核的势场和其他电子的作用下,它们分列在不同的能级上,形成电子壳层,分别用1s, 2s, 2p, 3s, 3p, 3d等符号表示,每一电子壳层对应与一个确定的能级。原子相互接近形成晶体时,不同原子的电子壳层之间就有了一定程度的交叠,此时,电子不再局限于某一个原子上,而是可以由一个原子转移到相邻的原子上去,因而,电子将可以在整个晶体中运动,称为电子的共有化运动,但需要注意的是,电子只能在相同能级的壳层之间转移
\begin{itemize}
    \item 内层壳层的交叠较弱,因此内层壳层对应的共有化运动较弱。
    \item 外层壳层的交叠较强,因此外层壳层对应的共有化运动较强。
\end{itemize}
原子构成晶体后,电子做共有化运动时能量又是怎么样的呢?如\xref{fig:能级分裂与能带}所示,设想由$N$个原子构成的晶体(通常$N$是很大的数值),假设$N$个原子最初相距很远,尚未结合成晶体时,而此时,每个原子都可以视为孤立原子,每个原子的电子都具有相同的1s, 2s, 2p等能级,或者说,此时电子在1s, 2s, 2p上是$N$重简并的\footnote{所谓简并,就是指一系列特征函数具有相同的特征值,更具体的说,就是指一系列不同的电子波函数具有相同的能量。},而当$N$个原子逐渐靠近结合为晶体时,我们知道,电子作为一种费米子需要遵循泡利不相容准则,因此,电子的$N$重简并之间将产生逐渐增大的排斥,从而促使原先的每个能级分裂为$N$个彼此相距很近的能级,由于原子数$N$的值非常大,这$N$个能级组成一个准连续的\uwave{能带}(Energy Band)。即,原子在孤立状态下电子能量只能取一系列的分立值,原子在晶体状态下电子能量就可以在一系列的分立的区间中取值
\begin{itemize}
    \item 我们将晶体状态下,电子能量允许取值的那些区间,称为\uwave{允带}(Permitted Band)。
    \item 我们将晶体状态下,电子能量不能取值的那些区间,称为\uwave{禁带}(Forbidden Band)。
\end{itemize}
\begin{Figure}[能级分裂与能带]
    \includegraphics[scale=0.68]{build/Chapter01B_01.fig.pdf}
\end{Figure}
很明显,允带和禁带是相互交错的,它们互为彼此的空隙。

需要说明的是,如果原先的能级本身就具有简并,例如2p三重简并了2p$_x$,2p$_y$,2p$_z$三个不同取向的轨道,那么这种简并会在能级分裂时一并被释放,例如2p分裂后将产生$3N$个能级。

除此之外,能级分裂的程度并不是相同的,如\xref{fig:能级分裂与能带}
\begin{itemize}
    \item 内层电子原先处于低能级,共有化运动弱,其能级分裂很小,能带很窄。
    \item 外层电子原先处于高能级,共有化运动强,其能级分裂很大,能带很宽。
\end{itemize}

但是,由于杂化作用的影响,包括金刚石结构在内的硅和锗的能带与能级其实并不是一一对应的,如\xref{fig:能级分裂与能带}所示,首先2s的一个轨道与2p的三个轨道将杂化为四个2sp$^3$的轨道,每个轨道上各有一个电子,故$N$个原子共有$4N$个电子。然而,与我们的预期有些不符
\begin{itemize}
    \item 四重简并的2sp$^3$的轨道并未分裂为$1$个具有$4N$个能级的能带。
    \item 四重简并的2sp$^3$的轨道将会分裂为$2$个具有$2N$个能级的能带。
\end{itemize}
在这两个能带中,有一个能带能量低于2sp$^3$能级,有一个能带能量高于2sp$^3$能级,两者中的轨道分别称为\uwave{成键轨道}(Bonding Orbit)和\uwave{反键轨道}(Antibonding Orbit),显然$4N$个电子将完全填充在能量较低的$2N$个成键轨道,而能量较高的$2N$个反键轨道上则没有任何电子填充,将两者构成的能带,分别称为\uwave{价带}(Valence Band)和\uwave{导带}(Conduction band)。\cite{W2}

\begin{Figure}[能级分裂与能带]
    \includegraphics[scale=0.68]{build/Chapter01B_02.fig.pdf}
\end{Figure}

简而言之,价带和导带是允带的两个细分概念\footnote{允带给出只是电子允许存在的能量,但是这些能量未必真的填充有电子。},\empx{价带中填满了电子},\empx{导带中没有电子}。

因此,价带也称为\uwave{满带}(Filled Band),导带也称为\uwave{空带}(Empty Band)。

\subsection{能带的定量分析}
在\xref{subsec:能级分裂与能带}中,我们定性的对比了晶体中的电子和孤立原子的电子的差异,阐述了能级是如何分裂并演化为能带的,而在这一小节,我们将定量研究晶体中的电子和自由电子的关系。

实际上,从定态薛定谔方程的角度看
\begin{eqnarray}
    -\frac{\hbar^2}{2m_e}\dv[2]{\psi}{x}+V(x)\psi=E\psi
\end{eqnarray}
自由电子即势能$V(x)=0$恒为零,晶体中的电子即势能$V(x)=V(x+a)$具有晶格周期性。

自由电子的波函数就是普通的平面波
\begin{Equation}
    \psi_k(x)=A\e^{\i kx}
\end{Equation}
自由电子的能量本征值(色散关系)满足抛物线关系
\begin{Equation}
    E(k)=\frac{\hbar^2k^2}{2m_e}
\end{Equation}
自由电子的波矢$k$可以连续取值,因此,自由电子的能量是连续能谱,任意取值都是允许的。


布洛赫曾经证明过,周期性势场$V(x)=V(x+a)$中的波函数必具有以下形式
\begin{BoxTheorem}[布洛赫定理]
    若势场$V(x)$具有周期性$V(x)=V(x+a)$,则其波函数一定有以下形式
    \begin{Equation}
        \psi_k(x)=A_k(x)\e^{\i kx}\qquad A_k(x)=A_k(x+a)
    \end{Equation}
    该结论称为\uwave{布洛赫定理}(Bloch's Theorem)。
\end{BoxTheorem}
布洛赫定理告诉我们,晶体周期性势场中的电子波函数具有$\psi_k(x)=A_k(x)\e^{\i kx}$的形式,通过和自由电子波函数$\psi_k(x)=A\e^{\i kx}$的对比,我们可以看出,\empx{晶体中的电子是以一个被周期性调幅的平面波在晶体中传播},振幅$A_k(x)$随$x$依照晶格周期$a$作周期性变化。除此之外,根据波函数的意义,在空间某点找到电子的概率密度与$|\psi|^2$成正比,因此,自由电子在空间格点处的概率密度$|\psi|^2=A^2$处处相同,晶体中的电子的概率密度$|\psi|^2=|A_kA_k^{*}|$则是周期性的。

现在的问题是,我们已经知道晶体中电子的波函数了,那么晶体中电子的能量本征值$E(k)$与波矢$k$的关系是什么呢?这是比较复杂的,我们直接在\xref{tab:能带结构的三种图示}中给出图像,我们注意到
\begin{enumerate}
    \item 重复区表示法是最完整的图像,我们看到,晶体中的电子能量$E(k)$与波矢$k$间的关系有若干组$E_1(k), E_2(k), E_3(k), \cdots$,这些色散关系均具有周期性,并且,这些色散关系两两之间不重叠,而是有一定间隙,这些间隙就是禁带,而每个$E(k)$就给出了一个允带。
    \item 扩展区表示法是指,将各组$E(k)$分别对应到各个布里渊区,我们以$E_n(k)$对应第$n$布里渊区,这种表示法的好处是,我们可以看到,晶体中电子的$E(k)$关系的整体趋势上,其实很接近自由粒子$E(k)$的抛物线关系,但是,能量在布里渊区的边界$\pm n\pi/a$处会出现不连续,因此我们常说,\empx{每个布里渊区对应一个允带},\empx{禁带出现在布里渊区的边界}。\footnote{尽管能带图只是关于能量的一维关系,与波矢和布里渊区并没有什么关系。}
    \item 约化区表示法是指,将各组$E(k)$利用周期性简约在第一布里渊区$[-\pi/a,+\pi/a]$。
\end{enumerate}
晶体中的原子数$N$总是有限的,考虑周期性边界条件,波矢$k$只能分立取值
\begin{Equation}
    k=\frac{2\pi n}{Na}\qquad n=0,\pm 1,\pm 2,\cdots
\end{Equation}
因此在第一布里渊区中$k$只有$N$个分立的取值,但由于$N$很大,可以认为是准连续的。
\begin{TableLong}[能带结构的三种图示]{|c|}
<>()
< >( )
    \xcell<c>[1.5ex][0.0ex]{\includegraphics[scale=0.95]{build/Chapter01B_03.fig.pdf}}\\*
    \xgp[0ex][3ex]{重复区表示法}\\ \hlinemid
    \xcell<c>[1.5ex][0.0ex]{\includegraphics[scale=0.95]{build/Chapter01B_04.fig.pdf}}\\*
    \xgp[0ex][3ex]{扩展区表示法}\\ \hlinemid
    \xcell<c>[1.5ex][0.0ex]{\includegraphics[scale=0.95]{build/Chapter01B_05.fig.pdf}}\\*
    \xgp[0ex][3ex]{约化区表示法}\\ \hlinemid
\end{TableLong}

\subsection{能带与固体导电性}
固体能够导电,是固体中的电子在外电场的作用下做定向运动的结果。现在,让我们用能带理论来分析半导体是如何导电的,如\xref{fig:半导体的能带}所示。试想,所谓导电,电场力就必然会对电子有加速作用,这样一来,电子的能量势必就会发生变化,因此,在能带理论中,\empx{导电就意味着电子能在两个能级间跃迁},通常这种跃迁发生是一个能带中,故可导电的能带需要满足以下条件
\begin{itemize}
    \item 在该能带上,需要有电子存在。
    \item 在该能带上,需要有电子可以存在的空能级。
\end{itemize}
依照这样的标准,半导体的价带和导带似乎都不能导电,前者被电子填满没有空能级,后者则没有电子。但是半导体之所以区别于绝缘体,就是因为其价带和导带间的\uwave{禁带宽} $E_\text{g}$较窄,以至于在一定温度下,部分\uwave{价带顶} $E_\text{v}$的电子可以越过禁带热激发至\uwave{导带底} $E_\text{c}$。\footnote{这里$E_\text{g},E_\text{v},E_\text{c}$的下标分别指代Gap(带隙,禁带的别称)、Valence(价带)、Conduction(导带)。}这样,导带上就有了可以参与导电的电子,价带上也有了空的量子态,价带导电的过程可以视为这些空的量子态在价带中移动,我们将这些空的量子态称之为\uwave{空穴}(Electron Hole)。所以说,在半导体中,\empx{导带的电子和价带的空穴都参与导电},这与金属导体中仅有电子参与导电是不同的。

我们将价带电子激发为导带电子的过程,称为半导体的\uwave{本征激发}(Intrinsic Excitation)。

\goodbreak

至此,我们也可以引出引出能带与固体导电性的关系
\begin{itemize}
    \item 绝缘体即价带和导带间的禁带宽度较大,电子难以从价带热激发至导带。
    \item 半导体即价带和导带间的禁带宽度较小,电子可以从价带热激发至导带。
    \item 导体是比较特殊的,其价带和导带间是重叠的,事实上是同一能带,这样在一个能带上就同时有大量的电子和大量空的电子量子态,因此,导体具有极为良好的导电性。
\end{itemize}

\begin{Figure}[半导体的能带]
    \includegraphics[scale=1]{build/Chapter01B_06.fig.pdf}
\end{Figure}
硅和锗的禁带宽分别为$1.12\si{eV}$和$0.67\si{eV}$,是半导体,金刚石的禁带宽为$6$-$7\si{eV}$,是绝缘体。