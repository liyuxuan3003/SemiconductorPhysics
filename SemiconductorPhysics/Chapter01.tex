\chapter{半导体中的电子状态}

半导体具有许多独特的物理性质,这与半导体中电子的状态及其运动特点有密切关系。为了研究半导体的这些物理性质,本章将简要介绍半导体单晶材料中的电子状态及其运动规律。

半导体单晶材料和其他固态晶体一样,是由大量原子周期性重复排列而成,但这是一个非常复杂的多体问题,直接求解其薛定谔方程是无法做到的,只能采取近似的方法,称为\uwave{单电子近似}。所谓单电子近似,即假设每个电子是在周期性排列且固定不动的原子核势场及其他电子的平均势场中运动,并且势场是与晶格同周期的周期性势场。这种通过单电子近似法研究晶体中电子状态的理论,称为\uwave{能带理论}(Band Theorey)。能带的概念是半导体物理的核心。