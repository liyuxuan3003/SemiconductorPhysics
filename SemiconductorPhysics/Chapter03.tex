\section{半导体中的载流子的统计分布}
在一定的温度下,如果没有其他外界作用,半导体中的导电电子和空穴是依靠电子的热激发作用而产生的,电子从不断热振动的晶格中获得一定的能量,就可能从低能量的量子态跃迁到高能量的量子态,例如,电子从价带跃迁到导带,形成导带电子和价带空穴。除了这种本征激发的方式,通过杂质电离亦可以引入导带电子和价带空穴。这些我们已经很熟悉了,但与此同时,还有一种相反的过程,即电子当然也可以从高能量的量子态跃迁到低能量的量子态,并向晶格放出一定能量,这将使电子与空穴复合而减少,这一过程称为\uwave{载流子的复合}(Carrier Recombination),这与先前\uwave{载流子的产生}(Carrier Generation)是两个相反的过程,在一定温度下,这两个相反的过程将建立动态平衡,称为\uwave{热平衡状态}。这时,半导体中的电子浓度和空穴浓度都将保持一个稳定的数值,这种热平衡状态下的电子和空穴称为\uwave{热平衡载流子}。

实践表明,半导体的导电性强烈的随温度而变化,实际上,这种变化主要是由于半导体载流子浓度随温度变化而造成的。因此,我们很有必要先探究载流子浓度随温度变化的规律,这也就是本章的中心问题,即载流子的统计分布。为此,我们需要两方面的知识
\begin{enumerate}
    \item 电子可以存在的量子态如何分布。
    \item 电子在其可以存在的量子态上如何分布。
\end{enumerate}
前者将以状态密度描述,后者将以费米--狄拉克分布或玻尔兹曼分布描述。