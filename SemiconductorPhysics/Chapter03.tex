\chapter{半导体中载流子的统计分布}
在一定的温度下,如果没有其他外界作用,半导体中的导电电子和空穴是依靠电子的热激发作用而产生的,电子从不断热振动的晶格中获得一定的能量,就可能从低能量的量子态跃迁到高能量的量子态,例如,电子从价带跃迁到导带,形成导带电子和价带空穴。除了这种本征激发的方式,通过杂质电离亦可以引入导带电子和价带空穴。这些我们已经很熟悉了,但与此同时,还有一种相反的过程,即电子当然也可以从高能量的量子态跃迁到低能量的量子态,并向晶格放出一定能量,这将使电子与空穴复合而减少,这一过程称为\uwave{载流子的复合}(Carrier Recombination),这与先前\uwave{载流子的产生}(Carrier Generation)是两个相反的过程,在一定温度下,这两个相反的过程将建立动态平衡,称为\uwave{热平衡状态}。这时,半导体中的电子浓度和空穴浓度都将保持一个稳定的数值,这种热平衡状态下的电子和空穴称为\uwave{热平衡载流子}。

实践表明,半导体的导电性强烈的随温度而变化,实际上,这种变化主要是由于半导体载流子浓度随温度变化而造成的。因此,我们很有必要先探究载流子浓度随温度变化的规律,这也就是本章的中心问题,即载流子的统计分布。为此,我们需要两方面的知识
\begin{enumerate}
    \item 电子可以存在的量子态如何分布。
    \item 电子在其可以存在的量子态上如何分布。
\end{enumerate}
前者将以状态密度描述,后者将以费米--狄拉克分布或玻尔兹曼分布描述。

\section{状态密度}

\subsection{状态密度的定义}
在半导体的导带和价带中,有很多能级的存在。但相邻能级的间隔很小,约为$10^{-23}$\si{eV}的数量级,因此,可以认为能级是准连续的。故可以用状态密度描述各个能值处状态数的疏密。
\begin{BoxDefinition}[状态密度]
    定义\uwave{状态密度}(Density of States),为单位能量间隔的量子态的数目
    \begin{Equation}
        g(E)=\dv{Z}{E}
    \end{Equation}
\end{BoxDefinition}

暂且,先让我们放下状态密度,而是先研究一下状态本身是如何分布于$\vb*{k}$空间的。

我们知道,受限于晶体边界的限制,事实上$\vb*{k}$只能取一系列的分立值
\begin{Align}[12pt]
    k_x&=\frac{2\pi n_x}{L}&(n_x&=0,\pm 1,\pm 2,\cdots)\\
    k_y&=\frac{2\pi n_y}{L}&(n_y&=0,\pm 1,\pm 2,\cdots)\\
    k_z&=\frac{2\pi n_z}{L}&(n_z&=0,\pm 1,\pm 2,\cdots)
\end{Align}
其中$n_x,n_y,n_z$是整数,而$L$是晶体的线度,因此
\begin{Equation}
    V=L^3
\end{Equation}
就是晶体的体积。显然在$\vb*{k}$空间,每一组整数$(n_x,n_y,n_z)$都将对应波矢$\vb*{k}$的一个可行的取值$(k_x,k_y,k_z)$,或者说,对应$\vb*{k}$空间中的一个点$(k_x,k_y,k_z)$,而$\vb*{k}$空间中的每个可取的点就代表要给可行的量子态。由于任意代表点的坐标,沿三条坐标轴的方向均为$2\pi/L$的整数倍,所以代表点在$\vb*{k}$空间中是均匀分布的,每一个代表点都可以具有$8\pi^3/L^3=8\pi^3/V$立方体积。

因此,在$\vb*{k}$空间中电子的能量状态密度是$V/8\pi^3$,由于每个能量上可以存在两个自旋方向相反的电子,所以,在$\vb*{k}$空间中电子的状态密度就是$V/4\pi^3$,是均匀的。但需要注意的是,此处求出的$V/4\pi^3$的“状态密度”与\xref{def:状态密度}中$g(E)$的“状态密度”是两回事,\empx{前者是单位体积的状态,后者是单位能量区间的状态}。不过,这两者间是有联系的,因为,能值在$\vb*{k}$空间中表现为等能面的形式,该等能面上的量子态都具有相应能值,换言之,单位能量区间的状态数实际上就是$\vb*{k}$空间中一个等能面薄层中的状态数。这也就是下面我们求解$g(E)$的思路。

\subsection{状态密度的计算}\setpeq{状态密度的计算}

下面我们来推导半导体能带极值附近的状态密度,简单起见,姑且考虑导带底,并假设导带底位于$\vb*{k}=\vb*{0}$且等能面为球面的情况,根据\xref{subsec:有效质量的引入}中的相关公式,能量--波矢关系为
\begin{Equation}&[1]
    E(\vb*{k})=E_\text{c}+\frac{\hbar^2k^2}{2\mne}
\end{Equation}
在$\vb*{k}$空间中,以$\abs{\vb*{k}}$为半径作一球面,它就是能量为$E(\vb*{k})$的等能面,而要计算$E$与$E+\dd{E}$间的量子态数,就相当于计算半径$\abs{\vb*{k}}$至$\abs{\vb*{k}+\dd{\vb*{k}}}$的球壳间的量子态数,而这两个球壳间的体积是$4\pi k^2\dd{k}$,而$\vb*{k}$空间中量子态密度是$V/4\pi^3$,故$E$至$E+\dd{E}$间的量子态数$\dd{Z}$为
\begin{Equation}&[2]
    \dd{Z}=\frac{V}{4\pi^3}\times 4\pi k^2\dd{k}=\frac{V}{\pi^2}k^2\dd{k}
\end{Equation}
而由\xrefpeq{1}可以解得
\begin{Equation}&[3]
    k^2=\frac{2\mne(E-E_\text{c})}{\hbar^2}
\end{Equation}
即
\begin{Equation}&[4]
    k=\frac{(2\mne)^{1/2}(E-E_\text{c})^{1/2}}{\hbar}
\end{Equation}
就\xrefpeq{4}求微分
\begin{Equation}&[5]
    \dd{k}=\frac{(2\mne)^{1/2}(E-E_\text{c})^{-1/2}}{2\hbar}\dd{E}
\end{Equation}
这样一来,将\xrefpeq{3}和\xrefpeq{5}代入\xrefpeq{2}
\begin{Equation}
    \dd{Z}=\frac{V}{2\pi^2}\frac{(2\mne)^{3/2}}{\hbar^3}(E-E_\text{c})^{1/2}\dd{E}
\end{Equation}
而根据\fancyref{def:状态密度}
\begin{Equation}
    g_\text{c}(E)=\dv{Z}{E}=\frac{V}{2\pi^2}\frac{(2\mne)^{3/2}}{\hbar^3}(E-E_\text{c})^{1/2}
\end{Equation}
这就表明,导带底附近的状态密度,随电子能量的增加以平方根关系增大。

\begin{Figure}[状态密度函数]
    \includegraphics[scale=0.85]{build/Chapter03A_01.fig.pdf}
\end{Figure}

而对于实际的半导体硅和锗而言,情况要比上述讨论的复杂很多,主要问题在于硅和锗的导带等能面并不是简单的球面,而是若干中心$\vb*{k}\neq\vb*{0}$对称分布的旋转椭球面,换言之,我们有两个麻烦,其一是等能面发生了变形,由球面变为了旋转椭球面,其二是导带底不只是一个状态,对于硅是$6$个,对于锗是$4$个\footnote{虽然锗有$8$个沿$\<1 1 1>$对称分布的旋转椭球面,但每个只有一半在布里渊区内,故实际只计为$4$个。},所幸的是,这两个麻烦都可以通过$\mne$重新表述解决。

\begin{BoxFormula}[导带底的状态密度]
    硅和锗在导带底的状态密度为
    \begin{Equation}
        g_\text{c}(E)=\frac{V}{2\pi^2}\frac{(2\mne)^{3/2}}{\hbar^3}(E-E_\text{c})^{1/2}
    \end{Equation}
    其中$\mne$为
    \begin{Equation}
        \mne=m_\text{dn}=s^{2/3}(m_lm_t^2)^{1/3}
    \end{Equation}
    其中$m_\text{dn}$称为导带底电子状态密度的有效质量,而$s$是对称状态的数目。
\end{BoxFormula}

需要说明的是,我们其实不必区分“有效质量”和“状态密度有效质量”的提法,因为对于硅和锗的导带电子而言,其实原本也就没有什么有效质量$\mne$的提法,只有横向有效质量$m_t$和纵向有效质量$m_l$的提法,故这里$m_t,m_l$给出的$m_\text{dn}$其实就可以视为对硅和锗$\mne$的定义。

实际上,价带顶的情况和导带底的情况是相似的,价带顶主要起作用的是极值重合的重空穴和轻空穴,如\xref{tab:硅的价带结构}所示,价带顶同样有等能面变形和存在多个极值的问题,因此价带空穴质量$\mpe$也不是简单常数,同样需用重空穴有效质量$(m_\text{p})_\text{h}$和轻空穴有效质量$(m_\text{p})_\text{l}$重表述。
\begin{BoxFormula}[价带顶的状态密度]*
    硅和锗在价带顶的状态密度为
    \begin{Equation}
        g_\text{v}(E)=\frac{V}{2\pi^2}\frac{(2\mpe)^{3/2}}{\hbar^3}(E_\text{v}-E)^{1/2}
    \end{Equation}
    其中$\mne$为
    \begin{Equation}
        \mpe=m_\text{dp}=\qty[(m_\text{p})_\text{l}^{3/2}+(m_\text{p})_\text{h}^{3/2}]^{2/3}
    \end{Equation}
    其中$m_\text{dp}$称为价带顶空穴状态密度的有效质量。
\end{BoxFormula}
在\xref{tab:半导体的状态密度有效质量}中,结合\xref{tab:硅和锗的载流子有效质量}以及硅$s=6$和锗$s=4$,列出了硅和锗等的$\mne$和$\mpe$
\begin{Table}[半导体的状态密度有效质量]{c|cccc|ccc|c}
    <
    \mrx<c>{3}{半导体材料}&\mc{4}(c|){电子}&\mc{3}(c|){空穴}&\mrx<c>{2}{比值}\\
    &纵向&横向&对称状态&有效质量&重&轻&有效质量&\\
    &$m_l$&$m_t$&$s$&$\mne$&$(m_\text{p})_\text{h}$&$(m_\text{p})_\text{l}$&$\mpe$&$\mne/\mpe$\\
    >
    硅&$0.98m_0$&$0.19m_0$&$6$&$1.06m_0$&$0.53m_0$&$0.16m_0$&$0.59m_0$&$0.59$\\
    锗&$1.64m_0$&$0.08m_0$&$4$&$0.56m_0$&$0.28m_0$&$0.044m_0$&$0.29m_0$&$0.52$\\
    砷化镓&--&--&$1$&$0.063m_0$&$0.50m_0$&$0.076m_0$&$0.52m_0$&$8.25$\\
    锑化铟&--&--&$1$&$0.012m_0$&$0.44m_0$&$0.016m_0$&$0.44m_0$&$36.7$\\
\end{Table}
在\xref{fig:状态密度函数}中,以红线和蓝线分别绘制了价带顶状态密度$g_\text{v}(E)$和导带底状态密度$g_\text{c}(E)$的曲线图像,由此可以看出,价带中能量越低,状态密度越大,导带中能量越高,状态密度越大。