\documentclass{xStandalone}

\begin{document}
\begin{tikzpicture}

\tikzset{dis/.style={thin,<->,gray}}

\path
    (0,+2) coordinate   (A1)
    ++(8,0) coordinate  (A2)
    (0,-2) coordinate   (B1)
    ++(8,0) coordinate  (B2);

\fill[red!10!white]  (A1) rectangle ($(A2)+(0,+1.5)$) coordinate (A2');
\fill[blue!10!white] (B1) rectangle ($(B2)+(0,-1.5)$) coordinate (B2');

\path ($(A1)!0.5!(A2')$) ++(0,+0.2) node {\textbf{导带}};
\path ($(B1)!0.5!(B2')$) ++(0,-0.2) node {\textbf{价带}};

\draw[thick] (A1)--(A2) node[right] {$E_\text{c}$};
\draw[thick] (B1)--(B2) node[right] {$E_\text{v}$};

\foreach \x in {0,1,2,...,7}
{
    \path (\fpeval{\x+0.5},+1) coordinate (P\x);
    \draw 
    ($(P\x)+(-0.25,0)$) -- 
    ($(P\x)+(+0.25,0)$);
}

\foreach \x in {2,5}
{
    \path (\fpeval{\x+0.5},-1) coordinate (Q\x);
    \draw 
    ($(Q\x)+(-0.25,0)$) -- 
    ($(Q\x)+(+0.25,0)$);
}

\foreach \x in {0,1,3,4,6,7}
{
    \draw[-latex] (P\x) ++(0,+0.2) node[circ] {} ++(0,+0.2) -- ++(0,1) coordinate (C\x); 
}

\foreach \x in {2,5}
{
    \draw[-latex] (P\x) ++(0,-0.2) node[circ] {} ++(0,-0.2) -- ++(0,-1.1) coordinate (C\x) node[below=0.2cm,ocirc] {} ; 
}

\path (P7-|A2)  node[right] {$E_\text{D}$};

\path (Q5-|A2)  node[right] {$E_\text{A}$};

\end{tikzpicture}
\end{document}