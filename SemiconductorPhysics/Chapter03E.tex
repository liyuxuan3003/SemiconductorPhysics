\section{杂质半导体的载流子浓度}
\uwave{杂质半导体}(Extrinsic Semiconductor)是指人为引入杂质后的半导体,杂质的电离将会影响原有的载流子分布。但是,杂质并不是什么奇妙的魔法,如\xref{fig:杂质半导体的能带图}所示,杂质半导体的状态密度$g(E)$与本征半导体完全一致,杂质半导体的概率密度$f(E)$仍然遵从费米分布,那么到底什么被改变了?使杂质型半导体,表现出与本征半导体不同的载流子浓度?答案是费米能级
\begin{itemize}
    \item N型半导体的费米能级较靠近导带底,使导带电子浓度增加,使价带空穴减小。
    \item P型半导体的费米能级较靠近价带顶,使价带空穴浓度增加,使导带电子减小。
\end{itemize}
\begin{Figure}[杂质半导体的能带图]
    \begin{FigureSub}[N型半导体]
        \includegraphics[width=0.95\linewidth]{build/Chapter03D_04.fig.pdf}
    \end{FigureSub}\vspace{0.5cm}
    \begin{FigureSub}[P型半导体]
        \includegraphics[width=0.95\linewidth]{build/Chapter03D_05.fig.pdf}
    \end{FigureSub}
\end{Figure}
那么,为什么掺杂会导致费米能级的行为发生变化呢?根据\xref{sec:本征半导体的载流子浓度},载流子的电中性条件决定费米能级,载流子的浓度又反过来由费米能级确定。所以,可以肯定的是,\empx{掺杂会以某种方式干预了电中性条件},尽管我们无法直接确定杂质会如何影响浓度分布,但是,掺入了多少杂质,掺入的杂质中有多少会发生电离,这是可以理论计算的,由此就能得到新的电中性条件。

\subsection{杂质能级上的电子和空穴}
那么首先要解决的问题是,掺入的杂质中有多少比例发生电离?其实,我们不妨换一个问法
\begin{itemize}
    \item 电子占据施主能级的概率$f_\text{D}(E_\text{D})$为多少?
    \item 空穴占据受主能级的概率$f_\text{A}(E_\text{A})$为多少?
\end{itemize}
当电子或空穴占据施主能级或受主能级时,这就是杂质没有电离的状态,我们关心的是杂质发生电离,向导带或价带输送电子或空穴的概率,这分别由$1-f_\text{D}(E_\text{D})$和$1-f_\text{A}(E_\text{A})$给出。

但现在的问题是,我们知道,费米分布$f(E)$事实上描述了电子占据某个能级的概率密度,那么,电子占据施主能级的概率$f_\text{D}(E_\text{D})$,空穴占据受主能级的概率$f_\text{A}(E_\text{A})$,是否也是服从费米分布的呢?遗憾的是,答案是否定的。因为杂质能级与能带中的能级是有区别,正如我们在\xref{subsec:施主杂质和受主杂质}所说,杂质能级是一系列具有相同能量的孤立能级,但是,其不同于能带中的能级可以由两个自旋相反的电子占有,杂质能级上只能进入一个任意自旋的电子,这是因为杂质原子也只能额外吸引一个电子,所以,\empx{杂质能级不服从费米分布}。有理论可以证明
\begin{BoxFormula}[施主能级的概率]
    电子占据施主能级的概率是
    \begin{Equation}
        f_\text{D}(E)=\frac{1}{1+g_\text{D}^{-1}\exp[(E_\text{D}-E_\text{F})/\kB T]}
    \end{Equation}
    其中,$g_\text{D}$是\uwave{施主能级的基态简并度}。
\end{BoxFormula}
\begin{BoxFormula}[受主能级的概率]
    空穴占据受主能级的概率是
    \begin{Equation}
        f_\text{A}(E)=\frac{1}{1+g_\text{A}^{-1}\exp[(E_\text{F}-E_\text{A})/\kB T]}
    \end{Equation}
    其中,$g_\text{A}$是\uwave{受主能级的基态简并度}。
\end{BoxFormula}
通常而言,对于硅、锗、砷化镓等材料,通常会取基态简并度(简并因子)为
\begin{Equation}
    g_\text{D}=2\qquad
    g_\text{A}=4
\end{Equation}
如果我们分别记施主浓度和受主浓度为$N_\text{D}$和$N_\text{A}$,那么就可以写出相关的浓度了,这里需要说明的是,所谓“电离施主浓度”和“电离受主浓度”,其实就是指由杂质发生电离,进入导带和价带的电子和空穴的浓度,它们等于相应杂质离子的浓度,故$n_\text{D}^{+}$和$p_\text{A}^{-}$就是我们所关心的。
\begin{BoxFormula}[施主浓度]
    施主能级上的电子浓度$n_\text{D}$和电离施主浓度$n_\text{D}^{+}$分别为
    \begin{Equation}
        n_\text{D}=N_\text{D}f_\text{D}(E_\text{D})\qquad 
        n_\text{D}^{+}=N_\text{D}[1-f_\text{D}(E_\text{D})]
    \end{Equation}
    其中$N_\text{D}$为掺入的施主杂质的浓度。
\end{BoxFormula}\nopagebreak
\begin{BoxFormula}[受主浓度]
    受主能级上的空穴浓度$p_\text{A}$和电离受主浓度$p_\text{A}^{-}$分别为
    \begin{Equation}
        p_\text{A}=N_\text{A}f_\text{A}(E_\text{A})\qquad 
        p_\text{A}^{-}=N_\text{A}[1-f_\text{A}(E_\text{A})]
    \end{Equation}
    其中$N_\text{A}$为掺入的受主杂质的浓度。
\end{BoxFormula}

\subsection{杂质半导体的电中性条件}
简洁起见,自本小节开始,我们都姑且以仅包含施主能级的N型半导体为例讨论。

杂质半导体中的杂质对电中性条件有何种影响?对于N型半导体,负电荷显然是由价带电子浓度$n_0$构成,这里面同时包含了本征激发和杂质电离的电子。正电荷则包含两部分,一部分来自导带空穴浓度$p_0$,一部分来自电离施主浓度$n_\text{D}^{+}$(考虑到电离后的施主离子带正电荷)。
\begin{BoxFormula}[杂质半导体的电中性条件]
    N型杂质半导体的电中性条件满足
    \begin{Equation}
        n_0=n_\text{D}^{+}+p_0
    \end{Equation}
\end{BoxFormula}
代入\fancyref{fml:导带电子浓度}、\fancyref{fml:价带空穴浓度}、\fancyref{fml:施主浓度}
\begin{Equation}
    \qquad\qquad
    N_\text{c}\exp(\frac{E_\text{F}-E_\text{c}}{\kB T})=
    N_\text{v}\exp(\frac{E_\text{v}-E_\text{F}}{\kB T})+
    \frac{N_\text{D}}{1+g_\text{D}\exp(E_\text{F}-E_\text{D}/\kB T)}
    \qquad\qquad
\end{Equation}
或许有些不自量力的人\footnote{没错是我……曾试图用Mathematica求解这个方程,但或许是由于技术不够娴熟,得到的解不知为何无法绘制出任何曲线,成功浪费了几个小时,一无所获,很郁闷。虽然如此,但确实很好奇这里费米能级$E_\text{F}$的精确解绘制成函数曲线,究竟会是何种样貌?}试图求解上面这个方程以得到$E_\text{F}$的表达式,事实上,经过恰当的变量代换,该方程大概可以转化为一个四次方程,而总所周知,四次方程的解是令人发指般的复杂,其实不利于我们把握简洁的物理图景。因而,我们需要做一些近似,我们会将$E_\text{F}(T)$的函数关系划分为若干个温度区间,在每个温度区间内适用相应的近似条件,从而简化电中性条件的表达式,由此得到若干适用不同温度范围的$E_\text{F}(T)$的表达式。这就是接下来我们要做的。

\subsection{低温弱电离区}
当温度很低时,大部分施主杂质能级仍然为电子所占据,只有很少量的施主杂质发生电离,这些极少量的电子进入导带,称为弱电离。而这种情况下,本征激发的电子就更加可以忽略了。
\begin{BoxDefinition}[低温弱电离区]
    \uwave{低温弱电离区}是指,施主杂质少量电离,本征激发可以忽略
    \begin{Equation}
        n_\text{D}^{+}\ll N_\text{D}\qquad
        p_0=0
    \end{Equation}
\end{BoxDefinition}

\begin{BoxFormula}[低温弱电离区的电中性条件]
    低温弱电离区适用的电中性条件为
    \begin{Equation}
        n_0=n_\text{D}^{+}
    \end{Equation}
    更具体的
    \begin{Equation}
        N_\text{c}\exp(\frac{E_\text{F}-E_\text{c}}{\kB T})=\frac{1}{g_\text{D}}N_\text{D}\exp(\frac{E_\text{D}-E_\text{F}}{\kB T})
    \end{Equation}
\end{BoxFormula}
\begin{Proof}
    根据\fancyref{fml:杂质半导体的电中性条件}
    \begin{Equation}&[1]
        n_0=n_\text{D}^{+}+p_0
    \end{Equation}
    根据\fancyref{def:低温弱电离区}中的$p_0=0$
    \begin{Equation}&[2]
        n_0=n_\text{D}^{+}
    \end{Equation}
    代入\fancyref{fml:导带电子浓度}和\fancyref{fml:施主浓度}
    \begin{Equation}&[3]
        N_\text{c}\exp(\frac{E_\text{F}-E_\text{c}}{\kB T})=\frac{N_\text{D}}{1+g_\text{D}\exp(E_\text{F}-E_\text{D}/\kB T)}
    \end{Equation}
    根据\fancyref{def:低温弱电离区}中的$n_0\ll N_\text{D}$,推定\xrefpeq{3}右端指数项必远大于$1$,故
    \begin{Equation}&[4]
        N_\text{c}\exp(\frac{E_\text{F}-E_\text{c}}{\kB T})=\frac{N_\text{D}}{g_\text{D}\exp(E_\text{F}-E_\text{D}/\kB T)}
    \end{Equation}
    因此
    \begin{Equation}*
        N_\text{c}\exp(\frac{E_\text{F}-E_\text{c}}{\kB T})=\frac{1}{g_\text{D}}N_\text{D}\exp(\frac{E_\text{D}-E_\text{F}}{\kB T})\qedhere
    \end{Equation}
\end{Proof}

而有了电中性条件后,计算费米能级就不是太困难的事情了。
\begin{BoxFormula}[低温弱电离区的费米能级]
    低温弱电离区的费米能级满足
    \begin{Equation}
        E_\text{F}=\frac{E_\text{c}+E_\text{D}}{2}+\qty(\frac{\kB T}{2})\ln\qty(\frac{N_\text{D}}{g_\text{D}N_\text{c}})
    \end{Equation}
\end{BoxFormula}
\begin{Proof}
    根据\fancyref{fml:低温弱电离区的电中性条件}
    \begin{Equation}&[1]
        N_\text{c}\exp(\frac{E_\text{F}-E_\text{c}}{\kB T})=\frac{1}{g_\text{D}}N_\text{D}\exp(\frac{E_\text{D}-E_\text{F}}{\kB T})
    \end{Equation}
    两端同时取对数
    \begin{Equation}&[2]
        \ln N_\text{c}+\frac{E_\text{F}-E_\text{c}}{\kB T}=\ln\frac{N_\text{D}}{g_\text{D}}+\frac{E_\text{D}-E_\text{F}}{\kB T}
    \end{Equation}
    移项整理
    \begin{Equation}&[3]
        \frac{2E_\text{F}-E_\text{c}-E_\text{D}}{\kB T}=\ln\frac{N_\text{D}}{g_\text{D}N_\text{c}}
    \end{Equation}
    再整理
    \begin{Equation}&[4]
        \frac{2E_\text{F}}{\kB T}=\frac{E_\text{c}+E_\text{D}}{\kB T}+\ln\frac{N_\text{D}}{g_\text{D}N_\text{c}}
    \end{Equation}
    即得
    \begin{Equation}*
        E_\text{F}=\frac{E_\text{c}+E_\text{D}}{2}+\qty(\frac{\kB T}{2})\ln\frac{N_\text{D}}{g_\text{D}N_\text{c}}\qedhere
    \end{Equation}
\end{Proof}

低温弱电离区的费米能级与温度的关系比较复杂(因为$N_\text{c}$同样与温度有关),分析如下。
\begin{BoxProperty}[低温弱电离区的费米能级的性质]
    低温弱电离区的费米能级$E_\text{F}(T)$随温度增加,先增大,后减小。

    当$T\to 0\si{K}$时,费米能级趋近导带底和杂质能级的中线
    \begin{Equation}
        \Lim[T\to 0]E_\text{F}(T)=\frac{E\text{c}+E_\text{D}}{2}
    \end{Equation}
    在$N_\text{c}$满足下左式时(此时$T$满足下右式),费米能级取最大值
    \begin{Equation}
        N_\text{c}=\qty(\frac{N_\text{D}}{g_\text{D}})\e^{-3/2}\qquad
        T=\frac{2\pi\hbar^2}{\mne\kB\e}\qty(\frac{N_\text{D}}{2g_\text{D}})^{2/3}
    \end{Equation}
    在$N_\text{c}$满足下左式时(此时$T$满足下右式),费米能级回落到中线
    \begin{Equation}
        N_\text{c}=\qty(\frac{N_\text{D}}{g_\text{D}})\qquad
        T=\frac{2\pi\hbar^2}{\mne\kB}\qty(\frac{N_\text{D}}{2g_\text{D}})^{2/3}
    \end{Equation}
\end{BoxProperty}

\begin{Proof}
    \paragraph{讨论绝对零度时的结论}
    根据\fancyref{fml:低温弱电离区的费米能级}
    \begin{Equation}&[1]
        E_\text{F}=\frac{E_\text{c}+E_\text{D}}{2}+\qty(\frac{\kB T}{2})\ln\qty(\frac{N_\text{D}}{g_\text{D}N_\text{c}})
    \end{Equation}
    根据\fancyref{fml:导带电子浓度}
    \begin{Equation}&[2]
        N_\text{c}=2\qty(\frac{\mne\kB T}{2\pi\hbar^2})^{3/2}
    \end{Equation}
    这就是说
    \begin{Equation}&[3]
        N_\text{c}\propto T^{3/2}
    \end{Equation}
    将\xrefpeq{3}中$N_\text{c}$代回\xrefpeq{1},注意到$T\to 0$时$T\ln(T^{-3/2})\to 0$,故
    \begin{Equation}&[4]
        \Lim[T\to 0]E_\text{F}(T)=\frac{E_\text{c}+E_\text{D}}{2}
    \end{Equation}
    \paragraph{讨论费米能级取何时最大值}
    接下来,我们通过对$E_\text{F}$求导来分析$E_\text{F}$的性质
    \begin{Equation}&[5]
        \dv{E_\text{F}}{T}=\frac{\kB}{2}\ln\qty(\frac{N_\text{D}}{g_\text{D}N_\text{c}})+\frac{\kB T}{2}\dv{T}\ln\qty(\frac{N_\text{D}}{g_\text{D}N_\text{c}})
    \end{Equation}
    运用复合函数的导数法则
    \begin{Equation}&[6]
        \dv{E_\text{F}}{T}=\frac{\kB}{2}\ln\qty(\frac{N_\text{D}}{g_\text{D}N_\text{c}})+\frac{\kB T}{2}\qty[\frac{g_\text{D}N_\text{c}}{N_\text{D}}\cdot\frac{N_\text{D}}{g_\text{D}}\cdot\dv{T}\qty(\frac{1}{N_\text{c}})]
    \end{Equation}
    化简整理得
    \begin{Equation}&[7]
        \dv{E_\text{F}}{T}=\frac{\kB}{2}\ln\qty(\frac{N_\text{D}}{g_\text{D}N_\text{c}})+\frac{\kB T}{2}\qty[N_\text{c}\dv{T}\qty(\frac{1}{N_\text{c}})]
    \end{Equation}
    而根据$1/N_\text{c}\propto T^{-3/2}$
    \begin{Equation}&[8]
        T\dv{T}\qty(\frac{1}{N_\text{c}})=-\frac{3}{2}\qty(\frac{1}{N_\text{c}})
    \end{Equation}
    将\xrefpeq{8}代回\xrefpeq{7}
    \begin{Equation}&[9]
        \dv{E_\text{F}}{T}=\frac{\kB}{2}\ln\qty(\frac{N_\text{D}}{g_\text{D}N_\text{c}})-\frac{\kB}{2}\cdot\frac{3}{2}
    \end{Equation}
    即
    \begin{Equation}&[10]
        \dv{E_\text{F}}{T}=\frac{\kB}{2}\qty[\ln\qty(\frac{N_\text{D}}{g_\text{D}N_\text{c}})-\frac{3}{2}]
    \end{Equation}
    我们注意到,如果$N_\text{c}$满足下式,一阶导数$\dv*{E_\text{F}}{T}$为零
    \begin{Equation}&[11]
        \ln(\frac{N_\text{D}}{g_\text{D}N_\text{c}})=\frac{3}{2}\qquad N_\text{c}=\qty(\frac{N_\text{D}}{g_\text{D}})\e^{-3/2}
    \end{Equation}
    依据\xrefpeq{2}不难求出此时$T$应当满足的值。

    \paragraph{讨论费米能级何时回落至中线}
    最后的问题是容易,让我们回归\xrefpeq{1},注意到$N_\text{c}$满足下式时有$E_\text{F}=(E_\text{c}+E_\text{D})/2$
    \begin{Equation}
        \ln(\frac{N_\text{D}}{g_\text{D}N_\text{c}})=0\qquad N_\text{c}=\qty(\frac{N_\text{D}}{g_\text{D}})
    \end{Equation}
    依据\xrefpeq{2}不难求出此时$T$应当满足的值。
\end{Proof}

\xref{fig:杂质半导体的费米能级}形象的展现了低温电离区费米能级如何随温度变化
\begin{Figure}[杂质半导体的特性]
    \begin{FigureSub}[杂质半导体的费米能级]
        \includegraphics[scale=0.85]{build/Chapter03D_01.fig.pdf}
    \end{FigureSub}
    \vspace{0.35cm}
    \begin{FigureSub}[杂质半导体的载流子浓度]
        \includegraphics[scale=0.85]{build/Chapter03D_02.fig.pdf}
    \end{FigureSub}
\end{Figure}
\xref{fig:杂质半导体的费米能级}中分别用$T_1,T_2$标识\xref{ppt:低温弱电离区的费米能级的性质}中,费米能级取最大值和回落到中线时的温度。

\xref{fig:杂质半导体的费米能级}在绘图时,取有效质量比$\mpe/\mne\approx 0.5$,取禁带宽度$E_\text{g}=1.0\si{eV}$,这是接近硅和锗的参数值,由此计算出的室温$300\si{K}$下的$N_\text{c}$在$1\times 10^{19}\si{cm^{-3}}$的数量级,而通常来说,作为非简并半导体,掺杂的浓度$N_\text{D}$应当满足$N_\text{D}\ll N_\text{c}$\;。但是,由于这里我们希望绘制一张能全面反映各个温度区段的费米能级变化趋势的示意图像,如果真的取$N_\text{D}\ll N_\text{c}$,低温弱电离区的特性在整体尺度上将完全不可见,例如,如果我们取$N_\text{D}=1\times 10^{16}\si{cm^{-3}}$,则$T_1,T_2$的值甚至不足$1\si{K}$,并且低温弱电离区$E_\text{F}$的最大值与$(E_\text{c}+E_\text{D})/2$的中线的间距几乎为零,使低温弱电离区的特性都无法被体现。为了凸显低温弱电离区的特性,我们被迫将掺杂浓度提高到了不合理的$N_\text{c}=1\times 10^{20}\si{cm^{-3}}$,很明显,这导致了诸多荒谬。首先,这个掺杂浓度不再符合非简并半导体的近似假设,需要适用简并半导体的理论。其次,低温弱电离区的特性范围至少可以延伸到$300\si{K}$的室温,这是不符合实际的,更明显的是,本征特性需要在高达$4000\si{K}$左右的高温才能显现,这个温度下任何半导体都已经融化了。但总之,作为一张示意图,它是成功的。

现在,有了低温弱电离区的费米能级,我们就可以计算低温弱电离区的载流子浓度了。
\begin{BoxFormula}[低温弱电离区的载流子浓度]*
    低温弱电离区中,导带电子浓度满足
    \begin{Equation}
        n_0=\qty(\frac{N_\text{D}N_\text{c}}{g_\text{D}})^{1/2}\exp(-\frac{\delt{E_\text{D}}}{2\kB T})
    \end{Equation}
    低温弱电离区中,价带空穴浓度满足
    \begin{Equation}
        p_0=0
    \end{Equation}
\end{BoxFormula}
\begin{Proof}
    根据\fancyref{fml:导带电子浓度}
    \begin{Equation}
        n_0=N_\text{c}\exp(\frac{E_\text{F}-E_\text{c}}{\kB T})
    \end{Equation}
    代入\fancyref{fml:低温弱电离区的费米能级}
    \begin{Equation}
        n_0=N_\text{c}\exp[\frac{(E_\text{c}+E_\text{D})/2+(\kB T/2)\ln(N_\text{D}/g_\text{D}N_\text{c})-E_\text{c}}{\kB T}]
    \end{Equation}
    化简得到
    \begin{Equation}
        n_0=N_\text{c}\exp[\frac{E_\text{D}-E_\text{c}}{2\kB T}+\frac{1}{2}\ln(\frac{N_\text{D}}{g_\text{D}N_\text{c}})]
    \end{Equation}
    即
    \begin{Equation}
        n_0=N_\text{c}\exp(\frac{E_\text{D}-E_\text{c}}{2\kB T})\qty(\frac{N_\text{D}}{g_\text{D}N_\text{c}})^{1/2}
    \end{Equation}
    整理,应用$\delt{E_\text{D}}=E_\text{c}-E_\text{D}$
    \begin{Equation}
        n_0=\exp(-\frac{\delt{E_\text{D}}}{2\kB T})\qty(\frac{N_\text{D}N_\text{c}}{g_\text{D}})^{1/2}\qedhere
    \end{Equation}
    这就求得了$n_0$,而$p_0=0$是由\fancyref{def:低温弱电离区}直接得到的。
\end{Proof}
由于$N_\text{c}\propto T^{3/2}$,因此$n_0\propto T^{3/4}\exp(-1/T)$,当$T$较大时$n_0\propto T^{3/4}$,如\xref{fig:杂质半导体的载流子浓度}所示。

\subsection{高温强电离区}
当温度较高时,大部分施主杂质都发生电离,称为强电离,此时本征激发仍然可以忽略。

\begin{BoxDefinition}[高温强电离区]
    高温强电离区是指,施主杂质全部电离,本征激发可以忽略
    \begin{Equation}
        n_\text{D}^{+}=N_\text{D}\qquad
        p_0=0
    \end{Equation}
\end{BoxDefinition}

\begin{BoxFormula}[高温强电离区的电中性条件]
    高温强电离区的电中性条件为
    \begin{Equation}
        n_0=n_\text{D}^{+}
    \end{Equation}
    更具体的
    \begin{Equation}
        N_\text{c}\exp(\frac{E_\text{F}-E_\text{c}}{\kB T})=N_\text{D}
    \end{Equation}
\end{BoxFormula}
\begin{Proof}
    根据\fancyref{fml:杂质半导体的电中性条件}
    \begin{Equation}
        n_0=n_\text{D}^{+}+p_0
    \end{Equation}
    根据\fancyref{def:高温强电离区}中的$p_0=0$
    \begin{Equation}
        n_0=n_\text{D}^{+}
    \end{Equation}
    根据\fancyref{fml:导带电子浓度}和\fancyref{def:高温强电离区}的$n_\text{D}^{+}=N_\text{D}$
    \begin{Equation}*
        N_\text{c}\exp(\frac{E_\text{F}-E_\text{c}}{\kB T})=N_\text{D}\qedhere
    \end{Equation}
\end{Proof}
而有了电中性条件后,计算费米能级就不是太困难的事情了。
\begin{BoxFormula}[高温强电离区的费米能级]
    高温强电离区的费米能级满足
    \begin{Equation}
        E_\text{F}=E_\text{c}+\kB T\ln(\frac{N_\text{D}}{N_\text{c}})
    \end{Equation}
\end{BoxFormula}
\begin{Proof}
    根据\fancyref{fml:高温强电离区的电中性条件}
    \begin{Equation}&[1]
        N_\text{c}\exp(\frac{E_\text{F}-E_\text{c}}{\kB T})=N_\text{D}
    \end{Equation}
    两端同时取对数
    \begin{Equation}&[2]
        \ln N_\text{c}+\frac{E_\text{F}-E_\text{c}}{\kB T}=\ln N_\text{D}
    \end{Equation}
    移项整理得
    \begin{Equation}*
        E_\text{F}=E_\text{c}+\kB T\ln(\frac{N_\text{D}}{N_\text{c}})\qedhere
    \end{Equation}
\end{Proof}
如\xref{fig:杂质半导体的费米能级}所示,高温强电离区和低温弱电离区的曲线是相似的,这是因为,如果我们仔细对比\xref{fml:低温弱电离区的费米能级}和\xref{fml:高温强电离区的费米能级},我们会发现,强电离区和弱电离区中,费米能级的函数形式是一致的,只不过参数有些不同。不过,高温强电离区中我们并不关心随温度先增大后减小的那部分特性,因为那里的温度低于高温强电离区的近似适用范围。另外,注意到,高温强电离区的曲线和低温弱电离区的曲线是无法相接的,这是因为两者实质都是不考虑本征激发的电中性条件$n_0=n_\text{D}^{+}$的极限情况,低温弱电离区假定$n_\text{D}^{+}\ll N_\text{D}$,高温强电离区则假定$n_\text{D}^{+}=N_\text{D}$。

\begin{BoxFormula}[高温强电离区的载流子浓度]
    高温强电离区中,导带电子浓度满足
    \begin{Equation}
        n_0=N_\text{D}
    \end{Equation}
    高温强电离区中,价带空穴浓度满足
    \begin{Equation}
        p_0=0
    \end{Equation}
\end{BoxFormula}
\begin{Proof}
    由\fancyref{def:高温强电离区}和\fancyref{fml:高温强电离区的电中性条件}立即得到。
\end{Proof}
如\xref{fig:杂质半导体的特性}所示,高温强电离区的电子浓度是常数,始终等于杂质浓度$N_\text{D}$。

现在的问题是,尽管我们不能精确求解弱电离区和强电离区间的中间状态,但是我们尚可以通过一些分析,确定杂质何时可以视为完全电离,换言之,确定何时可以适用强电离区的模型。

\begin{BoxFormula}[未电离的施主浓度]*
    未电离的施主浓度$n_\text{D}$可以表示为
    \begin{Equation}
        n_\text{D}=D_{-}N_\text{D}
    \end{Equation}
    其中$D_{-}$为
    \begin{Equation}
        D_{-}=\qty(\frac{g_\text{D}N_\text{D}}{N_\text{c}})\exp(\frac{\delt{E_\text{D}}}{\kB T})
    \end{Equation}
\end{BoxFormula}

\begin{Proof}
    根据\fancyref{fml:施主浓度}和\fancyref{fml:施主能级的概率}
    \begin{Equation}
        n_\text{D}=N_\text{D}f_\text{D}(E_\text{D})=\frac{N_\text{D}}{1+g_\text{D}^{-1}\exp[(E_\text{D}-E_\text{F})/\kB T]}
    \end{Equation}
    当$E_\text{D}-E_\text{F}\gg \kB T$时\footnote{为什么高温强电离区能满足这个近似条件呢?}
    \begin{Equation}
        n_\text{D}=g_\text{D}N_\text{D}\exp(\frac{E_\text{F}-E_\text{D}}{\kB T})
    \end{Equation}
    将\fancyref{fml:高温强电离区的费米能级}代入
    \begin{Equation}
        n_\text{D}=g_\text{D}N_\text{D}\exp[\frac{E_\text{c}+\kB T\ln(N_\text{D}/N_\text{c})-E_\text{D}}{\kB T}]
    \end{Equation}
    即
    \begin{Equation}
        n_\text{D}=g_\text{D}N_\text{D}\exp(\frac{E_\text{c}-E_\text{D}}{\kB T})\frac{N_\text{D}}{N_\text{c}}
    \end{Equation}
    整理得
    \begin{Equation}
        n_\text{D}=\qty(\frac{g_\text{D}N_\text{D}}{N_\text{c}})\exp(\frac{\delt{E_\text{D}}}{\kB T})N_\text{D}
    \end{Equation}
    如果我们愿意记
    \begin{Equation}
        D_{-}=\qty(\frac{g_\text{D}N_\text{D}}{N_\text{c}})\exp(\frac{\delt{E_\text{D}}}{\kB T})
    \end{Equation}
    那么
    \begin{Equation}*
        n_\text{D}=D_{-}N_\text{D}\qedhere
    \end{Equation}
\end{Proof}
这里$N_\text{D}$和$n_\text{D}$分别代表施主杂质的浓度和未电离的施主浓度,因此$D_{-}$代表的就是未电离的施主浓度的百分比,\empx{通常认为完全电离的标准是$90\%$发生电离},因此$D_{-}=10\%$时计算出的温度,就应当是杂质全部电离时的温度,换言之,即可以适用高温强电离区的温度下限。

这里由$D_{-}$的表达式亦可以看出,在一定温度下
\begin{itemize}
    \item 掺杂浓度$N_\text{D}$越大,则$D_{-}$越大,降低至$D_{-}=10\%$即全部电离所需的温度就越高。
    \item 掺杂浓度$N_\text{D}$越小,则$D_{-}$越小,降低至$D_{-}=10\%$即全部电离所需的温度就越低。
\end{itemize}
作为具体计算达到完全电离时温度的手段,除了取定$D_{-}=10\%$,我们还需要将$N_\text{c}$代入。

\begin{BoxEquation}[杂质全部电离的温度方程]
    杂质全部电离时的温度由以下方程确定
    \begin{Equation}
        \frac{\delt{E_\text{D}}}{\kB}\qty(\frac{1}{T})=\frac{3}{2}\ln T+\ln\frac{2(\mne\kB/2\pi)^{3/2}D_{-}}{g_\text{D}N_\text{D}\hbar^3}
    \end{Equation}
    上式中$D_{-}$取为$10\%$,作为完全电离的标准。
\end{BoxEquation}

\begin{Proof}
    根据\fancyref{fml:未电离的施主浓度}中$D_{-}$的表达式
    \begin{Equation}
        \exp(\frac{\delt{E_\text{D}}}{\kB T})=\frac{N_\text{c}D_{-}}{g_\text{D}N_\text{D}}
    \end{Equation}
    两端同时取对数
    \begin{Equation}
        \frac{\delt{E_\text{D}}}{\kB T}=\ln\frac{N_\text{c}D_{-}}{g_\text{D}N_\text{D}}
    \end{Equation}
    代入\fancyref{fml:导带电子浓度}中的$N_\text{c}$
    \begin{Equation}
        \frac{\delt{E_\text{D}}}{\kB T}=\ln\frac{2(\mne\kB T/2\pi\hbar^2)^{3/2}D_{-}}{g_\text{D}N_\text{D}}
    \end{Equation}
    作一些整理
    \begin{Equation}*
        \frac{\delt{E_\text{D}}}{\kB}\qty(\frac{1}{T})=\frac{3}{2}\ln T+\ln\frac{2(\mne\kB/2\pi)^{3/2}D_{-}}{g_\text{D}N_\text{D}\hbar^3}\qedhere
    \end{Equation}
\end{Proof}

\subsection{过渡区}
\begin{BoxFormula}[过渡区的电中性条件]
    过渡区的电中性条件为
    \begin{Equation}
        n_0=N_\text{D}+p_0
    \end{Equation}
\end{BoxFormula}

当温度进一步升高时,在杂质全部电离的基础上,本征激发产生的电子数量逐渐增多并需要被纳入考虑,不能忽略。这就是过渡区的含义,即杂质电离和本征激发共同作用的过渡阶段。

\begin{BoxFormula}[过渡区的费米能级]
    过渡区的费米能级满足
    \begin{Equation}
        E_\text{F}=E_\text{i}+\kB T\arsinh\qty(\frac{N_\text{D}}{2n_\text{i}})
    \end{Equation}
\end{BoxFormula}
\begin{Proof}
    过渡区费米能级的推导有一定技巧性,关键在于利用本征载流子浓度。

    根据\fancyref{fml:本征半导体的载流子浓度},我们固然可以将$n_\text{i}$表示为
    \begin{Equation}&[1]
        n_\text{i}=(N_\text{c}N_\text{v})^{1/2}\exp(-\frac{E_\text{g}}{2\kB T})
    \end{Equation}
    但是,这并不妨碍我们将$n_i$写作
    \begin{Equation}&[2]
        n_\text{i}=N_\text{c}\exp(\frac{E_\text{i}-E_\text{c}}{\kB T})=N_\text{v}\exp(\frac{E_\text{v}-E_\text{i}}{\kB T})
    \end{Equation}
    这样,$N_\text{c}$和$N_\text{v}$就可以分别被表示为
    \begin{Align}[8pt]
        N_\text{c}&=n_i\exp(\frac{E_\text{c}-E_\text{i}}{\kB T})\xlabelpeq{3}\\
        N_\text{v}&=n_i\exp(\frac{E_\text{i}-E_\text{v}}{\kB T})\xlabelpeq{4}
    \end{Align}
    由此,根据\fancyref{fml:导带电子浓度},并代入\xrefpeq{3}
    \begin{Equation}&[4]
        \qquad
        n_0=N_\text{c}\exp(\frac{E_\text{F}-E_\text{c}}{\kB T})=n_i\exp(\frac{E_\text{c}-E_\text{i}}{\kB T})\exp(\frac{E_\text{F}-E_\text{c}}{\kB T})=n_i\exp(\frac{E_\text{F}-E_\text{i}}{\kB T})
        \qquad
    \end{Equation}
    由此,根据\fancyref{fml:价带空穴浓度},并代入\xrefpeq{4}
    \begin{Equation}&[5]
        \qquad
        p_0=N_\text{v}\exp(\frac{E_\text{v}-E_\text{F}}{\kB T})=n_i\exp(\frac{E_\text{i}-E_\text{v}}{\kB T})\exp(\frac{E_\text{v}-E_\text{F}}{\kB T})=n_i\exp(\frac{E_\text{i}-E_\text{F}}{\kB T})
        \qquad
    \end{Equation}
    而\fancyref{fml:过渡区的电中性条件}指出
    \begin{Equation}&[6]
        N_\text{D}=n_0-p_0
    \end{Equation}
    代入\xrefpeq{4}和\xrefpeq{5}
    \begin{Equation}
        N_\text{D}=n_\text{i}\qty[\exp(\frac{E_\text{F}-E_\text{i}}{\kB T})-\exp\qty(\frac{E_\text{i}-E_\text{F}}{\kB T})]
    \end{Equation}
    引入双曲正弦
    \begin{Equation}
        N_\text{D}=2n_\text{i}\sinh\qty(\frac{E_\text{F}-E_\text{i}}{\kB T})
    \end{Equation}
    求解
    \begin{Equation}
        \frac{E_\text{F}-E_\text{i}}{\kB T}=\arsinh\frac{N_\text{D}}{2n_\text{i}}
    \end{Equation}
    即得
    \begin{Equation}*
        E_\text{F}=E_\text{i}+\kB T\arsinh\frac{N_\text{D}}{2n_\text{i}}\qedhere
    \end{Equation}
\end{Proof}

如\xref{fig:杂质半导体的费米能级}所示,过渡区的费米能级$E_\text{F}(T)$曲线,在温度较低时行为与强电离区一致,在温度较高时则与强电离区发生分歧,费米能级的减小趋缓,最终会趋于本征激发时的费米能级。

\begin{BoxFormula}[过渡区的载流子浓度]
    过渡区中,导带电子浓度满足
    \begin{Equation}
        n_0=\frac{N_\text{D}}{2}\qty[1+\qty(1+\frac{4n_\text{i}^2}{N_\text{D}^2})^{1/2}]
    \end{Equation}
    过渡区中,价带空穴浓度满足
    \begin{Equation}
        p_0=\frac{2n_\text{i}^2}{N_\text{D}}\qty[1+\qty(1+\frac{4n_\text{i}^2}{N_\text{D}^2})^{1/2}]^{-1}\hspace*{-0.75em}
    \end{Equation}
\end{BoxFormula}
\begin{Proof}
    根据\fancyref{fml:过渡区的电中性条件}
    \begin{Equation}&[1]
        p_0=n_0-N_\text{D}
    \end{Equation}
    根据\fancyref{fml:载流子的浓度乘积}
    \begin{Equation}&[2]
        p_0n_0=n_\text{i}^2
    \end{Equation}
    将\xrefpeq{1}代入\xrefpeq{2},消去$p_0$
    \begin{Equation}&[3]
        n_0(n_0-N_\text{D})=n_\text{i}^2
    \end{Equation}
    展开
    \begin{Equation}&[4]
        n_0^2-n_0N_\text{D}-n_\text{i}^2=0
    \end{Equation}
    运用二次方程的求根公式
    \begin{Equation}&[5]
        n_0=\frac{N_\text{D}\pm\sqrt{N_\text{D}^2+4n_\text{i}^2}}{2}
    \end{Equation}
    整理(负根无意义,舍去)
    \begin{Equation}&[6]
        n_0=\frac{N_\text{D}}{2}\qty[1+\sqrt{1+\frac{4n_\text{i}^2}{N_\text{D}^2}}]
    \end{Equation}
    这就求出了$n_0$,而$p_0$可以由\xrefpeq{2}解出
    \begin{Equation}&[7]
        p_0=\frac{n_\text{i}^2}{n_0}=\frac{2n_\text{i}^2}{N_\text{D}}\qty[1+\sqrt{1+\frac{4n_\text{i}^2}{N_\text{D}^2}}]^{-1}
    \end{Equation}
    这样$p_0$就也求出了。
\end{Proof}
\xref{fml:过渡区的载流子浓度}是正确的,但是为了简化,我们常常会关注它的两个极限情形
\begin{itemize}
    \item 当$N_\text{D}\gg n_\text{i}$时,即杂质电离占主导地位,是过渡区在强电离区一侧的近似。
    \item 当$N_\text{D}\ll n_\text{i}$时,即本征激发占主导地位,是过渡区在本征激发一侧的近似。
\end{itemize}\vspace{0.25cm}
\begin{BoxFormula}[过渡区的近强电离区近似]
    过渡区中,若$N_\text{D}\gg n_\text{i}$,则$n_0,p_0$分别满足
    \begin{Equation}
        n_0=N_\text{D}+\frac{n_\text{i}^2}{N_\text{D}}\qquad p_0=\frac{n_\text{i}^2}{N_\text{D}}
    \end{Equation}
\end{BoxFormula}
\begin{Proof}
    当$N_\text{D}\gg n_\text{i}$时,则$4n_\text{i}^2/N_\text{D}^2\ll 1$,故可运用泰勒展开
    \begin{Equation}&[1]
        \qty(1+\frac{4n_\text{i}^2}{N_\text{D}^2})=1+\frac{1}{2}\frac{4n_\text{i}^2}{N_\text{D}^2}=1+\frac{2n_\text{i}^2}{N_\text{D}^2}
    \end{Equation}
    将\xrefpeq{1}代入\fancyref{fml:过渡区的载流子浓度}中的$n_0$
    \begin{Equation}
        n_0=\frac{N_\text{D}}{2}\qty[1+\qty(1+\frac{4n_\text{i}^2}{N_\text{D}^2})^{1/2}]=\frac{N_\text{D}}{2}\qty[2+\frac{2n_\text{i}^2}{N_\text{D}^2}]=
        N_\text{D}+\frac{n_\text{i}^2}{N_\text{D}}
    \end{Equation}
    而$p_0$则依据\fancyref{fml:过渡区的电中性条件}给出
    \begin{Equation}
        p_0=n_0-N_\text{D}=\frac{n_\text{i}^2}{N_\text{D}}
    \end{Equation}
    由此就求出了$n_0$和$p_0$在$N_\text{D}\gg n_\text{i}$下的近似。
\end{Proof}

\xref{fml:过渡区的近强电离区近似}是很重要的,注意到由于$N_\text{D}\gg n_\text{i}$,有$n_0\gg p_0$,即电子浓度远大于空穴浓度,因此,电子称为\uwave{多数载流子}(Majority Carriers),空穴称为\uwave{少数载流子}(Minority Carrier)。这样我们就从理论上证明了杂质半导体中为何会有多子和少子。需要指出,尽管少数载流子的数量非常少,但是其在器件的工作中也起着极其重要的作用,许多器件就是少子导电的。

\begin{BoxFormula}[过渡区的近本征激发近似]
    过渡区中,若$N_\text{D}\ll n_\text{i}$,则$n_0,p_0$分别满足
    \begin{Equation}
        n_0=n_\text{i}+\frac{N_\text{D}}{2}\qquad
        p_0=n_\text{i}-\frac{N_\text{D}}{2}
    \end{Equation}
\end{BoxFormula}

\begin{Proof}
    根据\fancyref{fml:过渡区的载流子浓度},做一些变化
    \begin{Equation}&[1]
        n_0=\frac{N_\text{D}}{2}\qty[1+\qty(1+\frac{4n_\text{i}^2}{N_\text{D}^2})^{1/2}]=\frac{N_\text{D}}{2}+\frac{N_\text{D}}{2}\qty(1+\frac{4n_\text{i}^2}{N_\text{D}^2})^{1/2}
    \end{Equation}
    将\xrefpeq{1}的第二项中的$N_\text{D}$并入
    \begin{Equation}&[2]
        n_0=\frac{N_\text{D}}{2}+\frac{1}{2}\qty(N_\text{D}+4n_\text{i}^2)^{1/2}
    \end{Equation}
    转而将$4n_\text{i}^2$提出
    \begin{Equation}&[3]
        n_0=\frac{N_\text{D}}{2}+n_\text{i}\qty(1+\frac{N_\text{D}^2}{4n_\text{i}^2})^{1/2}
    \end{Equation}
    由于$N_\text{D}\ll n_\text{i}$,因此可以简化为\footnote{其实前续这几步倒腾的唯一目的就是,将$4n_\text{i}^2/N_\text{D}^2$翻转为$N_\text{D}/4n_\text{i}^2$,从而适用近似$N_\text{D}\ll n_\text{i}$。}
    \begin{Equation}
        n_0=n_\text{i}+\frac{N_\text{D}}{2}
    \end{Equation}
    而$p_0$仍然由\fancyref{fml:过渡区的电中性条件}
    \begin{Equation}
        p_0=n_0-N_\text{D}=n_\text{i}-\frac{N_\text{D}}{2}
    \end{Equation}
    由此就求出了$n_0$和$p_0$在$N_\text{D}\ll n_\text{i}$下的近似。
\end{Proof}

\xref{fml:过渡区的近本征激发近似}表明$n_0,p_0$数量相近,都趋于$n_\text{i}$,这是过渡区更接近本征激发一侧的情况。

\subsection{本征激发区}
\begin{BoxFormula}[本征激发区的电中性条件]
    本征激发区的电中性条件为
    \begin{Equation}
        n_0=p_0
    \end{Equation}
\end{BoxFormula}
当温度再升高时,此时本征激发产生的载流子已经远多于杂质电离产生的载流子,杂质反而可以忽略了,即有$n_0\gg N_\text{D}$和$p_0\gg N_\text{D}$成立。此时杂质半导体的特性就回到了本征半导体的特性,适用\fancyref{fml:本征半导体的费米能级}和\fancyref{fml:本征半导体的载流子浓度}。

\subsection{N型半导体和P型半导体的对比}
以上我们都是以N型半导体为例再讨论,而P型半导体的情况是完全类似的。\vspace{0.5ex}

\begin{TableLong}[N型半导体和P型半导体的对比]
<工作区&特性&N型半导体&P型半导体\\>
\mr{4}[-0.8ex]{基本性质}
&杂质类型&\xgp[4pt]{施主杂质}&\xgp[4pt]{受主杂质}\\*
&多子类型&\xgp[4pt]{电子}&\xgp[4pt]{空穴}\\*
&少子类型&\xgp[4pt]{空穴}&\xgp[4pt]{电子}\\*
&简并因子
&\xgp[4pt]{$g_\text{D}=2$}
&\xgp[4pt]{$g_\text{A}=4$}\\
\mr{2}[-2.5ex]{弱电离区}
&费米能级
&\xgp[6pt]{$\mathlarger{E_\text{F}=\frac{E_\text{c}+E_\text{D}}{2}+\frac{\kB T}{2}\ln(\frac{N_\text{D}}{g_\text{D}N_\text{C}})}$}
&\xgp[6pt]{$\mathlarger{E_\text{F}=\frac{E_\text{v}+E_\text{A}}{2}-\frac{\kB T}{2}\ln(\frac{N_\text{A}}{g_\text{A}N_\text{v}})}$}\\*
&多子浓度
&\xgp[6pt]{$\mathlarger{n_0=\qty(\frac{N_\text{D}N_\text{c}}{g_\text{D}})\exp(-\frac{\delt{E_\text{D}}}{2\kB T})}$}
&\xgp[6pt]{$\mathlarger{p_0=\qty(\frac{N_\text{A}N_\text{v}}{g_\text{A}})\exp(-\frac{\delt{E_\text{A}}}{2\kB T})}$}\\ \hlinelig
\mr{4}[-6.5ex]{强电离区}
&费米能级
&\xgp[6pt]{$\mathlarger{E_\text{F}=E_\text{c}+\kB T\ln(\frac{N_\text{D}}{N_\text{c}})}$}
&\xgp[6pt]{$\mathlarger{E_\text{F}=E_\text{v}-\kB T\ln(\frac{N_\text{A}}{N_\text{v}})}$}\\*
&多子浓度
&\xgp[6pt]{$\mathlarger{\vphantom{\frac{D}{D}}n_0=N_\text{D}}$}
&\xgp[6pt]{$\mathlarger{\vphantom{\frac{D}{D}}p_0=N_\text{A}}$}\\*
&未电离杂质
&\xgp[6pt]{$\mathlarger{\vphantom{\frac{D}{D}}n_\text{D}=D_{-}N_\text{D}}$}
&\xgp[6pt]{$\mathlarger{\vphantom{\frac{D}{D}}p_\text{A}=D_{+}N_\text{A}}$}\\*
&未电离比例
&\xgp[6pt]{$\mathlarger{D_{-}=\qty(\frac{4N_\text{D}}{N_\text{c}})\exp(\frac{\delt{E_\text{D}}}{\kB T})}$}
&\xgp[6pt]{$\mathlarger{D_{+}=\qty(\frac{4N_\text{A}}{N_\text{v}})\exp(\frac{\delt{E_\text{A}}}{\kB T})}$}\\ \hlinelig
\mr{4}[-6.5ex]{过渡区}
&费米能级
&\xgp[6pt]{$\mathlarger{E_\text{F}=E_\text{i}+\kB T\arsinh\qty(\frac{N_\text{D}}{2n_\text{i}})}$}
&\xgp[6pt]{$\mathlarger{E_\text{F}=E_\text{i}-\kB T\arsinh\qty(\frac{N_\text{A}}{2n_\text{i}})}$}\\*
&多子浓度
&\xgp[6pt]{$\mathlarger{n_0=\frac{N_\text{D}}{2}\qty[1+\qty(1+\frac{4n_\text{i}^2}{N_\text{D}^2})^{1/2}]}$}
&\xgp[6pt]{$\mathlarger{p_0=\frac{N_\text{A}}{2}\qty[1+\qty(1+\frac{4n_\text{i}^2}{N_\text{A}^2})^{1/2}]}$}\\
&少子浓度
&\xgp[6pt]{$\mathlarger{p_0=\frac{2n_\text{i}^2}{N_\text{D}}\qty[1+\qty(1+\frac{4n_\text{i}^2}{N_\text{D}^2})^{1/2}]^{-1}}$}
&\xgp[6pt]{$\mathlarger{n_0=\frac{2n_\text{i}^2}{N_\text{A}}\qty[1+\qty(1+\frac{4n_\text{i}^2}{N_\text{A}^2})^{1/2}]^{-1}}$}\\
\end{TableLong}

尽管前面讨论了这么多,但就实际应用而言,很多时候我们只要把握定性的结论就可以了。

就费米能级而言,\xref{fig:不同掺杂情况下的费米能级}很好的揭示了费米能级与掺杂的关系
\begin{itemize}
    \item N型掺杂的费米能级靠近导带底,掺杂越多,越靠近导带底。
    \item P型掺杂的费米能级靠近价带顶,掺杂越多,越靠近价带顶。
    \item 本征型的费米能级位于禁带中线。
\end{itemize}
\begin{Figure}[不同掺杂情况下的费米能级]
    \begin{FigureSub}[强P型]
        \includegraphics[width=2.7cm]{build/Chapter03E_01.fig.pdf}
    \end{FigureSub}
    \begin{FigureSub}[弱P型]
        \includegraphics[width=2.7cm]{build/Chapter03E_02.fig.pdf}
    \end{FigureSub}
    \begin{FigureSub}[本征型]
        \includegraphics[width=2.7cm]{build/Chapter03E_03.fig.pdf}
    \end{FigureSub}
    \begin{FigureSub}[弱N型]
        \includegraphics[width=2.7cm]{build/Chapter03E_04.fig.pdf}
    \end{FigureSub}
    \begin{FigureSub}[强N型]
        \includegraphics[width=2.7cm]{build/Chapter03E_05.fig.pdf}
    \end{FigureSub}
\end{Figure}
就载流子浓度而言,在一定温度下,如果掺杂浓度低于本征激发的载流子浓度,我们可以近似认为$n_0,p_0=n_i$,即材料是本征的。相反,如果掺杂浓度高于本征激发的载流子浓度,我们可以适用强电离区的观点,多子浓度就等于杂质浓度,多子随杂质浓度增加而增加,同时,很显然的是,少子浓度应当相应满足$n_0p_0=n_\text{i}^2$的关系\footnote{先前强电离区的分析中,我们认为少子浓度是零,这是讨论电中性条件时的近似,不应带到这里。},因而,少子随杂质浓度增加而减小。

对于N型半导体而言
\begin{Equation}
    n_0=N_\text{D}\qquad
    p_0=\frac{n_\text{i}^2}{N_\text{D}}
\end{Equation}
对于P型半导体而言
\begin{Equation}
    p_0=N_\text{A}\qquad
    n_0=\frac{n_\text{i}^2}{N_\text{A}}
\end{Equation}
这其实就是我们关于杂质半导体要把握的核心要点了。