\section{载流子的漂移运动和迁移率}

\subsection{欧姆定律}
在电磁场中,我们已经将过去宏观上$U=IR$的欧姆定律拓宽到微观了。
\begin{BoxLaw}[欧姆定律]
    欧姆定律的微观形式是
    \begin{Equation}
        J=\sigma\Emf
    \end{Equation}
    其中,$J,\Emf$分别代表电流体密度和电场强度,$\sigma$代表电导率。
\end{BoxLaw}
这里需要说明的是符号$\Emf$的意义。电磁场中倾向于用$\Emf$表示电动势的含义,而电场强度则常用$\vb*{E}$或$E$表示。但半导体物理中,由于$E$普遍被用于表示能量,惯用$\Emf$表示电场强度。

\subsection{漂移速度和迁移率}
在有外加电压时,导体内部的自由电子受到电场力的作用,沿着电场的反方向作定向运动,形成电流。电子在电场力的作用下的这种运动称为\uwave{漂移运动}(Drift),电子作定向运动的速度则称为\uwave{漂移速度}(Drift Velocity)。通常我们更多使用的是平均漂移速度,以符号$\bar{v_\text{d}}$表示。

根据电磁学的知识,电流密度的大小与电荷量和电荷的运动速度之间的关系是\setpeq{漂移速度和迁移率}
\begin{Equation}&[1]
    J=nq\bar{v_\text{d}}
\end{Equation}
这里$q$是电子所带的电荷量,而$n$是电子的数密度。

而\fancyref{law:欧姆定律}又告诉我们
\begin{Equation}&[2]
    J=\sigma\Emf
\end{Equation}
这样一来,联立\xrefpeq{1}和\xrefpeq{2},解出$\bar{v_\text{d}}$
\begin{Equation}&[3]
    \bar{v_\text{d}}=\frac{\sigma}{nq}\Emf
\end{Equation}
这里我们注意到两件事,首先\xrefpeq{3}说明,电子的平均漂移速度和电场强度成正比,我们可以将两者的比值定义成一个常数。其次,这个常数的表达式同时也由\xrefpeq{3}给出了。
\begin{BoxDefinition}[迁移率]
    \uwave{迁移率}定义为平均漂移速度和电场强度的比值
    \begin{Equation}
        \mu=\frac{\bar{v_\text{d}}}{\Emf}
    \end{Equation}
\end{BoxDefinition}

\begin{BoxFormula}[迁移率与电导率]
    迁移率$\mu$与电导率$\sigma$的关系是
    \begin{Equation}
        \mu=\frac{\sigma}{nq}
    \end{Equation}
    或者
    \begin{Equation}
        \sigma=nq\mu
    \end{Equation}
\end{BoxFormula}

引入迁移率后,电流密度就可以用迁移率表示了
\begin{Equation}
    J=nq\mu\Emf
\end{Equation}

注意下!上述这些内容都是对导体而言的!不过,实验发现,在电场强度不太大的情况下,半导体中的载流子在电场作用下也遵从欧姆定律。但是,半导体中存在两种载流子,即带正电的空穴和带负电的电子,而且两种载流子的浓度又随温度和掺杂的不同而不同,所以,它的导电机构要比导体复杂些,但其实只要将两者叠加起来就可以了。显然,电子和空穴是两种不同的载流子,因此,\empx{电子和空穴具有不同的迁移率},事实上,电子的迁移率比空穴要大一些。

如果以$\mu_\text{n},\mu_\text{p}$分别表示\uwave{电子迁移率}(Electron Mobility)和\uwave{空穴迁移率}(Hole Mobility)
\begin{Equation}
    J=J_\text{n}+J_\text{p}=(nq\mu_n+pq\mu_p)\Emf
\end{Equation}
这里$J_\text{n},J_\text{p}$分别代表电子和空穴的电流密度,而$n,p$分别代表电子和空穴浓度。

这样,我们就可以得到半导体中,电导率和迁移率的关系了
\begin{BoxFormula}[半导体的迁移率与电导率]*
    半导体中,电导率$\sigma$与迁移率$\mu_\text{n},\mu_\text{p}$的关系是
    \begin{Equation}
        \sigma=nq\mu_\text{n}+pq\mu_\text{p}
    \end{Equation}
    特别的,对于N型半导体,有$n\gg p$
    \begin{Equation}
        \sigma=nq\mu_\text{n}
    \end{Equation}
    特别的,对于P型半导体,有$n\ll p$
    \begin{Equation}
        \sigma=pq\mu_\text{n}
    \end{Equation}
    而对于本征半导体,由于$n_\text{i}=n=p$
    \begin{Equation}
        \sigma_\text{i}=n_\text{i}q(\mu_\text{p}+\mu_\text{n})
    \end{Equation}
\end{BoxFormula}