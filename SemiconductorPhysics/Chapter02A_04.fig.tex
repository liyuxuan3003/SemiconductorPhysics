\documentclass{xStandalone}

\begin{document}
\begin{tikzpicture}

\tikzset{dis/.style={thin,<->,gray}}

\path
    (0,+2) coordinate   (A1)
    ++(8,0) coordinate  (A2)
    (0,-2) coordinate   (B1)
    ++(8,0) coordinate  (B2);

\fill[red!10!white]  (A1) rectangle ($(A2)+(0,+1.5)$) coordinate (A2');
\fill[blue!10!white] (B1) rectangle ($(B2)+(0,-1.5)$) coordinate (B2');

\path ($(A1)!0.5!(A2')$) ++(0,+0.2) node {\textbf{导带}};
\path ($(B1)!0.5!(B2')$) ++(0,-0.2) node {\textbf{价带}};

\draw[thick] (A1)--(A2) node[right] {$E_\text{c}$};
\draw[thick] (B1)--(B2) node[right] {$E_\text{v}$};

\draw[dis] 
    ($(A1)!0.05!(A2)$) 
    -- node[left] {$E_\text{g}$} 
    ($(B1)!0.05!(B2)$);

\foreach \x in {1,2,...,7}
{
    \path (\fpeval{\x+0.75},-1) coordinate (P\x);
    \draw 
    ($(P\x)+(-0.25,0)$) -- 
    ($(P\x)+(+0.25,0)$);
}

\draw[dis] (P1) -- node[left] {$\delt{E_\text{A}}$} (P1|-B1);

\foreach \x in {2,3,4,5}
{
    \draw[latex-] (P\x) ++(0,-0.2) node[ocirc] {} ++(0,-0.2) -- ++(0,-1)coordinate (Q\x); 

    \path (Q\x) node[circ] {};
}

\foreach \x in {6,7}
{
    \draw (Q2-|P\x) node[ocirc] {};
    \draw (P\x) ++(0,+0.4) node[point] {$-$};
}

\path (P7) ++(0.25,0) node[right] {$E_\text{A}$};

\end{tikzpicture}
\end{document}