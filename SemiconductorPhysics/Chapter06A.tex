\section{PN结的形成}

首先的问题是,PN结是如何形成的?显然,将P型半导体和N型半导体拿些焊锡焊在一起并不会得到任何的PN结。我们需要使用一些工艺。工艺有很多种,但核心思路是一致的,即在N型半导体中通过一定方式将其局部转变为P型半导体,或反之。工艺主要有以下四种
\begin{enumerate}
    \item 合金法
    \item 扩散法
    \item 生长法
    \item 注入法
\end{enumerate}
这里我们重点介绍合金法和扩散法,如\xref{fig:PN结的两种工艺},它们分别代表了两类不同类型的PN结。
\begin{Figure}[PN结的两种工艺]
    \begin{FigureSub}[合金法]
        \includegraphics[scale=0.9]{build/Chapter06A_02.fig.pdf}
    \end{FigureSub}
    \hspace{2cm}
    \begin{FigureSub}[扩散法]
        \includegraphics[scale=0.9]{build/Chapter06A_03.fig.pdf}
    \end{FigureSub}
\end{Figure}

\subsection{合金法}
\xref{fig:合金法}表示用合金法制造PN结的过程,将一小粒铝放在一块N型单晶硅上,随后将铝粒加热到一定温度,铝粒和其接触的硅薄层形成硅铝熔融体,然后降低温度,硅铝熔融体开始凝固,这时N型硅片上就形成了一含有高浓度铝的P型硅片,构成PN结,称为铝硅合金结。

\uwave{合金结}(Alloy Junction)属于\uwave{突变结},其浓度分布如\xref{fig:突变结的掺杂浓度分布}
\begin{Figure}[突变结的掺杂浓度分布]
    \includegraphics[scale=1]{build/Chapter06A_04.fig.pdf}
\end{Figure}
设交界面为$x_j$,则有
\begin{Gather}
    N_\text{A}(x)=N_\text{A}\quad (x<x_\text{j})\qquad
    N_\text{D}(x)=N_\text{D}\quad (x>x_\text{j})
\end{Gather}
换言之,在突变结中,P型区仅有受主杂质$N_\text{A}$,N型区仅有施主杂质$N_\text{D}$,两者均是均匀分布的,交界面处,两者的浓度均会发生突变,$N_\text{A}(x)$由$N_\text{A}$变为$0$,$N_\text{D}(x)$由$0$变为$N_\text{D}$。

在实际的突变结中,两边的杂质浓度差异很大,例如这里在N型基底上通过铝形成了高浓度的P型区,则P型区的掺杂将远高于N型区,如$N_\text{A}=10^{16}\si{cm^{-3}}$,而$N_\text{D}=10^{19}\si{cm^{-3}}$。简而言之,\empx{通过合金法在局部形成的掺杂区比基底的掺杂浓度要高的多}。通常,这种结称为\uwave{单边突变结},这里P型区浓度较高,故称为P$^{+}$N结,反之若N型区浓度较高,则称为N$^{+}$P结。

\subsection{扩散法}
\xref{fig:扩散法}表示用扩散法制造PN结的过程,在N型单晶硅片上,第一步通过氧化作出一层均匀的\xce{SiO2}氧化物薄层,第二步通过光刻工艺将\xce{SiO2}刻蚀出特定的形状,使得待掺杂部分暴露出来,第三步通过扩散工艺掺入P型杂质,从而在(光刻暴露出的)特定位置形成P型区。

\uwave{扩散结}(Diffused Junction)属于\uwave{缓变结},其浓度分布如\xref{fig:扩散结的掺杂浓度分布}
\begin{Figure}[扩散结的掺杂浓度分布]
    \includegraphics[scale=1]{build/Chapter06A_05.fig.pdf}
\end{Figure}
扩散结中,基底的浓度可以视为不变的
\begin{Equation}
    N_\text{D}(x)=N_\text{D}
\end{Equation}
扩散结中,掺入的杂质(这里是受主杂质)的浓度则是随掺入深度的增加而逐渐减小的,并非是突变的,扩散是一个过程。通常,我们只关心PN结交界面$x=x_\text{j}$附近的性质,因此,无论$N_\text{A}(x)$实际的函数分布如何,我们总是可以用$x=x_\text{j}$附近的切线近似表示$N_\text{A}(x)$
\begin{Equation}
    N_\text{A}(x)=N_\text{D}+\alpha_\text{j}(x-x_\text{j})
\end{Equation}
这里$\alpha_\text{j}$是$x=x_\text{j}$处的切线斜率,称为\uwave{杂质浓度梯度}。若$\alpha_\text{j}$很大,亦可以近似为突变结。

总而言之,PN结主要包含两类情形,突变结和线性缓变结
\begin{enumerate}
    \item 合金结是突变结。
    \item 高表面浓度的浅扩散结(杂质浓度梯度大)是突变结。
    \item 低表面浓度的深扩散结(杂质浓度梯度小)是线性缓变结。
\end{enumerate}
这两种结在性质上有一定的差异,我们将会在本章稍后更为深入的探讨。\goodbreak