\section{非平衡载流子的产生与复合}

\setpeq{非平衡载流子的产生与复合}

半导体处于热平衡状态时,在一定温度下,载流子浓度是一定的。这种处于热平衡状态下的载流子浓度,称为\uwave{平衡载流子浓度}(Equilibrate Carrier Concentration)。前面各章我们讨论的都是平衡载流子,实际上$n_0$和$p_0$中的下标$0$就是为了强调这里描述的是平衡状态的电子浓度和空穴浓度。根据\fancyref{fml:载流子的浓度乘积},在非简并的情况下$n_0,p_0$的乘积满足
\begin{Equation}&[1]
    n_0p_0=N_\text{c}N_\text{v}\exp(-\frac{E_\text{g}}{\kB T})=n_\text{i}^2
\end{Equation}
换言之,在非简并时,无论半导体掺杂多少,只要半导体处于平衡状态上,载流子的浓度乘积就必然会满足\xrefpeq{1},掺杂,无非是使得$n_0$和$p_0$某个多些某个少些罢了,不改变乘积。

但是,有一些情况会打破这种平衡状态。试想,如果对半导体施加一些外界作用,如光照,而只要光子能量大于半导体的禁带宽度,那么,光子就可以将价带电子激发到导带上,从而形成额外的导带电子$\delt{n}$和价带空穴$\delt{p}$,这时,总的导带电子浓度和价带空穴浓度就分别是
\begin{Equation}
    n=n_0+\delt{n}\qquad p=p_0+\delt{p}
\end{Equation}
此时$np\neq n_\text{i}^2$,即热平衡状态被打破。

我们将$\delt{n},\delt{p}$称为\uwave{非平衡载流子}(Non-Equilibrium Carrier),取决于半导体的掺杂类型到底是N型还是P型,两者分别称为\uwave{非平衡多数载流子}和\uwave{非平衡少数载流子},具体而言
\begin{itemize}
    \item 对于N型半导体,$\delt{n}$是非平衡多数载流子,$\delt{p}$是非平衡少数载流子。
    \item 对于P型半导体,$\delt{n}$是非平衡少数载流子,$\delt{p}$是非平衡多数载流子。
\end{itemize}
我们将用光照产生非平衡载流子的方法,称为\uwave{光注入},其满足
\begin{Equation}
    \delt{n}=\delt{p}
\end{Equation}
而对于其他非平衡载流子的注入手段,这里$\delt{n},\delt{p}$就未必相等了。除了光注入,比较典型的还有电注入,实际上,我们后面将提到的PN结在其正向工作时,就是以电注入方式运作的。

通常,非平衡载流子浓度,远小于平衡多数载流子浓度,远大于平衡少数载流子浓度,满足该条件的注入称为\uwave{小注入},反之称为\uwave{大注入}。换言之,非平衡多数载流子通常可以忽略,但即便在小注入情况下,非平衡少数载流子的浓度仍然远大于平衡少数载流子,它的影响就十分显著了。因此,通常我们说非平衡载流子时指的都是非平衡少数载流子,介于后者的重要影响。

在\xref{fig:光照产生的非平衡载流子}中,我们描绘了光注入下,N型半导体中非平衡载流子和平衡载流子的浓度关系
\begin{Figure}[光照产生的非平衡载流子]
    \includegraphics{build/Chapter05A_01.fig.pdf}
\end{Figure}

光注入额外引入了载流子,因此势必会导致电导率增大,根据\xref{fml:半导体的迁移率与电导率}
\begin{Equation}
    \delt{\sigma}=
    \delt{n}q\mu_\text{n}+
    \delt{p}q\mu_\text{p}
\end{Equation}
而考虑到光注入时$\delt{n}=\delt{p}$,不妨统一记为$\delt{p}$
\begin{Equation}
    \delt{\sigma}=\delt{p} q(\mu_\text{n}+\mu_\text{p})
\end{Equation}
转而计算电阻率的变化,显然
\begin{Equation}
    \delt{\rho}=\frac{1}{\sigma}-\frac{1}{\sigma_0}
\end{Equation}
这里$\sigma_0$是半导体在平衡状态下的电导率,而$\sigma=\sigma_0+\delt{\sigma}$,通分得
\begin{Equation}
    \delt{\rho}=\frac{\sigma_0-\sigma}{\sigma\sigma_0}=-\frac{\delt{\sigma}}{\sigma\sigma_0}
\end{Equation}
这里我们可以作$\sigma\sigma_0=\sigma_0^2$的近似
\begin{Equation}
    \delt{\rho}=-\frac{\delt{\sigma}}{\sigma_0^2}
\end{Equation}
亦即
\begin{Equation}
    \delt{\rho}=
    -\frac{\delt{p}q(\mu_\text{n}+\mu_\text{p})}{\sigma_0^2}
\end{Equation}
进而考虑到电阻长度和截面积一定时,电阻正比于电阻率,根据欧姆定律
\begin{Equation}
    \delt{V}=I\delt{r}\propto\delt{\rho}\propto-\delt{p}
\end{Equation}\nopagebreak
这就表明,\empx{半导体上的电压降正比于非平衡少数载流子浓度的负值},这提供了一种测定非平衡少数载流子浓度的实验方法,通过电压降的变化直观反映非平衡少数载流子浓度的变化。\goodbreak

当产生非平衡载流子的外部作用撤出以后,半导体中将发生什么?仍然以光注入为例说明,我们知道,光照时$\delt{V}$的变化反映了$\delt{p}$的变化,而进一步的实验表明,光照停止后$\delt{V}$将很快的趋于零,大约只需要微秒到毫秒数量级的时间。这说明,注入的非平衡载流子并不能一直存在下去,原先激发到导带的电子重新回到价带,电子和而空穴又成对的消失,回到平衡态。

这里让我们来总结一下半导体在光注入中到底会发生什么
\begin{itemize}
    \item 光照施加时,电子由价带激发到导带,称为非平衡载流子的产生。
    \item 光照撤走后,电子由导带重新回到价带,称为非平衡载流子的复合。
\end{itemize}
这里,我们其实没有必要割裂非平衡载流子与平衡载流子,实际上,即便是平衡状态,也同时存在着载流子的产生与复合,但是两者速率相同,处于动态平衡中。光照时,在光子能量的作用下,产生速率显著提升,从而出现了非平衡载流子,光照撤走后,产生速率会立即恢复至平衡状态,但,复合速率却因为载流子浓度的增加而高于平衡状态,使非平衡载流子逐渐消失。
