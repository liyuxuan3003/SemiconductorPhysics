\documentclass{xStandalone}

\NewDocumentCommand{\drawIon}{O{gray}O{4}r()}
{
    \draw[fill=#1,thick] 
    (#3) circle (0.5)
    coordinate (TEMP)
    ($(TEMP)+(0.0,0)$) node[white]{\large $#2$};
}

\NewDocumentCommand{\drawElec}{r()}
{
    \draw[fill=black,very thin] (#1) circle(0.05);
}

\NewDocumentCommand{\drawHole}{r()}
{
    \draw[fill=white,very thin] (#1) circle(0.05)
}

\begin{document}
\begin{tikzpicture}

\tikzset{bound/.style={gray,very thin}}

\def\k{2.5}

\clip (-1.5*\k,-1.5*\k) rectangle (1.5*\k,1.5*\k);

\foreach \x in{-2,-1,...,2}
{
    \foreach \y in{-2,-1,...,2}
    {
        \path (\k*\x,\k*\y)
            coordinate (O)
            ++(-\k,0) coordinate (P);
        \drawElec($(P)!0.37!(O)$);
        \drawElec($(P)!0.63!(O)$);
        \draw[bound] (P) to[out=30,in=150] (O);
        \draw[bound] (P) to[out=-30,in=-150] (O);

        \path (\k*\x,\k*\y)
            coordinate (O)
            ++(0,-\k) coordinate (P);
        \drawElec($(P)!0.37!(O)$);
        \drawElec($(P)!0.63!(O)$);
        \draw[bound] (P) to[out=120,in=240] (O);
        \draw[bound] (P) to[out=60,in=-60] (O);
    }
}

\foreach \x in {-2,-1,...,2}
{
    \foreach \y in{-2,-1,...,2}
    {
        \drawIon[gray][\ce{Si}](\k*\x,\k*\y);
    }
}

\path ($(0,0)!0.37!(\k*1,0)$) coordinate(H);

\drawHole(H);

\draw[very thin,<-] ($(H)+(0,-0.13)$) -- ++(0,-1) node[below] {\footnotesize 空穴};

\drawIon[blue][$\hspace{0.1em}\ce{B-}$](0,0);

\draw[very thin,<-] (-0.5,0.5) -- ++(-0.7,0.7) node[above] {\footnotesize 受主杂质}  node[above=0.3cm] {\tiny (硼原子)};

\end{tikzpicture}
\end{document}