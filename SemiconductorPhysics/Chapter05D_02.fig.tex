\documentclass{xStandalone}

\begin{document}
\begin{tikzpicture}
    
\draw[thick] (0,0) coordinate (Ev) node[left] {$E_\text{v}$} -- ++(8,0) coordinate (Ev');

\draw[thick] ($(Ev)+(0,4)$) coordinate (Ec) node[left] {$E_\text{c}$} -- (Ec-|Ev')coordinate (Ec');

\draw[dashed] ($(Ev)+(0,2.25)$) coordinate (Et) node[left] {$E_\text{t}$} -- (Et-|Ev')coordinate (Et');

\path ($(Ec)!0.15!(Ec')$) node[circ] {} coordinate (P1);
\path ($(Et)!0.35!(Et')$) node[circ] {} coordinate (P2);
\path ($(Et)!0.65!(Et')$) node[circ] {} coordinate (P3);
\path ($(Ev)!0.85!(Ev')$) node[circ] {} coordinate (P4);


\draw[-latex] (P1) -- 
node[left,pos=0.25] {$1,R_\text{n}$} 
node[left,pos=0.75] {\footnotesize 俘获电子} 
(P1|-Et);

\draw[-latex] (P2) -- 
node[left,pos=0.75] {$2,G_\text{n}$} 
node[left,pos=0.25] {\footnotesize 发射电子} 
(P2|-Ec);

\draw[-latex] (P3) -- 
node[left,pos=0.25] {$3,R_\text{p}$} 
node[left,pos=0.75] {\footnotesize 俘获空穴} 
(P3|-Ev);

\draw[-latex] (P4) -- 
node[left,pos=0.75] {$4,G_\text{p}$} 
node[left,pos=0.25] {\footnotesize 发射空穴} 
(P4|-Et);

\path ($(Ec)!0.25!(Ec')$) node[above=0.1cm] {\footnotesize 导带电子浓度$n$};

\path ($(Et)!0.25!(Et')$) node[below=0.1cm] {\footnotesize 复合中心电子浓度$n_\text{t}$};

\path ($(Et)!0.75!(Et')$) node[above=0.1cm] {\footnotesize 复合中心空穴浓度$N_\text{t}-n_\text{t}$};

\path ($(Ev)!0.75!(Ev')$) node[below=0.1cm] {\footnotesize 价带空穴浓度$p$};

\end{tikzpicture}
\end{document}