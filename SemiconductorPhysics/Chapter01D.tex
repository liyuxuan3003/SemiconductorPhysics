\section{回旋共振}
在以上几节中,我们以一维情形为例,讨论了半导体能带结构的一些共同的基本特点。但是实际的三维半导体材料会更复杂一些,首先不同半导体材料的能带结构是不同的,其次还往往是各向异性的,即沿不同的波矢$\vb*{k}$方向,能量$E(\vb*{k})$与$\vb*{k}$的关系不同。本节,我们将首先了解三维情形下$E(\vb*{k})$与$\vb*{k}$的关系,随后,我们将了解如何通过回旋共振实验测定有效质量。

\subsection{椭球等能面}\setpeq{1}
在一维情形下,我们知道能带极值附近的$E(k)$与$k$的关系是
\begin{Equation}&[1]
    E(k)-E(0)=\frac{\hbar^2 k^2}{2\mne}
\end{Equation}
在三维情形下,由于$k=k_x^2+k_y^2+k_z^2$
\begin{Equation}&[2]
    E(\vb*{k})-E(\vb*{0})=\frac{\hbar^2}{2\mne}(k_x^2+k_y^2+k_z^2)
\end{Equation}
但是三维晶体往往是各向异性的,这反映在不同波矢$\vb*{k}$方向,能带极值处电子的有效质量将是不同的,除此之外,在比较一般的情况下,能带极值也不一定位于波矢$\vb*{k}=\vb*{0}$处,因此
\begin{Equation}&[3]
    E(\vb*{k})-E(\vb*{0})=\frac{\hbar^2}{2}\qty[\frac{(k_x-k_{0x})^2}{m_x^{*}}+\frac{(k_y-k_{0y})^2}{m_y^{*}}+\frac{(k_z-k_{0z})^2}{m_z^{*}}]
\end{Equation}
其中,各个方向的有效质量分别满足
\begin{Align}[8pt]
    \frac{1}{m_x^{*}}=\frac{1}{\hbar^2}\eval{\qty(\pdv[2]{E}{k_x})}_{k_{0x}}\\
    \frac{1}{m_y^{*}}=\frac{1}{\hbar^2}\eval{\qty(\pdv[2]{E}{k_y})}_{k_{0y}}\\
    \frac{1}{m_z^{*}}=\frac{1}{\hbar^2}\eval{\qty(\pdv[2]{E}{k_z})}_{k_{0z}}
\end{Align}
我们也可以将\xrefpeq{3}中的式子做一些改写
\begin{Equation}
    \frac{(k_x-k_{0x})^2}{2m_x^{*}(E-E_c)/\hbar^2}+
    \frac{(k_y-k_{0y})^2}{2m_y^{*}(E-E_c)/\hbar^2}+
    \frac{(k_z-k_{0z})^2}{2m_z^{*}(E-E_c)/\hbar^2}=1
\end{Equation}

其中$E, E_c$分别代表$E(\vb*{k})$和$E(\vb*{0})$,不难看出,这是一个$\vb*{k}$空间中的椭球等能面族。

\subsection{回旋共振}
根据大学物理的电磁学知识,我们知道,带电粒子在恒定均匀磁场中将作回旋运动,我们现在的想法,通过理论分析,求出半导体中的电子的回旋频率和其有效质量的关系,从而提供一种通过回旋共振实验(将半导体置于均匀恒定磁场,测定回旋频率)测定电子有效质量的途径。
\begin{BoxFormula}[回旋共振实验]
    电子的回旋频率与有效质量的关系是
    \begin{Equation}
        \omega_c=\frac{qB}{\mne}
    \end{Equation}
    其中,如果半导体材料是各向异性的,那么$\mne$实际代表
    \begin{Equation}
        \frac{1}{\mne}=\sqrt{\frac{m_x^{*}\alpha^2+m_y^{*}\beta^2+m_z^{*}\gamma^2}{m_x^{*}m_y^{*}m_z^{*}}}
    \end{Equation}
    其中$\alpha,\beta,\gamma$分别是磁场$\vb*{B}$沿$k_x,k_y,k_z$的方向余弦。\footnote[2]{注意,这里$\alpha,\beta,\gamma$本身就是方向余弦,而不是和相应坐标轴的夹角。}
\end{BoxFormula}

\begin{Proof}
    作为一些启发,我们先来考虑各向同性的情形,接下来再考虑各向异性的情形。

    若磁感应强度为$\vb*{B}$,而电子初速为$\vb*{v}$,则电子受到的磁场力$\vb*{f}$为
    \begin{Equation}&[1]
        \vb*{f}=-q\vb*{v}\times\vb*{B}
    \end{Equation}
    而磁场力的大小$f$为,其中$\sin\theta$是$B$和$v$之间的夹角
    \begin{Equation}&[2]
        f=qv\sin\theta B
    \end{Equation}
    根据大学物理的力学知识,向心加速度为
    \begin{Equation}&[3]
        a=\frac{(v\sin\theta)^2}{r}
    \end{Equation}
    根据大学物理的力学知识,角速度(角频率)为
    \begin{Equation}&[4]
        \omega_c=\frac{v\sin\theta}{r}
    \end{Equation}
    依据\xrefpeq{3},有$r=(v\sin\theta)^2/a$,将其代入\xrefpeq{4}
    \begin{Equation}&[5]
        \omega_c=\frac{v\sin\theta}{(v\sin\theta)^2/a}=\frac{a}{v\sin\theta}
    \end{Equation}
    而根据\fancyref{fml:半导体中电子的加速度}
    \begin{Equation}&[6]
        \omega_c=\frac{f}{m_n^{*} v\sin\theta}
    \end{Equation}
    将\xrefpeq{2}代入\xrefpeq{6}
    \begin{Equation}&[7]
        \omega_c=\frac{qv\sin\theta B}{m_n^{*}v\sin\theta}=\frac{qB}{m_n^{*}}
    \end{Equation}
    以上是各向同性的情况,而对于各向异性的情况,其力的三个分量分别为
    \begin{Split}&[8]
        f_x&=-qB(v_x\gamma-v_z\beta)\\[6pt]
        f_y&=-qB(v_z\alpha-v_x\gamma)\\[6pt]
        f_z&=-qB(v_x\beta-v_y\alpha)
    \end{Split}
    电子在三个方向的运动方程为(介于没有圆周运动的结论了,我们必须从运动方程出发)
    \begin{Split}&[9]
        m_x^{*}\dv{v_x}{t}&+qB(v_x\gamma-v_z\beta)=0\\[8pt]
        m_y^{*}\dv{v_y}{t}&+qB(v_z\alpha-v_x\gamma)=0\\[8pt]
        m_z^{*}\dv{v_z}{t}&+qB(v_x\beta-v_y\alpha)=0
    \end{Split}
    电子应当作周期运动
    \begin{Equation}&[10]
        v_x=v_{0x}\e^{\i\omega_c t}\qquad
        v_y=v_{0y}\e^{\i\omega_c t}\qquad
        v_z=v_{0z}\e^{\i\omega_c t}
    \end{Equation}
    注意到
    \begin{Equation}&[11]
        \dv{v_x}{t}=\i\omega_cv_x\qquad
        \dv{v_y}{t}=\i\omega_cv_y\qquad
        \dv{v_z}{t}=\i\omega_cv_z
    \end{Equation}
    将\xrefpeq{10}和\xrefpeq{11}代入\xrefpeq{9},并以矩阵形式表示
    \begin{Equation}&[12]
        \begin{pmatrix}
            \i\omega_c&\dfrac{qB}{m_x^{*}}\gamma&-\dfrac{qB}{m_x^{*}}\\[8pt]
            -\dfrac{qB}{m_y^{*}}\gamma&\i\omega_c&\dfrac{qB}{m_y^{*}}\alpha\\[8pt]
            \dfrac{qB}{m_z^{*}}\beta&-\dfrac{qB}{m_z^{*}}\alpha&\i\omega_c
        \end{pmatrix}
        \begin{pmatrix}
            v_x\vphantom{\dfrac{qB}{m_x^{*}}}\\[8pt]
            v_y\vphantom{\dfrac{qB}{m_y^{*}}}\\[8pt]
            v_z\vphantom{\dfrac{qB}{m_z^{*}}}
        \end{pmatrix}=0
    \end{Equation}
    如果\xrefpeq{12}有解,那\xrefpeq{12}的系数矩阵对应的行列式的值应为零,故
    \begin{Equation}&[13]
        (\i\omega_c)^3-\frac{qB}{m_x^{*}m_y^{*}m_z^{*}}\alpha\beta\gamma+\frac{qB}{m_x^{*}m_y^{*}m_z^{*}}\alpha\beta\gamma
        +\frac{(qB)^2}{m_z^{*}m_x^{*}}\beta^2(\i\omega C)
        +\frac{(qB)^2}{m_x^{*}m_y^{*}}\gamma^2(\i\omega C)
        +\frac{(qB)^2}{m_y^{*}m_z^{*}}\alpha^2(\i\omega C)=0
    \end{Equation}
    稍作整理
    \begin{Equation}&[14]
        (\i\omega_c)^3-(qB)^2\qty[\frac{\beta^2}{m_z^{*}m_x^{*}}+\frac{\gamma^2}{m_x^{*}m_y^{*}}+\frac{\alpha^2}{m_y^{*}m_z^{*}}](\i\omega_c)=0
    \end{Equation}
    将括号内通分,并两端约掉一个$(\i\omega_c)$
    \begin{Equation}&[15]
        (\i\omega_c)^2-(qB)^2\qty[\frac{\alpha^2m_x^{*}+\beta^2m_y^{*}+\gamma^2m_z^{*}}{m_x^{*}m_y^{*}m_z^{*}}]=0
    \end{Equation}
    即得
    \begin{Equation}&[16]
        \omega_c=qB\sqrt{\frac{\alpha^2m_x^{*}+\beta^2m_y^{*}+\gamma^2m_z^{*}}{m_x^{*}m_y^{*}m_z^{*}}}=\frac{qB}{\mne}
    \end{Equation}
    由此我们就完成了各向异性的情形。
\end{Proof}

注意,虽然各向同性和各向异性的$\omega_C$的数学形式是一致的,但是,各向同性时$m_n^{*}$即为有效质量,是常数,各向异性时$m_n^{*}$是各个方向的有效质量$m_x^{*},m_y^{*},m_z^{*}$的组合,但并非常数,这中间有磁场$\vb*{B}$沿各轴的方向余弦$\alpha,\beta,\gamma$介入,换言之,各向异性时$m_n^{*}$将随$\vb*{B}$的方向而变。