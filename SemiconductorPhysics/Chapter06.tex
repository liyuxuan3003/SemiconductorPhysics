\chapter{PN结}
在前几章中,我们分别研究了N型半导体和P型半导体中载流子的浓度和运动情况,认识了掺杂半导体在热平衡和非平衡状态下的一些物理性质。而如果将一块P型半导体和一块N型半导体结合在一起,在两者的交界面上,就形成了所谓的\uwave{PN结}(PN Junction),那么这种具有PN结的半导体将具有什么性质呢?这就是本章我们将要讨论的主要问题。由于PN结是很多半导体器件的基本结构,因此,了解和掌握PN结的性质就具有很重要的实际意义。

\begin{Figure}[PN结的示意图]
    \includegraphics{build/Chapter06A_01.fig.pdf}
\end{Figure}

在本章,主要将讨论PN结的几个重要特性,包含:电流电压特性、电容效应、击穿特性。

\section{PN结的形成}

首先的问题是,PN结是如何形成的?显然,将P型半导体和N型半导体拿些焊锡焊在一起并不会得到任何的PN结。我们需要使用一些工艺。工艺有很多种,但核心思路是一致的,即在N型半导体中通过一定方式将其局部转变为P型半导体,或反之。工艺主要有以下四种
\begin{enumerate}
    \item 合金法
    \item 扩散法
    \item 生长法
    \item 注入法
\end{enumerate}
这里我们重点介绍合金法和扩散法,如\xref{fig:PN结的两种工艺},它们分别代表了两类不同类型的PN结。
\begin{Figure}[PN结的两种工艺]
    \begin{FigureSub}[合金法]
        \includegraphics[scale=0.9]{build/Chapter06A_02.fig.pdf}
    \end{FigureSub}
    \hspace{2cm}
    \begin{FigureSub}[扩散法]
        \includegraphics[scale=0.9]{build/Chapter06A_03.fig.pdf}
    \end{FigureSub}
\end{Figure}

\subsection{合金法}
\xref{fig:合金法}表示用合金法制造PN结的过程,将一小粒铝放在一块N型单晶硅上,随后将铝粒加热到一定温度,铝粒和其接触的硅薄层形成硅铝熔融体,然后降低温度,硅铝熔融体开始凝固,这时N型硅片上就形成了一含有高浓度铝的P型硅片,构成PN结,称为铝硅合金结。

\uwave{合金结}(Alloy Junction)属于\uwave{突变结},其浓度分布如\xref{fig:突变结的掺杂浓度分布}
\begin{Figure}[突变结的掺杂浓度分布]
    \includegraphics[scale=1]{build/Chapter06A_04.fig.pdf}
\end{Figure}
设交界面为$x_j$,则有
\begin{Gather}
    N_\text{A}(x)=N_\text{A}\quad (x<x_\text{j})\qquad
    N_\text{D}(x)=N_\text{D}\quad (x>x_\text{j})
\end{Gather}
换言之,在突变结中,P型区仅有受主杂质$N_\text{A}$,N型区仅有施主杂质$N_\text{D}$,两者均是均匀分布的,交界面处,两者的浓度均会发生突变,$N_\text{A}(x)$由$N_\text{A}$变为$0$,$N_\text{D}(x)$由$0$变为$N_\text{D}$。

在实际的突变结中,两边的杂质浓度差异很大,例如这里在N型基底上通过铝形成了高浓度的P型区,则P型区的掺杂将远高于N型区,如$N_\text{A}=10^{16}\si{cm^{-3}}$,而$N_\text{D}=10^{19}\si{cm^{-3}}$。简而言之,\empx{通过合金法在局部形成的掺杂区比基底的掺杂浓度要高的多}。通常,这种结称为\uwave{单边突变结},这里P型区浓度较高,故称为P$^{+}$N结,反之若N型区浓度较高,则称为N$^{+}$P结。

\subsection{扩散法}
\xref{fig:扩散法}表示用扩散法制造PN结的过程,在N型单晶硅片上,第一步通过氧化作出一层均匀的\xce{SiO2}氧化物薄层,第二步通过光刻工艺将\xce{SiO2}刻蚀出特定的形状,使得待掺杂部分暴露出来,第三步通过扩散工艺掺入P型杂质,从而在(光刻暴露出的)特定位置形成P型区。

\uwave{扩散结}(Diffused Junction)属于\uwave{缓变结},其浓度分布如\xref{fig:扩散结的掺杂浓度分布}
\begin{Figure}[扩散结的掺杂浓度分布]
    \includegraphics[scale=1]{build/Chapter06A_05.fig.pdf}
\end{Figure}
扩散结中,基底的浓度可以视为不变的
\begin{Equation}
    N_\text{D}(x)=N_\text{D}
\end{Equation}
扩散结中,掺入的杂质(这里是受主杂质)的浓度则是随掺入深度的增加而逐渐减小的,并非是突变的,扩散是一个过程。通常,我们只关心PN结交界面$x=x_\text{j}$附近的性质,因此,无论$N_\text{A}(x)$实际的函数分布如何,我们总是可以用$x=x_\text{j}$附近的切线近似表示$N_\text{A}(x)$
\begin{Equation}
    N_\text{A}(x)=N_\text{D}+\alpha_\text{j}(x-x_\text{j})
\end{Equation}
这里$\alpha_\text{j}$是$x=x_\text{j}$处的切线斜率,称为\uwave{杂质浓度梯度}。若$\alpha_\text{j}$很大,亦可以近似为突变结。

总而言之,PN结主要包含两类情形,突变结和线性缓变结
\begin{enumerate}
    \item 合金结是突变结。
    \item 高表面浓度的浅扩散结(杂质浓度梯度大)是突变结。
    \item 低表面浓度的深扩散结(杂质浓度梯度小)是线性缓变结。
\end{enumerate}
这两种结在性质上有一定的差异,我们将会在本章稍后更为深入的探讨。\goodbreak
\section{PN结的结构与能带图}\nopagebreak

\subsection{PN结的结构}
试想,在PN结中,N型部分电子很多而空穴很少,P型部分空穴很多而电子很少,这就是说,载流子在PN结中存在浓度梯度,而\xref{sec:载流子的扩散运动}指出,载流子会在浓度梯度下作扩散运动
\begin{itemize}
    \item P型区的多子是空穴,空穴将由P区向N区扩散。
    \item N型区的多子是电子,电子将由N区向P区扩散。
\end{itemize}
我们知道,当P型半导体和N型半导体各自独立存在时,两者都是电中性的,因为杂质电离产生了多少空穴和电子,就相应的有多少电荷量相反的电离受主杂质和电离施主杂质。然而在PN结中,空穴扩散流和电子扩散流从P区和N区带走了相当数量的空穴和电子,电离杂质却无法跟随其运动,将会留在原位,这就使得在PN结交界面附近出现了一些带电区
\begin{itemize}
    \item P型区一侧出现了由电离受主构成的\uwave{负电荷区}(与P区多子空穴带正电相反)。
    \item N型区一侧出现了由电离施主构成的\uwave{正电荷区}(与N区多子电子带负电相反)。
\end{itemize}
我们通常将PN结附近的电离施主和电离受主所带的电荷称为\uwave{空间电荷}(Space Charge),而它们所存在的区域则称为\uwave{空间电荷区}(Space Charge Region)。正如我们在\xref{sec:爱因斯坦关系式}中熟悉的那样,这在PN结中,形成了一个由N区指向P区的内建电场,而载流子在内建电场的作用下将作漂移运动,显然,电子和空穴的漂移运动方向与它们的扩散运动是相反的,换言之,空间电荷区形成的内建电场将会反过来限制其自身进一步扩大。在一段时间后,电子和空穴的扩散电流和漂移电流会达成平衡,大小相等,方向相反,相互抵消。此时,流过PN结的净电流为零,空间电荷区保持一定宽度不再继续扩展,空间电荷区中存在一定的内建电场。

\begin{Figure}[PN结的空间电荷区]
    \includegraphics[scale=0.9]{build/Chapter06A_06.fig.pdf}
\end{Figure}

我们将这种热平衡状态下的PN结,称为\uwave{平衡PN结}(PN Junction in Equilibrium)。


\subsection{PN结的能带图}
在理解PN结的能带图前,我们不妨先来回顾一些,对于两块独立的P型半导体和N型半导体,它们的能带图是什么样的?如\xref{fig:P型和N型半导体的能带图},两者应具有相同的价带底$E_\text{v}$和导带底$E_\text{c}$以及相同的禁带中线$E_\text{i}$\ \footnote{确切的说$E_\text{i}$是本征状态下的费米能级,根据\xref{chap:半导体中载流子的统计分布},尽管$E_\text{i}$随着温度升高会偏离中线,但偏离的不多,可以近似相等。}。然而,两者的费米能级$E_\text{Fp},E_\text{Fn}$是不同的,我们知道,费米能级与载流子的分布是密切相关的,P区空穴较多故$E_\text{Fp}$靠近价带,N区电子较多故$E_\text{Fn}$靠近导带。

\vspace{0.25cm}
\begin{Figure}[平衡PN结的能带图]
    \begin{FigureSub}[P型和N型半导体的能带图]
        \includegraphics[scale=0.75]{build/Chapter06A_07.fig.pdf}
    \end{FigureSub}
    \\ \vspace{1cm}
    \begin{FigureSub}[PN结的能带图]
        \includegraphics[scale=0.75]{build/Chapter06A_08.fig.pdf}
    \end{FigureSub}
\end{Figure}
现在的问题是,当P型半导体和N型半导体形成PN结后,在交界区域,这些能量将会以何种方式变化?很自然的一种想法是,原先相同的$E_\text{c},E_\text{v},E_\text{i}$仍然相同,以直线相连,原先不同的$E_\text{F}$在空间电荷区以某种平滑的方式由$E_\text{Fp}$抬升至$E_\text{Fc}$,但很遗憾的是,这并不正确。正如\xref{fig:PN结的能带图}所示的那样,恰相反!费米能级反而在PN结中变得相等!费米能级$E_\text{Fp}$和$E_\text{Fn}$在各自区域中相对$E_\text{cp},E_\text{vp}$和$E_\text{cn},E_\text{vp}$的位置亦没有变化,这就是说,能带发生了相对移动!\goodbreak

我们可能会对这个结果感到不解,毕竟,为何要假设能带相对移动呢?但其实细想一下,能带移动才是更为合理的假设。因为PN结中存在由N区指向P区的内建电场,换言之,PN结中的电势不是均等的,P区电势较低,N区电势较高。尽管此刻我们尚不知道电势$V(X)$具体的数学形式\footnote{稍后有一张PN结电势$V(x)$的图像(其实就是把$E_\text{c},E_\text{v}$的趋势倒过来),再晚些我们会推导$V(X)$的表达式。},但我们可以预见的是,若以P区电势作为电势$V(X)$的零参考点
\begin{itemize}
    \item 电势$V(x)$在负端,即远离空间电荷区的P区为$0$。
    \item 电势$V(x)$在正端,即远离空间电荷区的N区为某一定值,记作$V_\text{D}$。
    \item 电势$V(X)$在空间电荷区中,逐渐由$0$增长至$V_\text{D}$(姑且不关心这个函数的具体形式)。
\end{itemize}

而能带反映的是电子的能量,这当然要包含电子的电势能$-qV(X)$,而$-qV(X)$从负到正是逐渐减小的,这也就是为何在PN结中,N型区能带相对P型区能带会向下偏移。所以说,能带移动是合理的。但这仍然没有解释,能带移动后为何恰好对齐至费米能级一致$E_\text{Fp}=E_\text{Fn}$的位置了呢?实际上,这是平衡PN结中没有净电流的直接结果,下面我们来证明这一点。\setpeq{电流密度与费米能级的关系}

我们先来计算电子电流,根据\fancyref{fml:载流子的漂移扩散}
\begin{Equation}&[1]
    J_\text{n}=qn\mu_\text{n}\Emf+qD_\text{n}\dv{n}{x}
\end{Equation}
根据\fancyref{law:爱因斯坦关系式},将$D_\text{n}$代换
\begin{Equation}&[2]
    J_\text{n}=nq\mu_\text{n}\Emf+\kB T\mu_\text{n}\dv{n}{x}
\end{Equation}
考虑到
\begin{Equation}&[3]
    \dv{\ln n}{x}=\frac{1}{n}\dv{n}{x}
\end{Equation}
将\xrefpeq{3}代入\xrefpeq{2}
\begin{Equation}&[4]
    J_\text{n}=nq\mu_\text{n}\Emf+n\kB T\mu_\text{n}\dv{\ln n}{x}
\end{Equation}
试着提取一些公共项
\begin{Equation}&[5]
    J_\text{n}=nq\mu_\text{n}\qty[\Emf+\frac{\kB T}{q}\dv{\ln n}{x}]
\end{Equation}
根据\fancyref{fml:非平衡态下的电子浓度}
\begin{Equation}&[6]
    n=n_\text{i}\exp(\frac{E_\text{F}-E_\text{i}}{\kB T})
\end{Equation}
在\xrefpeq{6}两端取对数
\begin{Equation}&[7]
    \ln n=\ln n_i+\frac{E_\text{F}-E_\text{i}}{\kB T}
\end{Equation}
进而是
\begin{Equation}&[8]
    \dv{\ln n}{x}=\frac{1}{\kB T}\qty(\dv{E_\text{F}}{x}-\dv{E_\text{i}}{x})
\end{Equation}
将\xrefpeq{8}代入\xrefpeq{5},得到
\begin{Equation}&[9]
    J_\text{n}=nq\mu_\text{n}\qty[\Emf+\frac{1}{q}\qty(\dv{E_\text{F}}{x}-\dv{E_\text{i}}{x})]
\end{Equation}
而根据前面的讨论,$E_\text{i}$的变化与$-qV(x)$是一致的,故
\begin{Equation}&[10]
    \dv{E_\text{i}}{x}=-q\dv{V(x)}{x}=-q\Emf
\end{Equation}
将\xrefpeq{10}代入\xrefpeq{9}中
\begin{Equation}&[11]
    J_\text{n}=nq\mu_\text{n}\dv{E_\text{F}}{x}
\end{Equation}
类似的,我们亦可以证明
\begin{Equation}&[12]
    J_\text{p}=p\mu_\text{p}\dv{E_\text{F}}{x}
\end{Equation}
而在平衡PN结中,电子和空穴的净电流$J_\text{n},J_\text{p}$均为零
\begin{Equation}
    J_\text{n}=J_\text{p}=0
\end{Equation}
因而
\begin{Equation}
    \dv{E_\text{F}}{x}=0\qquad E_\text{F}=C
\end{Equation}
这就论证了在平衡PN结中,费米能级$E_\text{F}$必须是一个常数。

这里\xrefpeq{11}和\xrefpeq{12}亦有理论价值,其将电流密度与费米能级联系在了一起。
\begin{BoxFormula}[电流密度与费米能级]
    电子的电流密度,正比于电子浓度和费米能级的变化率的积
    \begin{Equation}
        J_\text{n}=nq\mu_\text{n}\dv{E_\text{F}}{x}
    \end{Equation}
    空穴的电流密度,正比于空穴浓度和费米能级的变化率的积
    \begin{Equation}
        J_\text{p}=pq\mu_\text{P}\dv{E_\text{F}}{x}
    \end{Equation}
\end{BoxFormula}

\subsection{PN结的接触电势差}
在本小节,我们解决一个问题,即先前的$V_D$到底是多少?我们将平衡PN结空间电荷区两端间的电势差$V_\text{D}$称为PN结的\uwave{内建电势差}或\uwave{接触电势差}(Contact Potential Difference)。

在\xref{fig:PN结的能带图}中,我们看到,PN结的接触电势差$V_\text{D}$使PN结在空间电荷区中的能带发生弯曲,而因能带弯曲,电子从势能低的N区向势能高的P区运动时,必须克服这一势能“壁垒”,故空间电荷区亦被形象的称为\uwave{势垒区}(Barrier Region),势垒的高$qV_D$则称为\uwave{势垒高度}(Barrier Height)。很明显,势垒高度$qV_\text{D}$正好补偿了N区和P区原有的费米能级之差,即有
\begin{BoxFormula}[势垒高度]
    势垒高度补偿了N区和P区的费米能级之差,即
    \begin{Equation}
        qV_\text{D}=E_\text{Fn}-E_\text{Fp}
    \end{Equation}
\end{BoxFormula}
那么,接触电势差$V_\text{D}$到底如何求呢?关键在于利用载流子浓度。\setpeq{接触电势差}

在PN结中,载流子浓度有四个,以下分别表示:N区空穴、N区电子、P区空穴、P区电子
\begin{Equation}&[1]
    p_\text{N0}\qquad n_\text{N0}\qquad
    p_\text{P0}\qquad n_\text{P0}
\end{Equation}
在这里我们要用的是N区和P区中的电子浓度$n_\text{N0},n_\text{P0}$,根据\fancyref{fml:导带电子浓度}
\begin{Equation}&[2]
    n_\text{N0}=n_\text{i}\exp(\frac{E_\text{Fn}-E_\text{i}}{\kB T})\qquad
    n_\text{P0}=n_\text{i}\exp(\frac{E_\text{Fp}-E_\text{i}}{\kB T})
\end{Equation}
两式相除,取对数
\begin{Equation}&[3]
    \ln\frac{n_\text{N0}}{n_\text{P0}}=\frac{1}{\kB T}\qty(E_\text{Fn}-E_\text{Fp})=\frac{qV_\text{D}}{\kB T}
\end{Equation}
关注到$n_\text{N0}$为N型区多子浓度,而$n_\text{P0}$为P型区少子浓度
\begin{Equation}&[4]
    n_\text{N0}=N_\text{D}\qquad
    n_\text{P0}=\frac{n_\text{i}^2}{N_\text{A}}
\end{Equation}
将\xrefpeq{4}代入\xrefpeq{3}左端
\begin{Equation}&[5]
    \ln\frac{n_\text{N0}}{n_\text{P0}}=\ln\frac{N_\text{D}N_\text{A}}{n_\text{i}^2}
\end{Equation}
联立\xrefpeq{3}和\xrefpeq{5}
\begin{BoxFormula}[接触电势差]
    PN结的接触电势差$V_\text{D}$满足
    \begin{Equation}
        V_\text{D}=\frac{\kB T}{q}\ln\frac{N_\text{D}N_\text{A}}{n_\text{i}^2}
    \end{Equation}
\end{BoxFormula}

\xref{fml:接触电势差}表明,PN结的接触电势差由PN结自身特性确定,掺杂越多,温度越高,禁带宽度越大($n_\text{i}$越大),接触电势差就越大。硅的禁带比锗宽,因此,硅PN结的$V_\text{D}$也比锗PN结要更大些,若$N_\text{A}=10^{17}\si{cm}^{-3},N_\text{D}=10^{15}\si{cm^{-3}}$,室温下,硅$V_\text{D}=0.70\si{V}$,锗$V_\text{D}=0.32\si{V}$。

\subsection{PN结的载流子分布}
在本小节,我们要研究在PN结中,形成空间电荷区后,载流子的浓度将会如何分布?
\begin{BoxFormula}[PN结的平衡载流子浓度]
    PN结在平衡状态下,电子的平衡载流子浓度为
    \begin{Equation}&[A]
        n(x)=n_\text{N0}\exp[\frac{qV(x)-qV_\text{D}}{\kB T}]
    \end{Equation}
    PN结在平衡状态下,空穴的平衡载流子浓度为
    \begin{Equation}&[B]
        p(x)=p_\text{P0}\exp[\frac{-qV(x)}{\kB T}]
    \end{Equation}
    这里$n_\text{N0}$和$p_\text{P0}$分别是P区和N区的多子浓度。
\end{BoxFormula}
\begin{Proof}
    根据\fancyref{fml:导带电子浓度},这里$E_\text{cn}$代表N区的导带底
    \begin{Equation}
        \qquad\qquad
        n(x)=N_\text{c}\exp[\frac{E_\text{F}-E_\text{c}(x)}{\kB T}]=N_\text{c}\exp[\frac{E_\text{F}-E_\text{cn}}{\kB T}]\exp[\frac{E_\text{cn}-E_\text{c}(x)}{\kB T}]
        \qquad\qquad
    \end{Equation}
    引入$n_\text{N0}$,如\xref{fig:PN结的能带图}所示,取$E_\text{cp}$为零参考点,则$E_\text{c}(x)=-qV(x)$,而$E_\text{cn}=-qV_\text{D}$
    \begin{Equation}
        n(x)=n_\text{N0}\exp[\frac{E_\text{cn}-E_\text{c}(x)}{\kB T}]=n_\text{N0}\exp[\frac{qV(x)-qV_\text{D}}{\kB T}]
    \end{Equation}
    根据\fancyref{fml:导带电子浓度},这里$E_\text{vp}$代表P区价带底
    \begin{Equation}
        \qquad\qquad
        p(x)=N_\text{v}\exp[\frac{E_\text{v}(x)-E_\text{F}}{\kB T}]=N_\text{v}\exp[\frac{E_\text{vp}-E_\text{F}}{\kB T}]\exp[\frac{E_\text{v}(x)-E_\text{vp}(x)}{\kB T}]
        \qquad\qquad
    \end{Equation}
    引入$p_\text{P0}$,如\xref{fig:PN结的能带图}所示,取$E_\text{vp}$为零参考点,则$E_\text{v}(x)=-qV(x)$,而$E_\text{vp}=0$
    \begin{Equation}
        p(x)=p_\text{P0}\exp[\frac{E_\text{v}(x)-E_\text{vp}}{\kB T}]=p_\text{P0}\exp[\frac{-qV(x)}{\kB T}]
    \end{Equation}
    由此,我们就得到了\xrefpeq{A}和\xrefpeq{B}。
\end{Proof}

若以$x_\text{p}$和$x_\text{n}$表示空间电荷区在P区和N区的边界,则
\begin{Equation}
    V(x_\text{p})=0\qquad V(x_\text{n})=V_\text{D}
\end{Equation}
而显然少子浓度分别满足$n_\text{P0}=n(x_\text{p})$和$p_\text{N0}=p(x_\text{n})$,将上式代入,即得以下结论。
\begin{BoxFormula}[PN结多子浓度与少子浓度的关系]*
    PN结中,电子在P区的少子浓度$n_\text{P0}$和其在N区的多子浓度$n_\text{N0}$的关系是
    \begin{Equation}
        n_\text{P0}=n(x_\text{p})=n_\text{N0}\exp[\frac{-qV(x)}{\kB T}]
    \end{Equation}
    PN结中,空穴在N区的少子浓度$p_\text{N0}$和其在P区的多子浓度$p_\text{P0}$的关系是
    \begin{Equation}
        p_\text{N0}=p(x_\text{n})=p_\text{P0}\exp[\frac{-qV(x)}{\kB T}]
    \end{Equation}
\end{BoxFormula}


\section{PN结的电流电压特性}
在平衡PN结中,存在着具有一定宽度(空间上)和高度(能量上)的势垒,在势垒中出现了内建电场,载流子的扩散电流和漂移电流相互抵消,没有净电流通过PN结。本节讨论PN结的电流电压特性,就相当于是要讨论PN结在外加电压的非平衡状态下,将会如何工作?

简而言之,本节的目标是:\empx{研究外加电场作用下非平衡PN结的性质}。

\subsection{正向偏压下PN结的定性分析}
当对PN结外加正向偏压$V$时(即,P区接正极,N区接负极),能带变化如\xref{fig:正向偏压下的能带结构}所示。
\begin{Figure}[正向偏压下的能带结构]
    \includegraphics[scale=0.75]{build/Chapter06A_09.fig.pdf}
\end{Figure}
\begin{itemize}
    \item 势垒区内载流子浓度很小,电阻很大。
    \item 势垒区外载流子浓度很大,电阻很小。
\end{itemize}
因此,正向偏压$V$基本降落在势垒区,正向偏压在势垒区中产生了与内建电场方向相反的电场,从而削弱了内建电场,因此势垒区宽度减小,并且势垒高度由$qV_\text{D}$减小至$q(V_\text{D}-V)$。

势垒区的电场减弱,破坏了载流子的扩散运动和漂移运动原有的平衡,电场导致漂移,电场减弱,势垒区的扩散流就将大于漂移流。这样一来,当少子扩散通过势垒区后在势垒区边界出的浓度$n(x_\text{p})$和$p(x_\text{n})$,比相应的平衡少子浓度$n_\text{P0}, p_\text{P0}$要高。而这一部分多出来的少子就分别成为了P区和N区的非平衡载流子,它们将分别形成由势垒区边界$x_\text{p}$和$x_\text{n}$向P区和N区内部的电子扩散流和空穴扩散流。在扩散过程中,非平衡少子将逐渐与多子复合,最终将减小至$n_\text{P0}$和$p_\text{N0}$,理论上来说,扩散过程需要蔓延无限长的距离,但通常可以认为经过数倍扩散长度$L_\text{n}$和$L_\text{p}$的距离后,扩散流中的少子就被基本复合殆尽了,这一段区域即称为\uwave{扩散区}。

在\xref{chap:非平衡载流子}中,我们曾讨论过光注入的非平衡载流子,而在这里,这种通过对PN结外加正向偏压,使得PN结的势垒区边界出现的非平衡载流子,相应的就称为电注入的非平衡载流子。

\subsection{反向偏压下PN结的定性分析}
当对PN结外加反向偏压$V$时(即,P区接负极,N区接正极),能带变化如\xref{fig:反向偏压下的能带结构}所示。

\begin{Figure}[反向偏压下的能带结构]
    \includegraphics[scale=0.75]{build/Chapter06A_10.fig.pdf}
\end{Figure}

正向偏压和反相偏压下,PN结的特性有相似之处,我们下面对比着来看。

关于PN结势垒的变化,如\xref{fig:PN结的电势分布}所示
\begin{itemize}
    \item 正向偏压下,电场方向与内建电场相反,势垒宽度减小,势垒高度减小至$q(V_\text{D}-V)$。
    \item 反相偏压下,电场方向与内建电场相同,势垒宽度增大,势垒高度增大至$q(V_\text{D}+V)$。
\end{itemize}\goodbreak

关于PN结载流子浓度的变化,如\xref{fig:PN结的载流子浓度}所示
\begin{itemize}
    \item 正向偏压下,势垒区的扩散流强于漂移流,势垒边缘的少数载流子浓度$n(x_\text{p})$和$p(x_\text{n})$高于两端的少子浓度$n_\text{P0}$和$p_\text{N0}$,非平衡载流子将由PN结的势垒边缘向两端扩散。
    \item 反向偏压下,势垒区的扩散流弱于漂移流,势垒边缘的少数载流子浓度$n(x_\text{p})$和$p(x_\text{n})$低于两端的少子浓度$n_\text{P0}$和$p_\text{N0}$,非平衡载流子将由PN结的两端向势垒边缘扩散。
\end{itemize}
其实关于反向偏压下的载流子浓度,我们也可以这么理解,势垒区的扩散流弱于漂移流,漂移驱动的是少子,由于扩散流无法提供足够的少子,漂移将从PN结两端(即P区和N区的内部)进一步获得少子。这被形象的称为少数载流子的抽取或吸出。而在极端情况下,即假若反向电压很大,此时少数载流子几乎被抽取殆尽,在势垒区边界附近的少子浓度可以认为是零。

\subsection{非平衡PN结的能带图}
\xref{fig:正向偏压下的能带结构}和\xref{fig:反向偏压下的能带结构}中,注意到PN结在外加电压时,费米能级$E_\text{F}$分裂为了$E_\text{Fp},E_\text{Fn}$准费米能级\footnote{这里有必要澄清一下$E_\text{Fp},E_\text{Fn}$,在本章伊始,由\xref{fig:P型和N型半导体的能带图}引出\xref{fig:PN结的能带图}的过程中,记号$E_\text{Fn},E_\text{Fp}$是代表P区和N区的费米能级,我们看到,两块分离的P型和N型半导体$E_\text{Fn}\neq E_\text{Fp}$,两者形成PN结后则有$E_\text{Fn}=E_\text{Fp}$,即能带偏移,使P区和N区具有一致的费米能级。而当\xref{fig:PN结的能带图}作为平衡PN结引出正偏\xref{fig:正向偏压下的能带结构}和反偏\xref{fig:反向偏压下的能带结构}时,此时,记号$E_\text{Fn},E_\text{Fp}$的意义则分别转向表示电子和空穴的准费米能级,在平衡PN结中两者相等(平衡态的标志即统一的费米能级),在非平衡PN结中两者则不总是相等。}
\begin{itemize}
    \item 平衡区内有$E_\text{Fp}=E_\text{Fn}$,这很合理,平衡区和扩散区的分界即扩散的非平衡载流子的浓度可以忽略不计处,平衡区不存在非平衡载流子,而平衡态的标志即统一的费米能级。
    \item 平衡区中不妨记$E_\text{F}=E_\text{Fp}=E_\text{Fn}$,注意到P区和N区的费米能级$E_\text{F}$是不同的
    \begin{itemize}
        \item 正向偏压下,势垒减小为$V_\text{D}-V$,N区能带随之上移,N区$E_\text{F}$将大于P区。
        \item 反相偏压下,势垒增大为$V_\text{D}+V$,N区能带随之下移,N区$E_\text{F}$将小于P区。
    \end{itemize}
    \item 扩散区,由于有尚未复合完全的非平衡少数载流子的存在,费米能级发生分裂
    \begin{itemize}
        \item 正向偏压下,在P区,少子电子浓度因扩散增加,故$E_\text{Fn}$上移接近导带。
        \item 反向偏压下,在P区,少子电子浓度因抽取减小,故$E_\text{Fn}$下移远离导带。
        \item 正向偏压下,在N区,少子空穴浓度因扩散增加,故$E_\text{Fp}$下移接近价带。
        \item 反向偏压下,在N区,少子空穴浓度因抽取减小,故$E_\text{Fp}$上移远离价带。
    \end{itemize}
    \item 扩散区中准费米能级以线性方式变化,尚不清楚这是实际情况,还只是作图上的简化。
    \item 势垒区远小于扩散区\footnote{实际如此,\xref{fig:正向偏压下的能带结构}和\xref{fig:反向偏压下的能带结构}中势垒区和扩散区的长度未按此要求绘制。},故可以近似认为费米能级在势垒区不变。
\end{itemize}
这里我们可能会有疑问,注意到在反向偏压时,如\xref{fig:反向偏压下的能带结构}所示,准费米能级已经跑到价带或导带中了,此时难道不该适用\xref{sec:简并半导体}中简并半导体的理论了吗?这是图像观察不仔细所致的,这里,进入价带的是电子准费米能级,进入导带的是空穴准费米能级,反而是更加非简并了。\goodbreak

\subsection{非平衡PN结的载流子浓度}
在定性计算非平衡PN结的载流子浓度前,我们还要做出几点理想化的假设
\begin{enumerate}
    \item 小注入条件:注入的少数载流子浓度比平衡多数载流子浓度小得多。
    \item 突变耗尽层条件:耗尽层外的半导体是电中性的,即外加电压和接触电势差完全降落在耗尽层上,因此,注入的非平衡少数载流子在P区和N区的运动是纯扩散运动。
    \item 玻尔兹曼边界条件:耗尽层两端,载流子分布满足玻尔兹曼统计分布。
\end{enumerate}
在本小节,我们只需要计算载流子作为少子在扩散区的分布,因为\fancyref{fml:PN结的平衡载流子浓度}在载流子进入对方扩散区前都是仍然适用的,只不过要将$qV_D$替换\footnote{应指出的是,在\fancyref{fml:PN结多子浓度与少子浓度的关系}中的$qV_\text{D}$无需替换为$q(V_\text{D}-V)$,该关系不随外加电压而变。}为$q(V_\text{D}-V)$
\begin{Align}[10pt]
    n(x)&=\begin{cases}
        \mal{n_\text{N0}\exp[\frac{qV(x)-q(V_\text{D}-V)}{\kB T}]},&x\geq x_\text{p}\\
        \mal{n_\text{P}(x)},&x<x_\text{p}
    \end{cases}\\
    p(x)&=\begin{cases}
        \mal{p_\text{P0}\exp[\frac{-qV(x)}{\kB T}]},&\hspace{5.4em}x\leq x_\text{n}\\
        \mal{p_\text{N}(x)},&\hspace{5.4em}x>x_\text{n}
    \end{cases}
\end{Align}
这里我们用$V$的正负来表示正偏压和负偏压,若$V$为正即正偏压,若$V$为负即负偏压。

这里我们要计算的就是$n_\text{P}(x)$和$p_\text{N}(x)$,平衡时两者即$n_\text{P0}$和$p_\text{N0}$的常数,当外加偏压时,两者分别是由PN结势垒边界的浓度$n(x_\text{p})$和$p(x_\text{n})$向PN结两端的$n_\text{P0}$和$p_\text{N0}$的扩散过程。

\begin{BoxFormula}[PN结外加偏压时的少子浓度]
    PN结外加偏压$V$时,在N区,空穴的载流子浓度为
    \begin{Equation}
        p_\text{N}(x)=p_\text{N0}+p_\text{N0}\qty[\exp(\frac{qV}{\kB T})-1]\exp(\frac{x_\text{n}-x}{L_\text{p}})
    \end{Equation}
    PN结外加偏压$V$时,在P区,电子的载流子浓度为
    \begin{Equation}
        n_\text{P}(x)=n_\text{P0}+n_\text{P0}\qty[\exp(\frac{qV}{\kB T})-1]\exp(\frac{x-x_\text{p}}{L_\text{n}})
    \end{Equation}
\end{BoxFormula}

\begin{Proof}
    根据\fancyref{fml:非平衡态下的载流子浓度积}
    \begin{Equation}&[1]
        np=n_\text{i}^2\exp\qty(\frac{E_\text{Fn}-E_\text{Fp}}{\kB T})
    \end{Equation}
    势垒区P区边界$x=x_\text{P}$处,如\xref{fig:正向偏压下的能带结构}和\xref{fig:反向偏压下的能带结构}所示,有$E_\text{Fn}-E_\text{Fp}=qV$,故
    \begin{Equation}&[2]
        n_\text{P}(x_\text{p})p_\text{P}(x_\text{p})=n_\text{i}^2\exp(\frac{qV}{\kB T})
    \end{Equation}
    这里$p_\text{P}(x_\text{P})$作为P区多数载流子,可以作为常数$p_\text{P0}$代入
    \begin{Equation}&[3]
        n_\text{P}(x_\text{p})p_\text{P0}=n_\text{i}^2\exp(\frac{qV}{\kB T})
    \end{Equation}
    而考虑到$n_\text{i}^2=n_\text{P0}p_\text{P0}$
    \begin{Equation}&[5]
        n_\text{P}(x_\text{p})=n_\text{P0}\exp(\frac{qV}{\kB T})
    \end{Equation}
    所以说,注入P区边界的非平衡少数载流子浓度为
    \begin{Equation}&[6]
        \delt{n_\text{P}}(x_\text{p})=n_\text{P0}\qty[\exp(\frac{qV}{\kB T})-1]
    \end{Equation}
    类似的,注入N区边界的非平衡少数载流子浓度为
    \begin{Equation}&[7]
        \delt{p_\text{N}}(x_\text{n})=p_\text{N0}\qty[\exp(\frac{qV}{\kB T})-1]
    \end{Equation}
    这两式子就是连续性方程求解所需的边界条件了。
    
    根据\fancyref{eqt:连续性方程},在稳定态下,在N区中扩散的非平衡空穴的连续性方程为
    \begin{Equation}&[8]
        D_\text{p}\dv[2]{\delt{p_\text{N}}}{x}-\mu_\text{p}\Emf\dv{\delt{p_\text{N}}}{x}-\mu_\text{p}p_\text{N}\dv{\Emf}{x}-\frac{\delt{p_\text{N}}}{\tau_\text{p}}=0
    \end{Equation}
    在我们的假设下,外加电场和内建电场都仅存在于势垒区,扩散区无电场,故
    \begin{Equation}&[9]
        D_\text{P}\dv[2]{\delt{p_\text{N}}}{x}-\frac{\delt{p_\text{N}}}{\tau_\text{p}}=0
    \end{Equation}
    这是一个最简单的二阶微分方程,其通解为(其中$L_\text{p}=\sqrt{D_\text{p}\tau_\text{p}}$是扩散长度)
    \begin{Equation}&[10]
        \delt{p}_\text{N}(x)=A\exp(-\frac{x}{L_\text{p}})+B\exp(\frac{x}{L_\text{p}})
    \end{Equation}
    由于$\delt{p}_\text{N}$是在正半轴的N区上,故$B=0$舍去正指数项
    \begin{Equation}&[11]
        \delt{p}_\text{N}(x)=A\exp(-\frac{x}{L_\text{p}})
    \end{Equation}
    令$x=x_\text{n}$
    \begin{Equation}&[12]
        \delt{p}_\text{N}(x_\text{n})=A\exp(-\frac{x_\text{n}}{L_\text{p}})
    \end{Equation}
    将\xrefpeq{12}和\xrefpeq{7}联立
    \begin{Equation}&[13]
        A\exp(-\frac{x_\text{n}}{L_\text{p}})=p_\text{N0}\qty[\exp(\frac{qV}{\kB T})-1]
    \end{Equation}
    解得常数$A$
    \begin{Equation}&[14]
        A=p_\text{N0}\qty[\exp(\frac{qV}{\kB T})-1]\exp(\frac{x_\text{n}}{L_\text{p}})
    \end{Equation}
    将\xrefpeq{14}代回\xrefpeq{11}
    \begin{Equation}
        \delt{p_\text{N}}(x)=p_\text{N0}\qty[\exp(\frac{qV}{\kB T})-1]\exp(\frac{x_\text{n}-x}{L_\text{p}})
    \end{Equation}
    而$p_\text{N}(x)=p_\text{N0}+\delt{p}_\text{N}(x)$,故
    \begin{Equation}
        p_\text{N}(x)=p_\text{N0}+p_\text{N0}\qty[\exp(\frac{qV}{\kB T})-1]\exp(\frac{x_\text{n}-x}{L_\text{p}})
    \end{Equation}
    类似的可以得到
    \begin{Equation}
        n_\text{P}(x)=n_\text{P0}+n_\text{P0}\qty[\exp(\frac{qV}{\kB T}-1)]\exp(\frac{x-x_\text{p}}{L_\text{n}})
    \end{Equation}
    这里$x_\text{n}-x$变为$x-x_\text{p}$与\xrefpeq{10}舍去正指数项还是负指数项有关。
\end{Proof}
\xref{fig:PN结的载流子浓度}极为生动的展现本小节的工作,即正偏和反偏下PN结的载流子分布将如何变化?

\subsection{PN结的电流电压特性}
当在PN结上加一定电压时,在PN结上就会通过一定的电流,这些电流就来自\xref{subsec:非平衡PN结的载流子浓度}中我们计算的少子扩散流。很明显,通过PN结的总电流密度$J$就等于势垒边界$x_\text{n}$和$x_\text{p}$处的空穴扩散电流密度$J_\text{p}(x_\text{n})$和电子扩散电流密度$J_\text{n}(x_\text{p})$,当然,我们会说,为什么偏偏选取势垒边界处的电流密度来计算?实际上,扩散过程伴随着复合,扩散流其实是越来越弱的,但根据电流连续性原理,这些电流不可能凭空消失,实际上,随着载流子的复合,少子的扩散电流逐渐转化为多子的漂移电流,因此,选取哪个截面都一样,那就不妨选取最便捷的势垒边界了。

\begin{BoxEquation}[肖克利方程]
    \uwave{肖克利方程}(Shockley Equation)描述了理想PN结的电流电压特性
    \begin{Equation}
        J=J_\text{s}\qty[\exp\qty(\frac{qV}{\kB T})-1]
    \end{Equation}
    其中$J_\text{s}$为
    \begin{Equation}
        J_\text{s}=\frac{qD_\text{n}n_\text{P0}}{L_\text{n}}+\frac{qD_\text{p}p_\text{N0}}{L_\text{p}}
    \end{Equation}
\end{BoxEquation}

\begin{Proof}
    根据\xref{fml:载流子的漂移扩散}和\fancyref{fml:PN结外加偏压时的少子浓度},计算空穴扩散流$J_\text{p}$
    \begin{Equation}&[1]
        J_\text{p}(x)=-qD_\text{p}\dv{p_\text{N}}{x}=\frac{qD_\text{p}}{L_\text{p}}p_\text{N0}\qty[\exp(\frac{qV}{\kB T})-1]\exp(\frac{x_\text{n}-x}{L_\text{p}})
    \end{Equation}
    根据\xref{fml:载流子的漂移扩散}和\fancyref{fml:PN结外加偏压时的少子浓度},计算电子扩散流$J_\text{n}$
    \begin{Equation}&[2]
        J_\text{n}(x)=qD_\text{n}\dv{n_\text{P}}{x}=\frac{qD_\text{p}}{L_\text{p}}n_\text{P0}\qty[\exp(\frac{qV}{\kB T})-1]\exp(\frac{x-x_\text{p}}{L_\text{n}})
    \end{Equation}
    在\xrefpeq{1}中代入$x=x_\text{n}$
    \begin{Equation}
        J_\text{p}(x_\text{n})=\frac{qD_\text{p}p_\text{N0}}{L_\text{p}}\qty[\exp(\frac{qV}{\kB T})-1]
    \end{Equation}
    在\xrefpeq{2}中代入$x=x_\text{p}$
    \begin{Equation}
        J_\text{n}(x_\text{p})=\frac{qD_\text{n}n_\text{P0}}{L_\text{n}}\qty[\exp(\frac{qV}{\kB T})-1]
    \end{Equation}
    因此
    \begin{Equation}*
        J=J_\text{p}(x_\text{n})+J_\text{n}(x_\text{p})=\qty(\frac{qD_\text{p}p_\text{N0}}{L_\text{p}}+\frac{qD_\text{n}n_\text{P0}}{L_\text{n}})\qty[\exp(\frac{qV}{\kB T})-1]
    \end{Equation}
    若引入$J_\text{s}$作为代换变量
    \begin{Equation}*
        J=J_\text{s}\qty[\exp(\frac{qV}{\kB T})-1]\qedhere
    \end{Equation}
\end{Proof}

根据\fancyref{eqt:肖克利方程}
\begin{itemize}
    \item 正向导通:PN结在正向偏压下,电流将随电压指数增大,故称为正向导通。
    \item 反向截止:PN结在反向偏压下,电流将趋于$-J_\text{s}$的定值,这表明,反向电流密度为是与外界常量无关的常数(称为反向饱和电流)。由于$J_\text{s}$很小,因此称PN结反向截止。
\end{itemize}
这表明,\empx{PN结具有单向导电性},正向导通,反向截止,这种特性也称为\uwave{整流效应}。
\section{PN结的非理想因素}

实验测量表明,\fancyref{eqt:肖克利方程}给出的理想PN结的电流电压方程式,在小注入条件下,与锗PN结的实验结果符合的较好,但是,与硅PN结的实验结果偏离较大。这表明在实际的PN结中,还存在诸多非理想因素,会使电流电压关系偏离理想状态,具体而言
\begin{itemize}
    \item 在正向偏压较小时,实验测得的正向电流比理论值偏大。
    \item 在正向偏压较大时,实验测得的正向电流比理论值偏小。
    \item 在反向偏压时,实验测得的反向电流是不饱和的,随着反向偏压路由增大。
\end{itemize}
引发上述的因素很多,这里主要讨论势垒区的产生与复合和大注入的影响。

\subsection{势垒区的产生电流}
在反向偏压下,势垒区内的电场加强,原先,势垒区中的复合率与产生率是平衡的,但是,在强电场的作用下,复合中心产生的电子空穴对来不及复合,就被强电场驱走了,也就是说势垒区内载流子的产生率大于复合率,具有净产生率,从而形成另一部分反向电流,即产生电流。

\begin{BoxFormula}[势垒区的产生电流]
    反向偏压下,势垒区的\uwave{产生电流密度}为
    \begin{Equation}
        J_\text{g}=\frac{qn_\text{i}X_\text{D}}{2\tau}
    \end{Equation}
    反向电流$J_\text{R}$是产生电流$J_\text{g}$与反向扩散电流$J_\text{RD}$的和,在P$^{+}$N结中
    \begin{Equation}
        J_\text{R}=J_\text{g}+J_\text{RD}=
        \frac{qD_\text{p}n_\text{i}^2}{L_\text{p}N_\text{D}}+
        \frac{qn_\text{i}X_\text{D}}{2\tau}
    \end{Equation}
    其中$X_\text{D}$为势垒宽度。
\end{BoxFormula}

\begin{Proof}
    根据\fancyref{fml:净复合率与复合中心能级}
    \begin{Equation}&[1]
        U=\frac{N_\text{t}r(np-n_\text{i}^2)}{n+p+2n_\text{i}\cosh(E_\text{t}-E_\text{i}/\kB T)}
    \end{Equation}
    近似认为复合中心能级$E_\text{t}$在费米能级$E_\text{F}$附近,即$E_\text{t}=E_\text{F}$
    \begin{Equation}&[2]
        U=\frac{N_\text{t}r(np-n_\text{i}^2)}{n+p+2n_\text{i}}
    \end{Equation}
    近似认为势垒区$n,p\ll n_\text{i}$
    \begin{Equation}&[3]
        U=-\frac{N_\text{t}rn_\text{i}^2}{2n_\text{i}}=-\frac{N_\text{t}rn_\text{i}}{2}
    \end{Equation}
    实际上\xref{fml:净复合率与复合中心能级}本身也是近似公式,其近似条件使$\tau=1/N_\text{t}r$,故
    \begin{Equation}&[4]
        U=-\frac{n_\text{i}}{2\tau}
    \end{Equation}
    这个负的净复合率,其实就是净产生率
    \begin{Equation}&[5]
        G=-U=\frac{n_\text{i}}{2\tau}
    \end{Equation}
    产生电流密度$J_\text{g}$应为净产生率$G$,乘以电荷量$q$和势垒宽度$X_\text{D}$
    \begin{Equation}&[6]
        J_\text{g}=qGX_\text{D}
    \end{Equation}
    将\xrefpeq{4}代入\xrefpeq{6}
    \begin{Equation}
        J_\text{g}=\frac{qn_\text{i}X_\text{D}}{2\tau}
    \end{Equation}
    这就得到了产生电流密度$J_\text{g}$的表达式。\goodbreak
    
    而反向扩散电流则依据\fancyref{eqt:肖克利方程}
    \begin{Equation}
        J_\text{RD}=J_\text{s}=\frac{qD_\text{n}n_\text{P0}}{L_\text{n}}+\frac{qD_\text{p}p_\text{N0}}{L_\text{p}}
    \end{Equation}
    在P$^{+}$N结中,只需要考虑空穴扩散
    \begin{Equation}
        J_\text{RD}=\frac{qD_\text{p}p_\text{N0}}{L_\text{p}}
    \end{Equation}
    由于$n_\text{N0}p_\text{N0}=n_\text{i}^2$,同时在N区$n_\text{N0}=N_\text{D}$,有
    \begin{Equation}
        J_\text{RD}=\frac{qD_\text{p}n_\text{i}^2}{L_\text{p}N_\text{D}}
    \end{Equation}
    这就得到了反向扩散电流密度$J_\text{RD}$的表达式。
\end{Proof}

\fancyref{fml:势垒区的产生电流}指出,PN结的反向电流$J_\text{R}$包含势垒区产生电流$J_\text{g}$与反向扩散电流$J_\text{RD}$两部分,后者属于原先理想模型的范畴。由于$J_\text{g}$与$J_\text{RD}$分别与$n_\text{i}$和$n_\text{i}^2$成正比,且\fancyref{fml:本征半导体的载流子浓度}指出$n_\text{i}$与禁带宽度$E_\text{g}$负相关,因此
\begin{itemize}
    \item 锗的禁带宽度$E_\text{g}$较小,故$n_\text{i}$较大,而考虑到$J_\text{g}\propto n_\text{i}$和$J_\text{RD}\propto n_\text{i}^2$,有$J_\text{g}\ll J_\text{RD}$,反向电流中反向扩散电流占主导地位,所以,锗PN结的反向电流与理想特性基本相符。
    \item 硅的禁带宽度$E_\text{g}$较大,故$n_\text{i}$较小,而考虑到$J_\text{g}\propto n_\text{i}$和$J_\text{RD}\propto n_\text{i}^2$,有$J_\text{g}\gg J_\text{RD}$,反向电流中势垒产生电流占主导地位,所以,硅PN结的反向电流将偏离理想特性。而我们知道,势垒宽度$X_\text{D}$将随反向偏压的增加逐渐变宽,且有$J_\text{g}\propto X_\text{D}$,因此,势垒区的产生电流$J_\text{g}$占还会随反向偏压的增加而增加,即,硅PN结的反向电流是非饱和的。
\end{itemize}

简而言之,禁带宽度小,则$J_\text{RD}$占主导,近似理想,禁带宽度大,则$J_\text{g}$占主导。

\subsection{势垒区的复合电流}
在正向偏压下,势垒区存在大量从P区和N区注入的载流子,但实际上,这些载流子会在势垒区中会复合一部分,因此,P区和N区就要注入额外的空穴和电子来弥补这种损失,这就是所谓的复合电流\cite{W10}。因此,复合电流并非因为复合而产生的电流,恰相反,复合电流是弥补复合造成的损失导致的电流。便于记忆,我们可以将复合电流和产生电流对应起来看
\begin{itemize}
    \item 正向偏压下出现的非理想因素,是势垒区的复合电流。
    \item 反向偏压下出现的非理想因素,是势垒区的产生电流。
\end{itemize}

求解复合电流的过程与产生电流类似,但略微复杂些。

\begin{BoxFormula}[势垒区的复合电流]*
    正向偏压下,势垒区的\uwave{复合电流密度}为
    \begin{Equation}
        J_\text{R}=\frac{qn_\text{i}X_\text{D}}{2\tau}\exp(\frac{qV}{2\kB T})
    \end{Equation}
    正向电流$J_\text{F}$是复合电流$J_\text{r}$与正向扩散电流$J_\text{FD}$的和,在P$^{+}$N结中
    \begin{Equation}
        \qquad\qquad\quad
        J_\text{F}=J_\text{r}+J_\text{FD}=
        \frac{qn_\text{i}X_\text{D}}{2\tau}\exp(\frac{qV}{2\kB T})+
        \frac{qD_\text{p}n_\text{i}^2}{L_\text{p}N_\text{D}}\exp(\frac{qV}{\kB T})
        \qquad\qquad\quad
    \end{Equation}
    其中$X_\text{D}$为势垒宽度。
\end{BoxFormula}

\begin{Proof}
    根据\fancyref{fml:净复合率与复合中心能级}
    \begin{Equation}&[1]
        U=\frac{N_\text{t}r(np-n_\text{i}^2)}{n+p+2n_\text{i}\cosh(E_\text{t}-E_\text{i}/\kB T)}
    \end{Equation}
    近似认为复合中心能级$E_\text{t}$在费米能级$E_\text{F}$附近,即$E_\text{t}=E_\text{F}$
    \begin{Equation}&[2]
        U=\frac{N_\text{t}r(np-n_\text{i}^2)}{n+p+2n_\text{i}}
    \end{Equation}
    但有所不同的是,这里不能再近似认为$n,p\ll n_\text{i}$,这可能是因为,如\xref{fig:PN结的载流子浓度}所示,势垒区的载流子浓度,在反偏时减小,在正偏时增加,而这里讨论复合电流时PN结是正偏的。

    根据\fancyref{fml:非平衡态下的载流子浓度积}
    \begin{Equation}&[3]
        np=n_\text{i}^2\exp(\frac{E_\text{Fn}-E_\text{Fp}}{\kB T})
    \end{Equation}
    在PN结的势垒区中,准费米能级之差$E_\text{Fn}-E_\text{Fp}$即外加偏压乘元电荷$qV$
    \begin{Equation}&[4]
        np=n_\text{i}^2\exp(\frac{qV}{\kB T})
    \end{Equation}
    在$n=p$处,电子和空穴相遇的机会最大,该处净复合率最大。作为估算,我们可以认为势垒区中各处的净复合率$U$都等于$n=p$处的最大净复合率$U_{\max}$,而$n=p$时由\xrefpeq{4}
    \begin{Equation}&[5]
        n=p=n_\text{i}\exp(\frac{qV}{2\kB T})
    \end{Equation}
    将\xrefpeq{4}和\xrefpeq{5}代入\xrefpeq{3}
    \begin{Equation}&[6]
        U_{\max}=
        \frac{N_\text{t}r[n_\text{i}^2\exp(qV/\kB T)-n_\text{i}^2]}{2[n_\text{i}\exp(qV/2\kB T)+n_\text{i}]}
    \end{Equation}
    不妨提出$n_\text{i}$
    \begin{Equation}&[7]
        U_{\max}=
        \frac{N_\text{t}rn_\text{i}[\exp(qV/\kB T)-1]}{2[\exp(qV/2\kB T)+1]}
    \end{Equation}
    当$qV\gg\kB T$时
    \begin{Equation}&[8]
        U_{\max}=
        \frac{N_\text{t}rn_\text{i}\exp(qV/\kB T)}{2\exp(qV/2\kB T)}
    \end{Equation}
    即
    \begin{Equation}&[9]
        U_{\max}=
        \frac{N_\text{t}rn_\text{i}}{2}\exp(\frac{qV}{2\kB T})
    \end{Equation}
    和计算产生电流时一样,代入$\tau=1/N_\text{t}r$
    \begin{Equation}&[9]
        U_{\max}=
        \frac{n_\text{i}}{2\tau}\exp(\frac{qV}{2\kB T})
    \end{Equation}
    复合电流密度$J_\text{r}$应为净复合率$U$,乘以元电荷$q$和势垒宽度$X_\text{D}$
    \begin{Equation}&[10]
        J_\text{r}=qU_{\max}X_\text{D}
    \end{Equation}
    将\xrefpeq{9}代入\xrefpeq{10}
    \begin{Equation}&[11]
        J_\text{r}=\frac{qn_\text{i}X_\text{D}}{2\tau}\exp(\frac{qV}{2\kB T})
    \end{Equation}
    这就得到了复合电流密度$J_\text{r}$的表达式。

    而正向扩散电流则依据\fancyref{eqt:肖克利方程}
    \begin{Equation}&[12]
        J_\text{FD}=J_\text{s}\qty[\exp(\frac{qV}{\kB T})-1]
    \end{Equation}
    这里$J_\text{s}=J_\text{RD}$,代入\fancyref{fml:势垒区的产生电流}中的结果(对于P$^{+}$N结)
    \begin{Equation}&[13]
        J_\text{FD}=\frac{qD_\text{p}n_\text{i}^2}{L_\text{p}N_\text{D}}\qty[\exp(\frac{qV}{\kB T})-1]
    \end{Equation}
    当$qV\gg\kB T$时
    \begin{Equation}
        J_\text{FD}=\frac{qD_\text{p}n_\text{i}^2}{L_\text{p}N_\text{D}}\exp(\frac{qV}{\kB T})
    \end{Equation}
    这就得到了正向扩散电流密度$J_\text{FD}$的表达式。
\end{Proof}

\fancyref{fml:势垒区的复合电流}告诉我们,在正向偏压下
\begin{itemize}
    \item 扩散电流$J_\text{FD}$的特点是与$\exp(qV/\kB T)$成正比。
    \item 复合电流$J_\text{r}$的特点是与$\exp(qV/2\kB T)$成正比。
\end{itemize}
因此,可以用以下经验公式表示正向电流密度
\begin{Equation}
    J_\text{F}\propto\exp(\frac{qV}{m\kB T})
\end{Equation}
当扩散电流$J_\text{RD}$占主导时$m=1$,当复合电流$J_\text{r}$占主导时$m=2$。

除此之外,扩散电流与复合电流的比值是
\begin{Equation}
    \frac{J_\text{FD}}{J_\text{r}}=
    \frac{qD_\text{p}n_\text{i}^2/L_\text{p}N_\text{D}}{qn_\text{i}X_\text{D}/2\tau}\exp(\frac{qV}{2\kB T})
\end{Equation}
即
\begin{Equation}
    \frac{J_\text{FD}}{J_\text{r}}=
    \frac{2\tau n_\text{i}D_\text{p}}{L_\text{p}N_\text{D}X_\text{D}}\exp(\frac{qV}{2\kB T})
\end{Equation}
根据\fancyref{def:扩散长度},这里$D_\text{p}=\sqrt{L_\text{p}\tau}$
\begin{Equation}
    \frac{J_\text{FD}}{J_\text{r}}=\frac{2n_\text{i}}{N_\text{D}X_\text{D}}\exp(\frac{qV}{2\kB T})
\end{Equation}
由此可见,$J_\text{FD}/J_\text{r}$与$n_\text{i}$和外加电压$V$有关
\begin{itemize}
    \item 当外加偏压$V$较小时,$J_\text{FD}/J_\text{r}$较小,复合电流占主导地位。
    \item 当外加偏压$V$较大时,$J_\text{FD}/J_\text{r}$较大,扩散电流占主导地位。
\end{itemize}
由于锗的禁带宽度较小,而$n_\text{i}$较小,因此复合电流在锗PN结中总是可以忽略。

\subsection{大注入情况}
过去我们对PN结的讨论,都是在小注入情况的背景下进行的
\begin{itemize}
    \item 小注入是指,注入的非平衡载流子远小于该区多子浓度的情况。
    \item 大注入是指,注入的非平衡载流子接近或超过该区多子浓度的情况。
\end{itemize}
随着正向偏压增大,注入的非平衡载流子相应增多,PN结的状态就会逐渐由小注入转变为大注入。在本小节,我们将讨论大注入情况带来的非理想特性,主要以P$^{+}$N结为例进行研究。

由于P$^{+}$N结的正向电流是由P$^{+}$区注入N区的空穴电流,因此我们只需要讨论空穴扩散区,即N区内的情况。当大注入时,首先,电注入到达势垒区边界的空穴浓度$\delt{p_\text{n}}(x_\text{n})$很大,接近或超过N区多子浓度$n_\text{N0}=N_\text{D}$,而事实是,当$\delt{p_\text{N}}(x_\text{n})$在N区扩散并形成稳定的浓度分布$\delt{p_\text{N}}$时,这会破坏N区的电中性,为保持电中性,其实会有相同分布的$\delt{n_\text{N}}$产生,过去在小注入情况下,由于$\delt{p_\text{N}}=\delt{n_\text{N}}$相对$n_\text{N0}$很小,$\delt{n_\text{N}}$可以忽略,但,在大注入情况下,两者来到了同一数量级,$\delt{n_\text{N}}$的影响需要被妥善考虑,这就是大注入在定量分析上的影响。

根据我们刚刚的讨论,应有\setpeq{大注入}
\begin{Equation}&[1]
    \delt{p_\text{N}}=\delt{n_\text{N}}
\end{Equation}
所以,两者的梯度也应当是相等的
\begin{Equation}&[2]
    \dv{\delt{p_\text{N}}}{x}=\dv{\delt{n_\text{N}}}{x}
\end{Equation}
然而,非平衡电子$\delt{p_\text{N}}$的产生虽然平衡了注入的非平衡空穴$\delt{n_\text{N}}$,但因为电子离开了原来的位置,这会产生一个内建电场,由电子产生的内建电场,自然将会使电子自身的漂移作用与扩散作用抵消,维持$\delt{n_\text{N}}$的稳定分布,即$J_\text{n}=0$。然而,该内建电场会反过来使得空穴的运动加速。这我们会在后面更详细的讨论,而眼下一个更重要的问题,由于现在势垒区和扩散区都具有内建电场,我们就不能仅认为外加电场$V$降落在势垒区了,而要
\begin{Equation}&[3]
    V=V_\text{J}+V_\text{P}
\end{Equation}
其中,$V_\text{J}$和$V_\text{P}$是外加电压分别在势垒区和空穴扩散区的分压。

现计算大注入时流过$x=x_\text{n}$截面处的电流密度,根据\fancyref{fml:载流子的漂移扩散}
\begin{Gather}[10pt]
    J_\text{p}=q\mu_\text{p}p_\text{N}(x_\text{n})\Emf(x_\text{n})-\eval{qD_\text{p}\dv{\delt{p_\text{N}}}{x}}_{x=x_\text{n}}\xlabelpeq{4}\\
    J_\text{n}=q\mu_\text{n}n_\text{N}(x_\text{n})\Emf(x_\text{n})+\eval{qD_\text{n}\dv{\delt{n_\text{N}}}{x}}_{x=x_\text{n}}\xlabelpeq{5}
\end{Gather}
根据前面的讨论$J_\text{n}=0$,由\xrefpeq{5}可知
\begin{Equation}&[6]
    \Emf(x_\text{n})=-\frac{qD_\text{n}}{q\mu_\text{n}n_\text{N}(x_\text{n})}\eval{\dv{\delt{n_\text{N}}}{x}}_{x=x_\text{n}}
\end{Equation}
即
\begin{Equation}&[7]
    \Emf(x_\text{n})=-\frac{D_\text{n}}{\mu_\text{n}n_\text{N}(x_\text{n})}\eval{\dv{\delt{n_\text{N}}}{x}}_{x=x_\text{n}}
\end{Equation}
根据\fancyref{law:爱因斯坦关系式}
\begin{Equation}&[8]
    \frac{D_\text{n}}{\mu_\text{n}}=
    \frac{D_\text{p}}{\mu_\text{p}}=
    \frac{\kB T}{q}
\end{Equation}
以及\xrefpeq{2},\xrefpeq{7}可以表示为
\begin{Equation}&[9]
    \Emf(x_\text{n})=-\frac{D_\text{p}}{\mu_\text{p}n_\text{N}(x_\text{n})}\eval{\dv{\delt{p_\text{N}}}{x}}_{x=x_\text{n}}
\end{Equation}
将\xrefpeq{9}代回\xrefpeq{4}
\begin{Equation}&[10]
    J_\text{p}=-qD_\text{p}\qty[\frac{p_\text{N}(x_\text{n})}{n_\text{N}(x_\text{n})}+1]\eval{\dv{\delt{p_\text{N}}}{x}}_{x=x_\text{n}}
\end{Equation}
这表明,当扩散区存在内建电场时,空穴电流密度在形式上仍然可以表达为原先扩散电流的形式,但是由于电场的漂移作用,空穴的扩散系数由$D_\text{p}$增大至$D_\text{p}[1+p_\text{N}(x_\text{n})/n_\text{N}(x_\text{n})]$。

在大注入时,可以近似认为$n_\text{N}(x_\text{n})=\delt{n_\text{N}}(x_\text{n})$和$p_\text{N}(x_\text{n})=\delt{p_\text{N}}(x_\text{n})$,故
\begin{Equation}&[11]
    J_\text{p}=-2qD_\text{p}\eval{\dv{\delt{p_\text{N}}}{x}}_{x=x_\text{n}}
\end{Equation}

这表明,大注入时,空穴的扩散系数由$D_\text{p}$增大至$2D_\text{p}$,扩散电流和漂移电流各占一半。

而我们可以证明(搞不太清楚为啥)
\begin{Equation}&[12]
    p_\text{N}(x_\text{n})=p_\text{N0}\exp(\frac{qV_\text{J}}{\kB T})\qquad
    n_\text{N}(x_\text{n})=n_\text{N0}\exp(\frac{qV_\text{P}}{\kB T})
\end{Equation}
因此
\begin{Equation}&[13]
    p_\text{N}(x_\text{n})n_\text{N}(x_\text{n})=n_\text{N0}p_\text{N0}\exp[\frac{q(V_\text{J}+V_\text{P})}{\kB T}]=n_\text{i}^2\exp(\frac{qV}{\kB T})
\end{Equation}
因为$p_\text{N}(x_\text{n})=n_\text{N}(x_\text{n})$
\begin{Equation}&[14]
    p_\text{N}(x_\text{n})=n_\text{N}(x_\text{n})=n_\text{i}\exp(\frac{qV}{2\kB T})
\end{Equation}
将空穴扩散区内的扩散视为线性分布,即
\begin{Equation}&[15]
    \eval{\dv{\delt{p_\text{N}}}{x}}_{x=x_\text{n}}=\frac{p_\text{N}(x_\text{n})-p_\text{N0}}{L_\text{p}}
\end{Equation}
由于$p_\text{N}(x_\text{n})\gg p_\text{N0}$,将\xrefpeq{14}代入\xrefpeq{15}
\begin{Equation}&[16]
    \eval{\dv{\delt{p_\text{N}}}{x}}_{x=x_\text{n}}=\frac{n_\text{i}^2}{L_\text{p}}\exp(\frac{qV}{2\kB T})
\end{Equation}
将\xrefpeq{16}代入\xrefpeq{11}
\begin{Equation}
    J_\text{p}=-\frac{2qD_\text{p}n_\text{i}}{L_\text{p}}\exp(\frac{qV}{2\kB T})
\end{Equation}
在P$^{+}$N结中,$J_\text{F}$实际就是$J_\text{p}$,因此
\begin{BoxFormula}[大注入情况]
    对于P$^{+}$N结,大注入情况下的电流电压关系为\footnote[2]{为什么这里电流是负的?}
    \begin{Equation}
        J_\text{F}=-\frac{2qD_\text{p}n_\text{i}}{L_\text{p}}\exp(\frac{qV}{2\kB T})
    \end{Equation}
\end{BoxFormula}
综上,考虑复合电流和大注入后,当P$^{+}$N结加正向偏压时,电流电压关系可以表示为
\begin{Equation}
    J_\text{F}\propto\exp(\frac{qV}{m\kB T})
\end{Equation}
其中$m$在$1$至$2$间取值
\begin{itemize}
    \item 当电压很小时,复合电流占主导,此时$m=2$。
    \item 当电压适中时,扩散电流占主导,此时$m=1$。
    \item 当电压很大时,适用大注入情况,此时$m=2$。
\end{itemize}
由于复合电流和大注入时$m=2$,即$J_\text{F}$的变化在这两个阶段慢于理想模型下的扩散电流,因此,以电压适中的扩散电流段为中心,电压很小时实际值偏高,电压很大时实际值偏低。\nopagebreak
\section{PN结电容}
PN结具有整流效应,但是它又包含着破坏整流效应的因素。当PN结在低频电压下,整流特性很好,但是,当电压频率增大,整流特性将变坏。频率为什么会对PN结的整流作用产生影响?这就因为,\empx{PN结具有电容特性}。那么,PN结为什么会有电容?PN结的电容大小与哪些因素有关?这就是本节要解决的问题,此外,本节还将推导出电势和势垒宽度的表达式。

\subsection{PN结电容的分类}

PN结电容包含势垒电容和扩散电容两部分,但无论是哪一部分,它们都会随外加电压的变化而变化,换言之,它们是可变电容,因此,我们常用微分电容的概念来表示PN结的电容,即
\begin{Equation}
    C=\dv{Q}{V}
\end{Equation}

\subsubsection{势垒电容}
当PN结外加正向偏压时,势垒区的电场随正向偏压而减弱,势垒区的宽度变窄,空间电荷数量减小。空间电荷的实质是由不能移动的杂质离子组成,因此,空间电荷的减少是由于N区的电子和P区的空穴中和了势垒区中的一部分电离施主和电离受主,形象的说,我们可以认为,当外加正向偏压增加时,有一部分电子和空穴“存入”势垒区。相反,当外加正向偏压减弱时,势垒区宽度增大,施主和受主重新电离,这部分电子和空穴将被“取出”势垒区。总而言之,PN结外加电压的变化,引起了电子和空穴在势垒区的“存入”和“取出”作用,相当于一个电容器的充电和放电。这就构成了\uwave{势垒电容}(Barrier Capacitance),以$C_\text{T}$表示。

但需要指出的是,势垒电容中的$Q$实际是指势垒区中的空间电荷。


\subsubsection{扩散电容}
当PN结外加正向偏压时,扩散区将存在非平衡少数载流子$\delt{p_\text{N}}$和$\delt{n_\text{P}}$的分布,而为了保持电中性,扩散区也会产生相应数量的非平衡多数载流子$\delt{n_\text{N}}$和$\delt{p_\text{P}}$,这些非平衡载流子的数量将随外加电压的变化而变,这就构成了\uwave{扩散电容}(Diffusion Capacitance),以$C_\text{D}$表示。

\subsection{突变结的势垒电容}
在本小节,我们将依次讨论突变结的电场分布与电势分布、势垒宽度、势垒电容。

根据\xref{subsec:合金法}的内容,对于突变结,其势垒区的空间电荷密度为
\begin{Equation}[突变结]
    \rho(x)=\begin{cases}
        -qN_\text{A},&x_\text{p}<x<0\\
        +qN_\text{D},&0<x<x_\text{n}
    \end{cases}
\end{Equation}\nopagebreak
左半P区,产生空穴的电离受主$N_\text{A}$带负电。右半N区,产生电子的电离施主$N_\text{D}$带正电。\goodbreak

% 势垒区的宽度为
% \begin{Equation}
%     X_\text{D}=x_\text{n}-x_\text{p}
% \end{Equation}
% 势垒区的正负电荷总量应相等,因为半导体整体满足电中性条件
% \begin{Equation}
%     Q=-qN_\text{A}x_\text{p}=qN_\text{D}x_\text{n}
% \end{Equation}
% 这里的$Q$是势垒区中单位面积上所鸡肋的

\begin{BoxFormula}[突变结的电场分布]*
    在突变结中,电场分布满足
    \begin{Equation}
        \Emf(x)=\begin{cases}
            \Emf_1(x)=\mal{\frac{qN_\text{A}(x_\text{p}-x)}{\varepsilon_\text{r}\varepsilon_0}},&x_\text{p}<x<0\\[4mm]
            \Emf_2(x)=\mal{\frac{qN_\text{D}(x-x_\text{n})}{\varepsilon_\text{r}\varepsilon_0}},&0<x<x_\text{n}
        \end{cases}
    \end{Equation}
    在$x=0$处,电场取最大值
    \begin{Equation}
        \Emf_\text{m}=\frac{qN_\text{A}x_\text{p}}{\varepsilon_\text{r}\varepsilon_0}=-\frac{qN_\text{D}x_\text{n}}{\varepsilon_\text{r}\varepsilon_0}
    \end{Equation}
\end{BoxFormula}
\begin{Proof}
    根据电磁学中的泊松方程$\laplacian V=-\rho/\varepsilon_\text{r}\varepsilon_0$,改写为一维形式,代入\xrefeq{突变结}
    \begin{Equation}&[1]
        \dv[2]{V_1}{x}=\frac{qN_\text{A}}{\varepsilon_\text{r}\varepsilon_0}\qquad
        \dv[2]{V_2}{x}=-\frac{qN_\text{D}}{\varepsilon_\text{r}\varepsilon_0}
    \end{Equation}
    这里$V_1,V_2$分别为负电荷区$x_\text{p}<x<0$和正电荷区$0<x<x_\text{n}$的电势。

    将\xrefpeq{1}积分一次得
    \begin{Equation}&[2]
        \dv{V_1}{x}=\qty(\frac{qN_\text{A}}{\varepsilon_\text{r}\varepsilon_0})x+C_1\qquad
        \dv{V_2}{x}=-\qty(\frac{qN_\text{D}}{\varepsilon_\text{r}\varepsilon_0})x+C_2
    \end{Equation}
    由于$\Emf=-\dv*{V}{x}$
    \begin{Equation}&[3]
        \Emf_1(x)=-\qty(\frac{qN_\text{A}}{\varepsilon_\text{r}\varepsilon_0})x-C_1\qquad
        \Emf_2(x)=\qty(\frac{qN_\text{D}}{\varepsilon_\text{r}\varepsilon_0})x-C_2
    \end{Equation}
    由于势垒区边界的电场为零
    \begin{Equation}&[3.5]
        \Emf_1(x_\text{p})=0\qquad
        \Emf_2(x_\text{n})=0
    \end{Equation}
    即
    \begin{Equation}&[4]
        \qquad\qquad\qquad
        \Emf_1(x_\text{p})=-\qty(\frac{qN_\text{A}}{\varepsilon_\text{r}\varepsilon_0})x_\text{p}-C_1=0\qquad
        \Emf_2(x_\text{n})=\qty(\frac{qN_\text{D}}{\varepsilon_\text{r}\varepsilon_0})x_\text{n}-C_2=0
        \qquad\qquad\qquad
    \end{Equation}
    定出积分常数
    \begin{Equation}&[5]
        C_1=-\frac{qN_\text{A}x_\text{p}}{\varepsilon_\text{r}\varepsilon_0}\qquad
        C_2=\frac{qN_\text{D}x_\text{n}}{\varepsilon_\text{r}\varepsilon_0}
    \end{Equation}
    将\xrefpeq{5}代入\xrefpeq{3}
    \begin{Equation}*
        \Emf_1(x)=\mal{\frac{qN_\text{A}(x_\text{p}-x)}{\varepsilon_\text{r}\varepsilon_0}}\qquad
        \Emf_2(x)=\mal{\frac{qN_\text{D}(x-x_\text{n})}{\varepsilon_\text{r}\varepsilon_0}}\qedhere
    \end{Equation}
\end{Proof}

\begin{BoxFormula}[突变结的电势分布]
    在突变结中,电势分布满足
    \begin{Equation}
        V(x)=
        \begin{cases}
            \mal{V_1(x)=\frac{qN_\text{A}}{2\varepsilon_\text{r}\varepsilon_0}(x-x_\text{p})^2},&x_\text{p}<x<0\\[4mm]
            \mal{V_2(x)=V_\text{D}-\frac{qN_\text{A}}{2\varepsilon_\text{r}\varepsilon_0}(x-x_\text{n})^2},&0<x<x_\text{n}
        \end{cases}
    \end{Equation}
    其中接触电势差$V_\text{D}$为
    \begin{Equation}
        V_\text{D}=\frac{q(N_\text{A}x_\text{p}^2+N_\text{D}x_\text{n}^2)}{2\varepsilon_\text{r}\varepsilon_0}
    \end{Equation}
\end{BoxFormula}

\begin{Proof}
    顺承\fancyref{fml:突变结的电场分布}的结果,考虑到$\Emf=-\dv*{V}{x}$
    \begin{Equation}&[1]
        \dv{V_1}{x}=\frac{qN_\text{A}(x-x_\text{p})}{\varepsilon_\text{r}\varepsilon_0}\qquad
        \dv{V_2}{x}=-\frac{qN_\text{D}(x-x_\text{n})}{\varepsilon_\text{r}\varepsilon_0}
    \end{Equation}
    积分得
    \begin{Equation}
        \qquad\qquad\qquad
        V_1(x)=\frac{qN_\text{A}}{2\varepsilon_\text{r}\varepsilon_0}(x-x_\text{p})^2+D_1\qquad
        V_2(x)=-\frac{qN_\text{D}}
        {2\varepsilon_\text{r}\varepsilon_0}(x-x_\text{n})^2+D_2
        \qquad\qquad\qquad
    \end{Equation}
    电势的参考点是可以选取的,按我们前面的习惯,取P型区一侧的电势为零
    \begin{Equation}
        V_1(x_\text{p})=0\qquad
        V_2(x_\text{n})=V_\text{D}
    \end{Equation}
    这里$V_\text{D}$仍然是待定的常数,但它的意义是明确的,即接触电势差。

    这样一来,容易定出$D_1=0, D_2=V_\text{D}$
    \begin{Equation}
        V_1(x)=\frac{qN_\text{A}}{2\varepsilon_\text{r}\varepsilon_0}(x-x_\text{p})^2\qquad
        V_2(x)=V_\text{D}-\frac{qN_\text{D}}
        {2\varepsilon_\text{r}\varepsilon_0}(x-x_\text{n})^2
    \end{Equation}
    电势在$x=0$处应当连续
    \begin{Equation}
        V_1(0)=V_2(0)
    \end{Equation}
    从而
    \begin{Equation}
        \frac{qN_\text{A}}{2\varepsilon_\text{r}\varepsilon_0}x_\text{p}^2=
        V_\text{D}-\frac{qN_\text{D}}
        {2\varepsilon_\text{r}\varepsilon_0}x_\text{n}^2
    \end{Equation}
    得到
    \begin{Equation}*
        V_\text{D}=\frac{q(N_\text{A}x_\text{p}^2+N_\text{D}x_\text{n}^2)}{2\varepsilon_\text{r}\varepsilon_0}\qedhere
    \end{Equation}
\end{Proof}

由此可见,如\xref{fig:突变结}所示,突变结的电势是线性函数,在$x=0$处取最大值,向势垒区两端逐渐减小,在势垒区边缘$x=x_\text{n}, x=x_\text{p}$减小至零,电场始终为负,这代表电场总是会由N区指向P区。突变结的电势是抛物线,确切的说,是一个开口向上的抛物型的右半与一个开口向下的抛物线的左半的衔接,先前\xref{fig:PN结的载流子浓度与电势分布}等中的$V(x)$均是参照此处突变结的结果绘制的。

\begin{Figure}[突变结]
    \begin{FigureSub}[突变结的电荷密度]
        \includegraphics[width=4.5cm]{build/Chapter06E_01.fig.pdf}
    \end{FigureSub}
    \begin{FigureSub}[突变结的电场]
        \includegraphics[width=4.5cm]{build/Chapter06E_02.fig.pdf}
    \end{FigureSub}
    \begin{FigureSub}[突变结的电势]
        \includegraphics[width=4.5cm]{build/Chapter06E_03.fig.pdf}
    \end{FigureSub}
\end{Figure}

接下来,我们计算突变结的势垒宽度
\begin{BoxFormula}[突变结的势垒宽度]
    在突变结中,势垒宽度满足
    \begin{Equation}
        X_\text{D}=\sqrt{\frac{2\varepsilon_\text{r}\varepsilon_0(N_\text{A}+N_\text{D})(V_\text{D}-V)}{qN_\text{A}N_\text{D}}}
    \end{Equation}
\end{BoxFormula}
\begin{Proof}
    根据\fancyref{fml:突变结的电势分布}
    \begin{Equation}&[1]
        V_\text{D}=\frac{q(N_\text{A}x_\text{p}^2+N_\text{D}x_\text{n}^2)}{2\varepsilon_\text{r}\varepsilon_0}
    \end{Equation}
    势垒宽度$X_\text{D}$可以表示为
    \begin{Equation}&[2]
        X_\text{D}=x_\text{n}-x_\text{p}
    \end{Equation}
    另一方面,由于负电荷区的电荷$-qN_\text{A}x_\text{p}$与正电荷区的电荷$qN_\text{D}x_\text{n}$应相同,保持电中性
    \begin{Equation}&[3]
        N_\text{D}x_\text{n}=-N_\text{A}x_\text{p}
    \end{Equation}
    联立\xrefpeq{2}和\xrefpeq{3}
    \begin{Equation}&[4]
        \begin{pmatrix}
            1&-1\\
            N_\text{D}&N_\text{A}
        \end{pmatrix}
        \begin{pmatrix}
            x_\text{n}\\
            x_\text{p}
        \end{pmatrix}
        =
        \begin{pmatrix}
            X_\text{D}\\
            0
        \end{pmatrix}
    \end{Equation}
    即
    \begin{Equation}&[5]
        D=\begin{vmatrix}
            1&-1\\
            N_\text{D}&N_\text{A}
        \end{vmatrix}=N_\text{D}+N_\text{A}\qquad
        D_\text{n}=
        \begin{vmatrix}
            X_\text{D}&-1\\
            0&N_\text{A}
        \end{vmatrix}=N_\text{A}X_\text{D}\qquad
        D_\text{p}=
        \begin{vmatrix}
            1&X_\text{D}\\
            N_\text{D}&0
        \end{vmatrix}=-N_\text{D}X_\text{D}
    \end{Equation}
    通过\xrefpeq{2}和\xrefpeq{3}不难解出
    \begin{Equation}&[6]
        x_\text{n}=\frac{D_\text{n}}{D}=\frac{N_\text{A}X_\text{D}}{N_\text{D}+N_\text{A}}\qquad
        x_\text{p}=\frac{D_\text{p}}{D}=-\frac{N_\text{D}X_\text{D}}{N_\text{D}+N_\text{A}}
    \end{Equation}
    由此易得
    \begin{Equation}&[7]
        \qquad\qquad
        N_\text{D}x_\text{n}^2+
        N_\text{A}x_\text{p}^2=
        \frac{N_\text{D}N_\text{A}^2X_\text{D}^2+N_\text{A}N_\text{D}^2X_\text{D}^2}{(N_\text{D}+N_\text{A})^2}
        =
        \frac{N_\text{D}N_\text{A}X_\text{D}^2(N_\text{D}+N_\text{A})}
        {(N_\text{D}+N_\text{A})^2}
        \qquad\qquad
    \end{Equation}
    即
    \begin{Equation}&[8]
        N_\text{D}x_\text{n}^2+
        N_\text{A}x_\text{p}^2=
        \frac{N_\text{D}N_\text{A}X_\text{D}^2}
        {N_\text{D}+N_\text{A}}
    \end{Equation}
    将\xrefpeq{8}代入\xrefpeq{1}中,即得
    \begin{Equation}&[9]
        V_\text{D}=\frac{qN_\text{D}N_\text{A}X_\text{D}^2}{2\varepsilon_\text{r}\varepsilon_0(N_\text{A}+N_\text{D})}
    \end{Equation}
    由\xrefpeq{9},反解出$X_\text{D}$
    \begin{Equation}&[10]
        X_\text{D}=\sqrt{\frac{2\varepsilon_\text{r}\varepsilon_0(N_\text{A}+N_\text{D})V_\text{D}}{qN_\text{A}N_\text{D}}}
    \end{Equation}
    这是平衡状态下的PN结,而外加偏压时,势垒由$V_\text{D}$变为$V_\text{D}-V$
    \begin{Equation}
        X_\text{D}=\sqrt{\frac{2\varepsilon_\text{r}\varepsilon_0(N_\text{A}+N_\text{D})(V_\text{D}-V)}{qN_\text{A}N_\text{D}}}
    \end{Equation}
    这就求得了势垒宽度。
\end{Proof}

在进一步讨论势垒宽度之前,我们先来看一个重要的事实,刚刚推导中有
\begin{Equation}
    N_\text{D}x_\text{n}=-N_\text{A}x_\text{p}
\end{Equation}
这是电中性的要求,而这告诉我们
\begin{itemize}
    \item 对于P$^{+}$N结,P区掺杂远大于N区,即$N_\text{A}\gg N_\text{D}$,则有$x_\text{p}\ll x_\text{n}$,势垒主要在N区。
    \item 对于N$^{+}$P结,P区掺杂远小于N区,即$N_\text{A}\ll N_\text{D}$,则有$x_\text{p}\gg x_\text{n}$,势垒主要在P区。
\end{itemize}
换言之,重掺杂侧势垒宽度小,轻掺杂侧势垒宽度大,势垒区主要向轻掺杂区生长。

对于P$^{+}$N结,由于$N_\text{A}\gg N_\text{D}$,有$x_\text{p}\ll x_\text{n}$,故$X_\text{D}=+x_\text{n}$
\begin{Equation}
    V_\text{D}=\frac{qN_\text{D}X_\text{D}^2}{2\varepsilon_\text{r}\varepsilon_0}\qquad
    X_\text{D}=+x_\text{n}=\sqrt{\frac{2\varepsilon_\text{r}\varepsilon_0(V_\text{D}-V)}{qN_\text{D}}}
\end{Equation}
对于N$^{+}$P结,由于$N_\text{A}\ll N_\text{D}$,有$x_\text{p}\gg x_\text{n}$,故$X_\text{D}=-x_\text{p}$
\begin{Equation}
    V_\text{D}=\frac{qN_\text{A}X_\text{D}^2}{2\varepsilon_\text{r}\varepsilon_0}\qquad
    X_\text{D}=-x_\text{p}=\sqrt{\frac{2\varepsilon_\text{r}\varepsilon_0(V_\text{D}-V)}{qN_\text{A}}}
\end{Equation}
这表明,在外加电压一定时,单边突变PN结的势垒宽度随轻掺杂侧的浓度增大而减小。

需要指出的是,这里$V_\text{D}$关于$X_\text{D}$的表达式是仅在平衡状态下成立的,对于一个掺杂浓度一定的PN结,其$V_\text{D}$是一个确定的值,而$X_\text{D}$则会随外加偏压而变化(即将$V_\text{D}$替换为$V_\text{D}-V$)。

再让我们回到最一般的$X_\text{D}$的表达式
\begin{Equation}
    X_\text{D}=\sqrt{\frac{2\varepsilon_\text{r}\varepsilon_0(N_\text{A}+N_\text{D})(V_\text{D}-V)}{qN_\text{A}N_\text{D}}}
\end{Equation}
在掺杂一定的时候,改变外加电压$V$
\begin{itemize}
    \item 当加正向电压时,有$V>0$,此时势垒宽度$X_\text{D}$相较平衡时减小。
    \item 当加反向电压时,有$V<0$,此时势垒宽度$X_\text{D}$相较平衡时增大。
\end{itemize}
且这种变化正比于$V_\text{D}-V$的平方根,这就解释了\xref{fig:PN结的载流子浓度与电势分布}中势垒宽度随外加电压$V$的变化了。

\begin{BoxFormula}[突变结的势垒电容]
    在突变结中,势垒电容满足
    \begin{Equation}
        C_\text{T}=A
        \sqrt{\frac{\varepsilon_\text{r}\varepsilon_0qN_\text{A}N_\text{D}}
        {2(N_\text{D}+N_\text{A})(V_\text{D}-V)}}
    \end{Equation}
\end{BoxFormula}

\begin{Proof}
    势垒宽度$X_\text{D}$可以表示为
    \begin{Equation}&[1]
        X_\text{D}=x_\text{n}-x_\text{p}
    \end{Equation}
    势垒区单位面积上的电荷量可以表示为
    \begin{Equation}&[2]
        |Q_0|=qN_\text{D}x_\text{n}=-qN_\text{A}x_\text{p}
    \end{Equation}
    即有
    \begin{Equation}&[3]
        x_\text{n}=\frac{|Q_0|}{qN_\text{D}}\qquad
        x_\text{p}=-\frac{|Q_0|}{qN_\text{A}}
    \end{Equation}
    将\xrefpeq{3}代入\xrefpeq{1}中
    \begin{Equation}
        X_\text{D}=\frac{Q_0(N_\text{A}+N_\text{D})}{qN_\text{A}N_\text{D}}
    \end{Equation}
    即
    \begin{Equation}
        |Q_0|=\frac{N_\text{A}N_\text{D}qX_\text{D}}{N_\text{A}+N_\text{D}}
    \end{Equation}
    就$X_\text{D}$代入\fancyref{fml:突变结的势垒宽度}
    \begin{Equation}
        |Q_0|=\frac{N_\text{A}N_\text{D}q}{N_\text{A}+N_\text{D}}\sqrt{\frac{2\varepsilon_\text{r}\varepsilon_0(N_\text{A}+N_\text{D})(V_\text{D}-V)}{qN_\text{A}N_\text{D}}}
    \end{Equation}
    整理得
    \begin{Equation}
        |Q_0|=\sqrt{\frac{2\varepsilon_\text{r}\varepsilon_0N_\text{A}N_\text{D}q(V_\text{D}-V)}{N_\text{A}+N_\text{D}}}
    \end{Equation}
    依照微分电容的定义,单位面积的势垒电容为
    \begin{Equation}
        C_\text{T0}=\abs{\dv{Q_0}{V}}=\sqrt{\frac{2\varepsilon_\text{r}\varepsilon_0N_\text{A}N_\text{D}q}{N_\text{A}+N_\text{D}}}\qty(\frac{1}{2}\frac{1}{\sqrt{V_\text{D}-V}})
    \end{Equation}
    即
    \begin{Equation}
        C_\text{T0}=\sqrt{\frac{\varepsilon_\text{r}\varepsilon_0N_\text{A}N_\text{D}q}{2(N_\text{A}+N_\text{D})(V_\text{D}-V)}}
    \end{Equation}
    进而,考虑截面积为$A$
    \begin{Equation}*
        C_\text{T}=AC_\text{T0}=A
        \sqrt{\frac{\varepsilon_\text{r}\varepsilon_0qN_\text{A}N_\text{D}}
        {2(N_\text{D}+N_\text{A})(V_\text{D}-V)}}\qedhere
    \end{Equation}
\end{Proof}

根据\fancyref{fml:突变结的势垒宽度}
\begin{Equation}
    X_\text{D}=\sqrt{\frac{2\varepsilon_\text{r}\varepsilon_0(N_\text{A}+N_\text{D})(V_\text{D}-V)}{qN_\text{A}N_\text{D}}}
\end{Equation}
根据\fancyref{fml:突变结的势垒电容}
\begin{Equation}
    C_\text{T}=A
    \sqrt{\frac{\varepsilon_\text{r}\varepsilon_0qN_\text{A}N_\text{D}}
    {2(N_\text{D}+N_\text{A})(V_\text{D}-V)}}
\end{Equation}
我们可以试着将$X_\text{D}$代入$C_\text{T}$
\begin{Equation}
    C_\text{T}=\frac{A\varepsilon_\text{r}\varepsilon_0}{X_\text{D}}
\end{Equation}
这一结果于平行板电容器的公式在形式上一致。因此,可以将势垒电容等效为一个间距为势垒宽度$X_\text{D}$的平行板电容器,但是,PN结势垒电容中的势垒宽度$X_\text{D}$与外加电压有关,并不是一个恒量,因此,PN结势垒电容是随外加电压变化的非线性电容。这是有所不同的。

而对于单边的P$^{+}$N结或N$^{+}$P结,\fancyref{fml:突变结的势垒电容}可以简化为
\begin{Equation}
    C_\text{T}=A\sqrt{\frac{\varepsilon_\text{r}N_\text{D}}{2(V_\text{D}-V)}}\qquad
    C_\text{T}=A\sqrt{\frac{\varepsilon_\text{r}N_\text{A}}{2(V_\text{D}-V)}}
\end{Equation}
这就表明,对于单边PN结
\begin{itemize}
    \item 突变结的势垒电容,正比于结的面积,正比于轻掺杂一侧的杂质浓度的平方根,由此可见,减小结面积以及轻掺杂一侧的杂质浓度是减小势垒电容,提高整流效果的有效途径。
    \item 突变结的势垒电容,反比于与电压$V_\text{D}-V$的平方根,因此,反向偏压越大,势垒电容就越小,反向偏压若随时间变化,则势垒电容也随时间变化,利用该特性可以制作变容器件。正向偏压的情况不予讨论,原因是推导$C_\text{T}$的过程使用了耗尽层近似,即假设外加电压完全降落在耗尽层,然而通过\xref{sec:PN结的非理想因素},我们知道这对于正向偏压是不成立的,因此正向偏压时$C_\text{T}$的公式不适用,通常以$4C_\text{T}(0)$作为正向偏压时的势垒电容的经验值。
\end{itemize}
以上结论在半导体器件的设计和生产中具有重要的意义。

\subsection{缓变结的势垒电容}
在本小节,我们将依次讨论线性缓变结的电场分布与电势分布、势垒宽度,势垒电容。

根据\xref{subsec:扩散法}的内容,对于线性缓变结,器势垒区的空间电荷密度为
\begin{Equation}[缓变结]
    \rho(x)=q(N_\text{D}-N_\text{A})=q\alpha_\text{j}x
\end{Equation}
其中$\alpha_\text{j}$为杂质的浓度梯度,很明显,由于势垒区正负空间电荷总量相等,线性缓变结的势垒区边界$x=\pm X_\text{D}/2$必然是对称分布的。这是线性缓变结与突变结间的一个重要的区别。

\begin{BoxFormula}[缓变结的电场分布]
    在线性缓变结中,电场分布满足
    \begin{Equation}
        \Emf=\frac{q\alpha_\text{j}}{2\varepsilon_\text{r}\varepsilon_0}\qty(x^2-\frac{X_\text{D}^2}{4})
    \end{Equation}
    在$x=0$处,电场取最大值
    \begin{Equation}
        \Emf_\text{m}=-\frac{q\alpha_\text{j}X_\text{D}^2}{8\varepsilon_\text{r}\varepsilon_0}
    \end{Equation}
\end{BoxFormula}

\begin{Proof}
    根据泊松方程,代入\xrefeq{缓变结}
    \begin{Equation}&[1]
        \dv[2]{V}{x}=-\frac{q\alpha_\text{j}x}{\varepsilon_\text{r}\varepsilon_0}
    \end{Equation}
    将\xrefpeq{1}积分一次得
    \begin{Equation}&[2]
        \dv{V}{x}=-\frac{q\alpha_\text{j}x^2}{2\varepsilon_\text{r}\varepsilon_0}+C
    \end{Equation}
    由于$\Emf=-\dv*{V}{x}$
    \begin{Equation}&[3]
        \Emf(x)=\frac{q\alpha_\text{j}x^2}{2\varepsilon_\text{r}\varepsilon_0}-C
    \end{Equation}
    由于势垒区边界的电场为零
    \begin{Equation}&[4]
        \Emf\qty(\pm\frac{X_\text{D}}{2})=0
    \end{Equation}
    解得
    \begin{Equation}&[5]
        C=\frac{q\alpha_\text{j}X_\text{D}^2}{8\varepsilon_\text{r}\varepsilon_0}
    \end{Equation}
    将\xrefpeq{5}代入\xrefpeq{3}
    \begin{Equation}
        \Emf(x)=\frac{q\alpha_\text{j}x^2}{2\varepsilon_\text{r}\varepsilon_0}-\frac{q\alpha_\text{j}X_\text{D}^2}{8\varepsilon_\text{r}\varepsilon_0}
    \end{Equation}
    即
    \begin{Equation}*
        \Emf=\frac{q\alpha_\text{j}}{2\varepsilon_\text{r}\varepsilon_0}\qty(x^2-\frac{X_\text{D}^2}{4})\qedhere
    \end{Equation}
\end{Proof}

\begin{BoxFormula}[缓变结的电势分布]
    在线性缓变结中,电势分布满足
    \begin{Equation}
        V(x)=\frac{qa_\text{j}}{2\varepsilon_\text{r}\varepsilon_0}\qty(-\frac{x^3}{3}+\frac{xX_\text{D}^2}{4})
    \end{Equation}
    其中接触电势差$V_\text{D}$为
    \begin{Equation}
        V_\text{D}=\frac{q\alpha_\text{j}X_\text{D}^3}{12\varepsilon_\text{r}\varepsilon_0}
    \end{Equation}
\end{BoxFormula}

\begin{Proof}
    顺承\fancyref{fml:缓变结的电场分布}的结果,考虑到$\Emf=-\dv{V}{x}$
    \begin{Equation}
        \dv{V}{x}=\frac{q\alpha_\text{j}}{2\varepsilon_\text{r}\varepsilon_0}\qty(-x^2+\frac{X_\text{D}^2}{4})
    \end{Equation}
    积分得
    \begin{Equation}
        V(x)=\frac{q\alpha_\text{j}}{2\varepsilon_\text{r}\varepsilon_0}\qty(-\frac{x^3}{3}+\frac{xX_\text{D}^2}{4})+D
    \end{Equation}
    电势的参考点是可以选取的,这里改设$x=0$为电势参考顶啊
    \begin{Equation}
        V\qty(0)=0
    \end{Equation}
    因此$D=0$
    \begin{Equation}
        V(x)=\frac{q\alpha_\text{j}}{2\varepsilon_\text{r}\varepsilon_0}\qty(-\frac{x^3}{3}+\frac{xX_\text{D}}{4})
    \end{Equation}
    而接触电势差
    \begin{Equation}
        V_\text{D}=V\qty(\frac{X_\text{D}}{2})-V\qty(-\frac{X_\text{D}}{2})
    \end{Equation}
    注意到
    \begin{Equation}
        \qquad\qquad
        V\qty(\frac{X_\text{D}}{2})=\frac{q\alpha_\text{j}}{2\varepsilon_\text{r}\varepsilon_0}\qty(-\frac{X_\text{D}^3}{24}+\frac{X_\text{D}^3}{8})=\frac{q\alpha_\text{j}X_\text{D}^3}{24\varepsilon_\text{r}\varepsilon_0}\qquad
        V\qty(-\frac{X_\text{D}}{2})=
        -V\qty(\frac{X_\text{D}}{2})
        \qquad\qquad
    \end{Equation}
    因此
    \begin{Equation}*
        V_\text{D}=\frac{q\alpha_\text{j}X_\text{D}^3}{12\varepsilon_\text{r}\varepsilon_0}\qedhere
    \end{Equation}
\end{Proof}

由此可见,如\xref{fig:缓变结}所示,线性缓变结的电场和电势分别是二次函数和三次函数。

到这里,“电荷密度--电场--电势”这条线已经非常清楚了,它们依次是二阶、一阶、零阶的导数
\begin{itemize}
    \item 对于突变结,依次是“阶跃函数--一次函数--二次函数”。
    \item 对于线性缓变结,依次是“一次函数--二次函数--三次函数”。
\end{itemize}

\begin{Figure}[缓变结]
    \begin{FigureSub}[缓变结的电荷密度]
        \includegraphics[width=4.5cm]{build/Chapter06E_04.fig.pdf}
    \end{FigureSub}
    \begin{FigureSub}[缓变结的电场]
        \includegraphics[width=4.5cm]{build/Chapter06E_05.fig.pdf}
    \end{FigureSub}
    \begin{FigureSub}[缓变结的电势]
        \includegraphics[width=4.5cm]{build/Chapter06E_06.fig.pdf}
    \end{FigureSub}
\end{Figure}
接下来,我们计算线性缓变结结的势垒宽度,这相较突变结容易很多
\begin{BoxFormula}[缓变结的势垒宽度]
    在线性缓变结中,势垒宽度满足
    \begin{Equation}
        X_\text{D}=\sqrt[3]{\frac{12\varepsilon_\text{r}\varepsilon_{0}(V_\text{D}-V)}{q\alpha_\text{j}}}
    \end{Equation}
\end{BoxFormula}

\begin{Proof}
    根据\fancyref{fml:缓变结的电势分布}
    \begin{Equation}
        V_\text{D}=\frac{q\alpha_\text{j}X_\text{D}^3}{12\varepsilon_\text{r}}
    \end{Equation}
    将$X_\text{D}$反表示
    \begin{Equation}
        X_\text{D}=\sqrt[3]{\frac{12\varepsilon_\text{r}\varepsilon_{0}V_\text{D}}{q\alpha_\text{j}}}
    \end{Equation}
    当有外加电压时,将$V_\text{D}$替换为$V_\text{D}-V$
    \begin{Equation}*
        X_\text{D}=\sqrt[3]{\frac{12\varepsilon_\text{r}\varepsilon_{0}(V_\text{D}-V)}{q\alpha_\text{j}}}\qedhere
    \end{Equation}
\end{Proof}
之所以简单了很多,是因为线性缓变结总是对称的。

\begin{BoxFormula}[缓变结的势垒电容]
    在线性缓变结中,势垒电容满足
    \begin{Equation}
        C_\text{T}=A\sqrt[3]{\frac{q\alpha_\text{j}\varepsilon_\text{r}^2\varepsilon_0^2}{12(V_\text{V}-V)}}
    \end{Equation}
\end{BoxFormula}

\begin{Proof}
    由于线性缓变结的$\rho(x)$不是常量,故$|Q_0|$需要通过积分计算
    \begin{Equation}&[1]
        |Q_0|=\Int[0][X_\text{D}/2]\rho(x)\dx=\Int[0][X_\text{D}/2]q\alpha_\text{j}x\dx=\frac{q\alpha_\text{j}X_\text{D}^2}{8}
    \end{Equation}
    就$X_\text{D}$代入\fancyref{fml:缓变结的势垒宽度}
    \begin{Equation}
        |Q_0|=\frac{q\alpha_\text{j}}{8}\sqrt[3]{\qty[\frac{12\varepsilon_\text{r}\varepsilon_{0}(V_\text{D}-V)}{q\alpha_\text{j}}]^2}
    \end{Equation}
    即
    \begin{Equation}
        |Q_0|=\sqrt[3]{\frac{9q\alpha_\text{j}\varepsilon_\text{r}^2\varepsilon_0^2}{32}}(V_\text{D}-V)^{2/3}
    \end{Equation}
    求导即得
    \begin{Equation}
        C_\text{T0}=\abs{\dv{Q_0}{V}}=\sqrt[3]{\frac{9q\alpha_\text{j}\varepsilon_\text{r}^2\varepsilon_0^2}{32}}\qty(\frac{2}{3}\frac{1}{\sqrt[3]{V_\text{D}-V}})
    \end{Equation}
    合并
    \begin{Equation}
        C_\text{T0}=\sqrt[3]{\frac{q\alpha_\text{j}\varepsilon_\text{r}^2\varepsilon_0^2}{12(V_\text{D}-V)}}
    \end{Equation}
    进而,考虑截面积为$A$
    \begin{Equation}*
        C_\text{T}=AC_\text{T0}=A\sqrt[3]{\frac{q\alpha_\text{j}\varepsilon_\text{r}^2\varepsilon_0^2}{12(V_\text{D}-V)}}\qedhere
    \end{Equation}
\end{Proof}

根据\fancyref{fml:缓变结的势垒宽度}
\begin{Equation}
    X_\text{D}=\sqrt[3]{\frac{12\varepsilon_\text{r}\varepsilon_{0}(V_\text{D}-V)}{q\alpha_\text{j}}}
\end{Equation}
根据\fancyref{fml:缓变结的势垒电容}
\begin{Equation}
    C_\text{T}=A\sqrt[3]{\frac{q\alpha_\text{j}\varepsilon_\text{r}^2\varepsilon_0^2}{12(V_\text{V}-V)}}
\end{Equation}
我们可以试着将$X_\text{D}$代入$C_\text{T}$
\begin{Equation}
    C_\text{T}=\frac{A\varepsilon_\text{r}\varepsilon_0}{X_\text{D}}
\end{Equation}
这表明,无论是突变结还是线性缓变结,形式上都可以表示为平行板电容器。

\subsection{扩散电容}
在\xref{subsec:PN结电容的分类}中我们已经指出,PN结加正向偏压时,由于少子的注入,在扩散区内,都有一定数量的少子和等量的多子的积累,且它们的浓度随正向偏压变化,从而形成了扩散电容。
\begin{BoxFormula}[扩散电容]
    在PN结中,扩散电容满足
    \begin{Equation}
        C_\text{D}=\frac{Aq^2(n_\text{P0}L_\text{n}+p_\text{N0}L_\text{p})}{\kB T}\exp(\frac{qV}{\kB T})
    \end{Equation}
\end{BoxFormula}

\begin{Proof}
    根据\fancyref{fml:PN结外加偏压时的少子浓度},在N区
    \begin{Equation}&[1]
        \delt{p_\text{N}}(x)=p_\text{N0}\qty[\exp(\frac{qV}{\kB T})-1]\exp(\frac{x_\text{n}-x}{L_\text{p}})
    \end{Equation}
    根据\fancyref{fml:PN结外加偏压时的少子浓度},在P区
    \begin{Equation}&[2]
        \delt{n_\text{P}}(x)=n_\text{P0}\qty[\exp(\frac{qV}{\kB T})-1]\exp(\frac{x-x_\text{p}}{L_\text{n}})
    \end{Equation}
    将\xrefpeq{1}和\xrefpeq{2}分别在各自的扩散区内积分,即得单位面积的载流子总电荷量
    \begin{Gather}[12pt]
        |Q_\text{p0}|=\Int[x_\text{n}][\infty]q\delt{p_\text{N}(x)}\dx=qL_\text{p}p_\text{N0}\qty[\exp(\frac{qV}{\kB T})-1]\\
        |Q_\text{n0}|=\Int[-\infty][x_\text{p}]q\delt{n_\text{P}(x)}\dx=qL_\text{n}n_\text{P0}\qty[\exp(\frac{qV}{\kB T})-1]
    \end{Gather}
    单位面积的P区扩散电容和N区扩散电容即分别是
    \begin{Gather}[12pt]
        C_\text{Dp0}=\abs{\dv{Q_\text{p0}}{V}}=
        \qty(\frac{q^2p_\text{N0}L_\text{p}}{\kB T})\exp(\frac{qV}{\kB T})\\
        C_\text{Dn0}=\abs{\dv{Q_\text{n0}}{V}}=
        \qty(\frac{q^2n_\text{P0}L_\text{n}}{\kB T})\exp(\frac{qV}{\kB T})
    \end{Gather}
    单位面积的总扩散电容,是上述两者和
    \begin{Equation}
        C_\text{D0}=C_\text{Dp0}+C_\text{Dn0}=
        \frac{q^2(p_\text{N0}L_\text{p}+n_\text{P0}L_\text{n})}{\kB T}\exp(\frac{qV}{\kB T})
    \end{Equation}
    考虑截面积$A$
    \begin{Equation}*
        C_\text{D}=AC_\text{D0}=\frac{Aq^2(n_\text{P0}L_\text{n}+p_\text{N0}L_\text{p})}{\kB T}\exp(\frac{qV}{\kB T})\qedhere
    \end{Equation}
\end{Proof}

\fancyref{fml:扩散电容}指出,扩散电容$C_\text{D}$随正向偏压以指数关系增加,而前面我们提到,势垒电容$C_\text{T}$在正向偏压上通常视作$4C_\text{T}(0)$的常值,因此,在较大的正向偏压下,扩散电容便起主要作用。除此之外,扩散电容的推导基于非平衡载流子的稳态分布,这仅适用于低频情况,有进一步的分析指出,扩散电容会随着频率的增加而减小,即,高频下扩散电容会减小。
\section{PN结击穿}
实验发现,当对PN结施加的反向偏压增大到某一数值$V_\text{BR}$是,反向电流将突然迅速增大,该现象称为PN结的\uwave{击穿}(Breakdown),而$V_\text{BR}$则称为\uwave{击穿电压}(Breakdown Voltage)。

在PN结击穿中,主要可以分为三种类型:雪崩击穿、隧道击穿、热电击穿。

\subsection{雪崩击穿}
\uwave{雪崩击穿}(Avalanche Breakdown)是指,当反向偏压很大时,势垒区中的电场很强,势垒区内的电子和空穴由于受到强电场的漂移作用,具有很大的动能,它们与势垒区内的晶格原子发生碰撞时,将价键上的电子碰撞出来,产生电子--空穴对,同时,产生的电子和空穴会继续在强电场的作用下加速,从而产生第二代、第三代、第四代的载流子。类似于核裂变反应,载流子将迅速倍增,这称为载流子的\uwave{倍增效应}。在倍增效应的作用下,载流子的数目像雪崩一样增加的越来越快,载流子数目的迅速增加也导致反向电流迅速增大,最终导致PN结的击穿。

\subsection{齐纳击穿}
\uwave{齐纳击穿}(Zener Breakdown)也称为\uwave{隧道击穿},“齐纳”得名于该现象的发现者,“隧道”则解释了原理,齐纳击穿基于的正是量子力学中的隧道效应。在\xref{fig:反向偏压较大}(这里简便起见将能带弯曲绘制为折线)或先前我们用于示意反向偏压的\xref{fig:反向偏压下的能带结构}中,我们看到,当PN结上的反向偏压较大时,N区的导带顶甚至可以低于P区的价带底(观察\xref{fig:反向偏压较小}至\xref{fig:反向偏压较大}的变化)。

\begin{Figure}[齐纳击穿]
    \begin{FigureSub}[反向偏压较小]
        \includegraphics[scale=0.8]{build/Chapter06F_01.fig.pdf}
    \end{FigureSub}\\ \vspace{0.5cm}
    \begin{FigureSub}[反向偏压较大]
        \includegraphics[scale=0.8]{build/Chapter06F_02.fig.pdf}
    \end{FigureSub}
\end{Figure}

这就产生了一种很有趣的情况,我们知道,所谓载流子的产生,无非就是一个电子从价带跃迁至导带,这通常需要$E_\text{g}$的能量越过禁带,然而,在\xref{fig:反向偏压较大}的能带中,由于能带的弯曲,我们注意到$A,B$两个点虽然分别位于价带顶和导带底,但两者的能量却在同一条线上。尽管如\xref{fig:PN结的三角形势垒}所示,从$A$至$B$仍然要跨过高$E_\text{g}$宽$\delt{x}$的三角形势垒,这在经典力学中是不可能发生的,但是,量子力学的原理告诉我们,只要起始点的能量高于或等于终末点的能量,无论路径上是否有势垒,电子都有概率通过,这就是\uwave{隧穿效应}(Tunneling Effect),电子就像打了隧道一样,从底部通过了高于其能量的势垒。所以只要反向偏压能使得N区导带顶低于P区的价带顶,隧穿就有可能会发生,无论$E_\text{g}$和$\delt{x}$的取值如何。但是只有当$E_\text{g}$和$\delt{x}$比较小即势垒较低且较窄时,隧穿才能有较大概率发生,产生较大的隧道电流,形成隧道击穿。

\begin{Figure}[PN结的三角形势垒]
    \includegraphics{build/Chapter06F_03.fig.pdf}
\end{Figure}

量子力学证明,上述隧道概率是\setpeq{隧道效应}
\begin{Equation}&[1]
    P=\exp{-\frac{2}{\hbar}(2\mne)^{1/2}\Int[x_1][x_2][E(x)-E]^{1/2}\dx}
\end{Equation}

其中$E(x)$表示点$x$处的势垒高度,而$E$为电子能量,$x_1,x_2$分别为势垒区的边界。

这里不妨令$E=0$,且记$x_1=0$而$x_2=\delt{x}$即\xref{fig:PN结的三角形势垒}中$A,B$的位置。而$E(x)$则设为
\begin{Equation}&[2]
    E(x)=q\Emf x
\end{Equation}
这是近似认为在$x$处有一$\Emf$的恒定电场(实际上由\xref{sec:PN结电容},我们知道势垒区电场并非恒定)。

将\xrefpeq{2}和相关量代入\xrefpeq{1}
\begin{Equation}&[3]
    P=\exp{-\frac{2}{\hbar}(2\mne)^{1/2}\Int[0][\delt{x}](q\Emf x)^{1/2}\dx}
\end{Equation}
计算积分得
\begin{Equation}&[4]
    P=\exp{-\frac{4}{3\hbar}(2\mne)^{1/2}(q\Emf)^{1/2}(\delt{x})^{3/2}}
\end{Equation}
由于$\delt{x}=E_\text{g}/q\Emf$
\begin{Equation}&[5]
    P=\exp[-\frac{4}{3\hbar}(2\mne)^{1/2}(E_\text{g})^{3/2}\frac{1}{q\Emf}]
\end{Equation}
由此可见,势垒中的电场$\Emf$越大,隧穿的概率就越大。\xrefpeq{5}亦可以改写为
\begin{Equation}
    P=\exp[-\frac{4}{3\hbar}(2\mne)^{1/2}(E_\text{g})^{1/2}\delt{x}]
\end{Equation}
这表明,隧穿概率与隧穿长度$\delt{x}$负相关。而由\xref{fig:反向偏压较大}易知(这里$V$对于反向电压取负值)
\begin{Equation}
    \frac{E_\text{g}}{\delt{x}}=\frac{q(V_\text{D}-V)}{X_\text{D}}
\end{Equation}
即
\begin{Equation}
    \delt{x}=\frac{E_\text{g}}{q}\frac{X_\text{D}}{V_\text{D}-V}
\end{Equation}
而代入\fancyref{fml:突变结的势垒宽度}
\begin{Equation}
    \delt{x}=\frac{E_\text{g}}{q}\frac{1}{V_\text{D}-V}\qty[\frac{2\varepsilon_\text{r}\varepsilon_0(N_\text{A}+N_\text{D})(V_\text{D}-V)}{qN_\text{A}N_\text{D}}]^{1/2}
\end{Equation}
化简得到
\begin{Equation}
    \delt{x}=\frac{E_\text{g}}{q}\qty[\frac{2\varepsilon_\text{r}\varepsilon_0(N_\text{A}+N_\text{D})}{qN_\text{A}N_\text{D}(V_\text{D}-V)}]^{1/2}
\end{Equation}
若引入$N=N_\text{A}N_\text{D}/(N_\text{A}+N_\text{D})$和$V_\text{A}=V_\text{D}-V$
\begin{Equation}
    \delt{x}=\frac{E_\text{g}}{q}\qty(\frac{2\varepsilon_\text{r}\varepsilon_0}{qNV_\text{A}})^{1/2}
\end{Equation}
因此,$(NV_\text{A})$越大,隧穿长度$\delt{x}$越小,隧穿概率$P$就越大,隧道击穿就越容易发生,这表明,隧道击穿需要一定的$NV_\text{A}$值,它既可以是$N$小$V_\text{A}$大,也可以是$N$大$V_\text{A}$小。轻掺杂时需要增大反向偏压才能发生隧道击穿,但这同时使得雪崩击穿也很容易发生,因此,轻掺时雪崩击穿是主要的。增大掺杂可以减小隧道击穿的电压,因而,重掺时隧道击穿是主要的。

\begin{itemize}
    \item 当$V_\text{BR}>6E_\text{g}/q$时,通常为雪崩击穿。
    \item 当$V_\text{BR}<4E_\text{g}/q$时,通常为隧道击穿。
\end{itemize}

\subsection{热电击穿}
热电击穿实际是前两种击穿的一种伴随现象,当反向电流急剧增大时,PN结会产生大量的热,若没有良好的散热条件,PN结的温度会急剧升高。根据\fancyref{eqt:肖克利方程}
\begin{Equation}
    J_\text{s}=\frac{qD_\text{n}n_\text{P0}}{L_\text{n}}+\frac{qD_\text{p}p_\text{N0}}{L_\text{p}}    
\end{Equation}
运用$n_\text{P0}=n_\text{i}^2/p_\text{P0}=n_\text{i}^2/N_\text{A}$和$n_\text{N0}=n_\text{i}^2/n_\text{N0}=n_\text{i}^2/N_\text{D}$
\begin{Equation}
    J_\text{s}=\qty(\frac{qD_\text{n}}{L_\text{n}N_\text{A}}+\frac{qD_\text{p}}{L_\text{p}N_\text{D}})n_\text{i}^2
\end{Equation}
而根据\fancyref{fml:载流子的浓度乘积},有$n_\text{i}^2\propto T^3\exp[-E_\text{g}/\kB T]$,因此随着PN结温度的增加,PN结的反向饱和电流$J_\text{s}$将急剧增大,这反过来会进一步导致温度的增加,最终导致击穿。这就是所谓的\uwave{热电击穿}(Thermal Breakdown)。实际上,雪崩击穿和隧道击穿本身都是一个安全可逆的电学过程,使击穿具有破坏性的是其附带的热电击穿,其在加剧击穿的同时产生的大量热会最终将PN结熔毁,造成不可逆的损坏。故通常我们不会让PN结在击穿状态下工作。但有特制的隧道PN结,其可以安全地发生隧道击穿,且不会伴随热电击穿。
