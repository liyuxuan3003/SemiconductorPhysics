\chapter{PN结}
在前几章中,我们分别研究了N型半导体和P型半导体中载流子的浓度和运动情况,认识了掺杂半导体在热平衡和非平衡状态下的一些物理性质。而如果将一块P型半导体和一块N型半导体结合在一起,在两者的交界面上,就形成了所谓的\uwave{PN结}(PN Junction),那么这种具有PN结的半导体将具有什么性质呢?这就是本章我们将要讨论的主要问题。由于PN结是很多半导体器件的基本结构,因此,了解和掌握PN结的性质就具有很重要的实际意义。

\begin{Figure}[PN结的示意图]
    \includegraphics{build/Chapter06A_01.fig.pdf}
\end{Figure}

在本章,主要将讨论PN结的几个重要特性,包含:电流电压特性、电容效应、击穿特性。

\section{PN结的形成}

首先的问题是,PN结是如何形成的?显然,将P型半导体和N型半导体拿些焊锡焊在一起并不会得到任何的PN结。我们需要使用一些工艺。工艺有很多种,但核心思路是一致的,即在N型半导体中通过一定方式将其局部转变为P型半导体,或反之。工艺主要有以下四种
\begin{enumerate}
    \item 合金法
    \item 扩散法
    \item 生长法
    \item 注入法
\end{enumerate}
这里我们重点介绍合金法和扩散法,如\xref{fig:PN结的两种工艺},它们分别代表了两类不同类型的PN结。
\begin{Figure}[PN结的两种工艺]
    \begin{FigureSub}[合金法]
        \includegraphics[scale=0.9]{build/Chapter06A_02.fig.pdf}
    \end{FigureSub}
    \hspace{2cm}
    \begin{FigureSub}[扩散法]
        \includegraphics[scale=0.9]{build/Chapter06A_03.fig.pdf}
    \end{FigureSub}
\end{Figure}

\subsection{合金法}
\xref{fig:合金法}表示用合金法制造PN结的过程,将一小粒铝放在一块N型单晶硅上,随后将铝粒加热到一定温度,铝粒和其接触的硅薄层形成硅铝熔融体,然后降低温度,硅铝熔融体开始凝固,这时N型硅片上就形成了一含有高浓度铝的P型硅片,构成PN结,称为铝硅合金结。

\uwave{合金结}(Alloy Junction)属于\uwave{突变结},其浓度分布如\xref{fig:突变结的掺杂浓度分布}
\begin{Figure}[突变结的掺杂浓度分布]
    \includegraphics[scale=1]{build/Chapter06A_04.fig.pdf}
\end{Figure}
设交界面为$x_j$,则有
\begin{Gather}
    N_\text{A}(x)=N_\text{A}\quad (x<x_\text{j})\qquad
    N_\text{D}(x)=N_\text{D}\quad (x>x_\text{j})
\end{Gather}
换言之,在突变结中,P型区仅有受主杂质$N_\text{A}$,N型区仅有施主杂质$N_\text{D}$,两者均是均匀分布的,交界面处,两者的浓度均会发生突变,$N_\text{A}(x)$由$N_\text{A}$变为$0$,$N_\text{D}(x)$由$0$变为$N_\text{D}$。

在实际的突变结中,两边的杂质浓度差异很大,例如这里在N型基底上通过铝形成了高浓度的P型区,则P型区的掺杂将远高于N型区,如$N_\text{A}=10^{16}\si{cm^{-3}}$,而$N_\text{D}=10^{19}\si{cm^{-3}}$。简而言之,\empx{通过合金法在局部形成的掺杂区比基底的掺杂浓度要高的多}。通常,这种结称为\uwave{单边突变结},这里P型区浓度较高,故称为P$^{+}$N结,反之若N型区浓度较高,则称为N$^{+}$P结。

\subsection{扩散法}
\xref{fig:扩散法}表示用扩散法制造PN结的过程,在N型单晶硅片上,第一步通过氧化作出一层均匀的\xce{SiO2}氧化物薄层,第二步通过光刻工艺将\xce{SiO2}刻蚀出特定的形状,使得待掺杂部分暴露出来,第三步通过扩散工艺掺入P型杂质,从而在(光刻暴露出的)特定位置形成P型区。

\uwave{扩散结}(Diffused Junction)属于\uwave{缓变结},其浓度分布如\xref{fig:扩散结的掺杂浓度分布}
\begin{Figure}[扩散结的掺杂浓度分布]
    \includegraphics[scale=1]{build/Chapter06A_05.fig.pdf}
\end{Figure}
扩散结中,基底的浓度可以视为不变的
\begin{Equation}
    N_\text{D}(x)=N_\text{D}
\end{Equation}
扩散结中,掺入的杂质(这里是受主杂质)的浓度则是随掺入深度的增加而逐渐减小的,并非是突变的,扩散是一个过程。通常,我们只关心PN结交界面$x=x_\text{j}$附近的性质,因此,无论$N_\text{A}(x)$实际的函数分布如何,我们总是可以用$x=x_\text{j}$附近的切线近似表示$N_\text{A}(x)$
\begin{Equation}
    N_\text{A}(x)=N_\text{D}+\alpha_\text{j}(x-x_\text{j})
\end{Equation}
这里$\alpha_\text{j}$是$x=x_\text{j}$处的切线斜率,称为\uwave{杂质浓度梯度}。若$\alpha_\text{j}$很大,亦可以近似为突变结。

总而言之,PN结主要包含两类情形,突变结和线性缓变结
\begin{enumerate}
    \item 合金结是突变结。
    \item 高表面浓度的浅扩散结(杂质浓度梯度大)是突变结。
    \item 低表面浓度的深扩散结(杂质浓度梯度小)是线性缓变结。
\end{enumerate}
这两种结在性质上有一定的差异,我们将会在本章稍后更为深入的探讨。\goodbreak
\section{PN结的结构与能带图}

\subsection{PN结的结构}
试想,在PN结中,N型部分电子很多而空穴很少,P型部分空穴很多而电子很少,这就是说,载流子在PN结中存在浓度梯度,而\xref{sec:载流子的扩散运动}指出,载流子会在浓度梯度下作扩散运动
\begin{itemize}
    \item P型区的多子是空穴,空穴将由P区向N区扩散。
    \item N型区的多子是电子,电子将由N区向P区扩散。
\end{itemize}
我们知道,当P型半导体和N型半导体各自独立存在时,两者都是电中性的,因为杂质电离产生了多少空穴和电子,就相应的有多少电荷量相反的电离受主杂质和电离施主杂质。然而在PN结中,空穴扩散流和电子扩散流从P区和N区带走了相当数量的空穴和电子,电离杂质却无法跟随其运动,将会留在原位,这就使得在PN结交界面附近出现了一些带电区
\begin{itemize}
    \item P型区一侧出现了由电离受主构成的\uwave{负电荷区}(与P区多子空穴带正电相反)。
    \item N型区一侧出现了由电离施主构成的\uwave{正电荷区}(与N区多子电子带负电相反)。
\end{itemize}
我们通常将PN结附近的电离施主和电离受主所带的电荷称为\uwave{空间电荷}(Space Charge),而它们所存在的区域则称为\uwave{空间电荷区}(Space Charge Region)。正如我们在\xref{sec:爱因斯坦关系式}中熟悉的那样,这在PN结中,形成了一个由N区指向P区的内建电场,而载流子在内建电场的作用下将作漂移运动,显然,电子和空穴的漂移运动方向与它们的扩散运动是相反的,换言之,空间电荷区形成的内建电场将会反过来限制其自身进一步扩大。在一段时间后,电子和空穴的扩散电流和漂移电流会达成平衡,大小相等,方向相反,相互抵消。此时,流过PN结的净电流为零,空间电荷区保持一定宽度不再继续扩展,空间电荷区中存在一定的内建电场。

\begin{Figure}[PN结的空间电荷区]
    \includegraphics[scale=0.9]{build/Chapter06A_06.fig.pdf}
\end{Figure}

我们将这种热平衡状态下的PN结,称为\uwave{平衡PN结}(PN Junction in Equilibrium)。


\subsection{PN结的能带图}
在理解PN结的能带图前,我们不妨先来回顾一些,对于两块独立的P型半导体和N型半导体,它们的能带图是什么样的?如\xref{fig:P型和N型半导体的能带图},两者应具有相同的价带底$E_\text{v}$和导带底$E_\text{c}$以及相同的禁带中线$E_\text{i}$\ \footnote{确切的说$E_\text{i}$是本征状态下的费米能级,根据\xref{chap:半导体中载流子的统计分布},尽管$E_\text{i}$随着温度升高会偏离中线,但偏离的不多,可以近似相等。}。然而,两者的费米能级$E_\text{Fp},E_\text{Fn}$是不同的,我们知道,费米能级与载流子的分布是密切相关的,P区空穴较多故$E_\text{Fp}$靠近价带,N区电子较多故$E_\text{Fn}$靠近导带。

\vspace{0.25cm}
\begin{Figure}[平衡PN结的能带图]
    \begin{FigureSub}[P型和N型半导体的能带图]
        \includegraphics[scale=0.75]{build/Chapter06A_07.fig.pdf}
    \end{FigureSub}
    \\ \vspace{1cm}
    \begin{FigureSub}[PN结的能带图]
        \includegraphics[scale=0.75]{build/Chapter06A_08.fig.pdf}
    \end{FigureSub}
\end{Figure}
现在的问题是,当P型半导体和N型半导体形成PN结后,在交界区域,这些能量将会以何种方式变化?很自然的一种想法是,原先相同的$E_\text{c},E_\text{v},E_\text{i}$仍然相同,以直线相连,原先不同的$E_\text{F}$在空间电荷区以某种平滑的方式由$E_\text{Fp}$抬升至$E_\text{Fc}$,但很遗憾的是,这并不正确。正如\xref{fig:PN结的能带图}所示的那样,恰相反!费米能级反而在PN结中变得相等!费米能级$E_\text{Fp}$和$E_\text{Fn}$在各自区域中相对$E_\text{cp},E_\text{vp}$和$E_\text{cn},E_\text{vp}$的位置亦没有变化,这就是说,能带发生了相对移动!\goodbreak

我们可能会对这个结果感到不解,毕竟,为何要假设能带相对移动呢?但其实细想一下,能带移动才是更为合理的假设。因为PN结中存在由N区指向P区的内建电场,换言之,PN结中的电势不是均等的,P区电势较低,N区电势较高。尽管此刻我们尚不知道电势$V(x)$具体的数学形式\footnote{稍后有一张PN结电势$V(x)$的图像(其实就是把$E_\text{c},E_\text{v}$的趋势倒过来),再之后我们会推导$V(x)$的表达式。},但我们可以预见的是,若以P区电势作为电势$V(x)$的零参考点
\begin{itemize}
    \item 电势$V(x)$在负端,即远离空间电荷区的P区为$0$。
    \item 电势$V(x)$在正端,即远离空间电荷区的N区为某一定值,记作$V_\text{D}$。
    \item 电势$V(x)$在空间电荷区中,逐渐由$0$增长至$V_\text{D}$(姑且不关心这个函数的具体形式)。
\end{itemize}

而能带反映的是电子的能量,这当然要包含电子的电势能$-qV(x)$,而$-qV(x)$从负到正是逐渐减小的,这也就是为何在PN结中,N型区能带相对P型区能带会向下偏移。所以说,能带移动是合理的。但这仍然没有解释,能带移动后为何恰好对齐至费米能级一致$E_\text{Fp}=E_\text{Fn}$的位置了呢?实际上,这是平衡PN结中没有净电流的直接结果,下面我们来证明这一点。\setpeq{电流密度与费米能级的关系}

我们先来计算电子电流,根据\fancyref{fml:载流子的漂移扩散}
\begin{Equation}&[1]
    J_\text{n}=qn\mu_\text{n}\Emf+qD_\text{n}\dv{n}{x}
\end{Equation}
根据\fancyref{law:爱因斯坦关系式},将$D_\text{n}$代换
\begin{Equation}&[2]
    J_\text{n}=nq\mu_\text{n}\Emf+\kB T\mu_\text{n}\dv{n}{x}
\end{Equation}
考虑到
\begin{Equation}&[3]
    \dv{\ln n}{x}=\frac{1}{n}\dv{n}{x}
\end{Equation}
将\xrefpeq{3}代入\xrefpeq{2}
\begin{Equation}&[4]
    J_\text{n}=nq\mu_\text{n}\Emf+n\kB T\mu_\text{n}\dv{\ln n}{x}
\end{Equation}
试着提取一些公共项
\begin{Equation}&[5]
    J_\text{n}=nq\mu_\text{n}\qty[\Emf+\frac{\kB T}{q}\dv{\ln n}{x}]
\end{Equation}
根据\fancyref{fml:非平衡态下的电子浓度}
\begin{Equation}&[6]
    n=n_\text{i}\exp(\frac{E_\text{F}-E_\text{i}}{\kB T})
\end{Equation}
在\xrefpeq{6}两端取对数
\begin{Equation}&[7]
    \ln n=\ln n_i+\frac{E_\text{F}-E_\text{i}}{\kB T}
\end{Equation}
进而是
\begin{Equation}&[8]
    \dv{\ln n}{x}=\frac{1}{\kB T}\qty(\dv{E_\text{F}}{x}-\dv{E_\text{i}}{x})
\end{Equation}
将\xrefpeq{8}代入\xrefpeq{5},得到
\begin{Equation}&[9]
    J_\text{n}=nq\mu_\text{n}\qty[\Emf+\frac{1}{q}\qty(\dv{E_\text{F}}{x}-\dv{E_\text{i}}{x})]
\end{Equation}
而根据前面的讨论,$E_\text{i}$的变化与$-qV(x)$是一致的,故
\begin{Equation}&[10]
    \dv{E_\text{i}}{x}=-q\dv{V(x)}{x}=-q\Emf
\end{Equation}
将\xrefpeq{10}代入\xrefpeq{9}中
\begin{Equation}&[11]
    J_\text{n}=nq\mu_\text{n}\dv{E_\text{F}}{x}
\end{Equation}
类似的,我们亦可以证明
\begin{Equation}&[12]
    J_\text{p}=p\mu_\text{p}\dv{E_\text{F}}{x}
\end{Equation}
而在平衡PN结中,电子和空穴的净电流$J_\text{n},J_\text{p}$均为零
\begin{Equation}
    J_\text{n}=J_\text{p}=0
\end{Equation}
因而
\begin{Equation}
    \dv{E_\text{F}}{x}=0\qquad E_\text{F}=C
\end{Equation}
这就论证了在平衡PN结中,费米能级$E_\text{F}$必须是一个常数。

这里\xrefpeq{11}和\xrefpeq{12}亦有理论价值,其将电流密度与费米能级联系在了一起。
\begin{BoxFormula}[电流密度与费米能级]
    电子的电流密度,正比于电子浓度和费米能级的变化率的积
    \begin{Equation}
        J_\text{n}=nq\mu_\text{n}\dv{E_\text{F}}{x}
    \end{Equation}
    空穴的电流密度,正比于空穴浓度和费米能级的变化率的积
    \begin{Equation}
        J_\text{p}=pq\mu_\text{P}\dv{E_\text{F}}{x}
    \end{Equation}
\end{BoxFormula}

\subsection{PN结的接触电势差}
在本小节,我们解决一个问题,即先前的$V_D$到底是多少?我们将平衡PN结空间电荷区两端间的电势差$V_\text{D}$称为PN结的\uwave{内建电势差}或\uwave{接触电势差}(Contact Potential Difference)。

在\xref{fig:PN结的能带图}中,我们看到,PN结的接触电势差$V_\text{D}$使PN结在空间电荷区中的能带发生弯曲,而因能带弯曲,电子从势能低的N区向势能高的P区运动时,必须克服这一势能“壁垒”,故空间电荷区亦被形象的称为\uwave{势垒区}(Barrier Region),势垒的高$qV_D$则称为\uwave{势垒高度}(Barrier Height)。很明显,势垒高度$qV_\text{D}$正好补偿了N区和P区原有的费米能级之差,即有
\begin{BoxFormula}[势垒高度]
    势垒高度补偿了N区和P区的费米能级之差,即
    \begin{Equation}
        qV_\text{D}=E_\text{Fn}-E_\text{Fp}
    \end{Equation}
\end{BoxFormula}
那么,接触电势差$V_\text{D}$到底如何求呢?关键在于利用载流子浓度。\setpeq{接触电势差}

在PN结中,载流子浓度有四个,以下分别表示:N区空穴、N区电子、P区空穴、P区电子
\begin{Equation}&[1]
    p_\text{N0}\qquad n_\text{N0}\qquad
    p_\text{P0}\qquad n_\text{P0}
\end{Equation}
在这里我们要用的是N区和P区中的电子浓度$n_\text{N0},n_\text{P0}$,根据\fancyref{fml:导带电子浓度}
\begin{Equation}&[2]
    n_\text{N0}=n_\text{i}\exp(\frac{E_\text{Fn}-E_\text{i}}{\kB T})\qquad
    n_\text{P0}=n_\text{i}\exp(\frac{E_\text{Fp}-E_\text{i}}{\kB T})
\end{Equation}
两式相除,取对数
\begin{Equation}&[3]
    \ln\frac{n_\text{N0}}{n_\text{P0}}=\frac{1}{\kB T}\qty(E_\text{Fn}-E_\text{Fp})=\frac{qV_\text{D}}{\kB T}
\end{Equation}
关注到$n_\text{N0}$为N型区多子浓度,而$n_\text{P0}$为P型区少子浓度
\begin{Equation}&[4]
    n_\text{N0}=N_\text{D}\qquad
    n_\text{P0}=\frac{n_\text{i}^2}{N_\text{A}}
\end{Equation}
将\xrefpeq{4}代入\xrefpeq{3}左端
\begin{Equation}&[5]
    \ln\frac{n_\text{N0}}{n_\text{P0}}=\ln\frac{N_\text{D}N_\text{A}}{n_\text{i}^2}
\end{Equation}
联立\xrefpeq{3}和\xrefpeq{5}
\begin{BoxFormula}[接触电势差]
    PN结的接触电势差$V_\text{D}$满足
    \begin{Equation}
        V_\text{D}=\frac{\kB T}{q}\ln\frac{N_\text{D}N_\text{A}}{n_\text{i}^2}
    \end{Equation}
\end{BoxFormula}

\xref{fml:接触电势差}表明,PN结的接触电势差由PN结自身特性确定,掺杂越多,温度越高,禁带宽度越大($n_\text{i}$越大),接触电势差就越大。硅的禁带比锗宽,因此,硅PN结的$V_\text{D}$也比锗PN结要更大些,若$N_\text{A}=10^{17}\si{cm}^{-3},N_\text{D}=10^{15}\si{cm^{-3}}$,室温下,硅$V_\text{D}=0.70\si{V}$,锗$V_\text{D}=0.32\si{V}$。

\subsection{PN结的载流子分布}
在本小节,我们要研究在PN结中,形成空间电荷区后,载流子的浓度将会如何分布?
\begin{BoxFormula}[PN结的平衡载流子浓度]
    PN结在平衡状态下,电子的平衡载流子浓度为
    \begin{Equation}&[A]
        n(x)=n_\text{N0}\exp[\frac{qV(x)-qV_\text{D}}{\kB T}]
    \end{Equation}
    PN结在平衡状态下,空穴的平衡载流子浓度为
    \begin{Equation}&[B]
        p(x)=p_\text{P0}\exp[\frac{-qV(x)}{\kB T}]
    \end{Equation}
    这里$n_\text{N0}$和$p_\text{P0}$分别是P区和N区的多子浓度。
\end{BoxFormula}
\begin{Proof}
    根据\fancyref{fml:导带电子浓度},这里$E_\text{cn}$代表N区的导带底
    \begin{Equation}
        \qquad\qquad
        n(x)=N_\text{c}\exp[\frac{E_\text{F}-E_\text{c}(x)}{\kB T}]=N_\text{c}\exp[\frac{E_\text{F}-E_\text{cn}}{\kB T}]\exp[\frac{E_\text{cn}-E_\text{c}(x)}{\kB T}]
        \qquad\qquad
    \end{Equation}
    引入$n_\text{N0}$,如\xref{fig:PN结的能带图}所示,取$E_\text{cp}$为零参考点,则$E_\text{c}(x)=-qV(x)$,而$E_\text{cn}=-qV_\text{D}$
    \begin{Equation}
        n(x)=n_\text{N0}\exp[\frac{E_\text{cn}-E_\text{c}(x)}{\kB T}]=n_\text{N0}\exp[\frac{qV(x)-qV_\text{D}}{\kB T}]
    \end{Equation}
    根据\fancyref{fml:导带电子浓度},这里$E_\text{vp}$代表P区价带底
    \begin{Equation}
        \qquad\qquad
        p(x)=N_\text{v}\exp[\frac{E_\text{v}(x)-E_\text{F}}{\kB T}]=N_\text{v}\exp[\frac{E_\text{vp}-E_\text{F}}{\kB T}]\exp[\frac{E_\text{v}(x)-E_\text{vp}(x)}{\kB T}]
        \qquad\qquad
    \end{Equation}
    引入$p_\text{P0}$,如\xref{fig:PN结的能带图}所示,取$E_\text{vp}$为零参考点,则$E_\text{v}(x)=-qV(x)$,而$E_\text{vp}=0$
    \begin{Equation}
        p(x)=p_\text{P0}\exp[\frac{E_\text{v}(x)-E_\text{vp}}{\kB T}]=p_\text{P0}\exp[\frac{-qV(x)}{\kB T}]
    \end{Equation}
    由此,我们就得到了\xrefpeq{A}和\xrefpeq{B}。
\end{Proof}

若以$x_\text{p}$和$x_\text{n}$表示空间电荷区在P区和N区的边界,则
\begin{Equation}
    V(x_\text{p})=0\qquad V(x_\text{n})=V_\text{D}
\end{Equation}
而显然少子浓度分别满足$n_\text{P0}=n(x_\text{p})$和$p_\text{N0}=p(x_\text{n})$,将上式代入,即得以下结论。
\begin{BoxFormula}[PN结多子浓度与少子浓度的关系]*
    PN结中,电子在P区的少子浓度$n_\text{P0}$和其在N区的多子浓度$n_\text{N0}$的关系是
    \begin{Equation}
        n_\text{P0}=n(x_\text{p})=n_\text{N0}\exp[\frac{-qV_\text{D}}{\kB T}]
    \end{Equation}
    PN结中,空穴在N区的少子浓度$p_\text{N0}$和其在P区的多子浓度$p_\text{P0}$的关系是
    \begin{Equation}
        p_\text{N0}=p(x_\text{n})=p_\text{P0}\exp[\frac{-qV_\text{D}}{\kB T}]
    \end{Equation}
\end{BoxFormula}
在\xref{fig:PN结的载流子浓度与电势分布},我们可以很直观的看到PN结中载流子浓度是如何分布的。
\begin{Figure}[PN结的载流子浓度与电势分布]
    \begin{FigureSub}[PN结的载流子浓度]
        \includegraphics[scale=0.75]{build/Chapter06B_01.fig.pdf}
    \end{FigureSub}\\ \vspace{0.5cm}
    \begin{FigureSub}[PN结的电势分布]
        \includegraphics[scale=0.75]{build/Chapter06B_02.fig.pdf}
    \end{FigureSub}
\end{Figure}

\begin{itemize}
    \item 空穴,在P区是多子,浓度较高,在N区则是少子,浓度较低。空穴浓度$p(x)$随着$x$的增加(由P区至N区),在势垒区会逐渐减小,由多子浓度$p_\text{P0}$\hspace{1ex}趋向少子浓度$p_\text{N0}$。
    \item 电子,在N区是多子,浓度较高,在P区则是少子,浓度较低。电子浓度$n(x)$随着$x$的减少(由N区至P区),在势垒区会逐渐减小,由多子浓度$n_\text{N0}$趋向少子浓度$n_\text{P0}$。
\end{itemize}

这里可能会有疑问,在\fancyref{fml:PN结的平衡载流子浓度}中的$n(x)$和$p(x)$内还包含未知的$V(x)$,\xref{fig:PN结的载流子浓度}中$n(x),p(x)$的图像,\xref{fig:PN结的电势分布}中$V(x)$的图像,又是怎么作出来的?我们会在本章稍后解答这个问题,我们将在计算PN结电容的过程中求出$V(x)$的具体表达式。
\section{PN结的电流电压特性}
在平衡PN结中,存在着具有一定宽度(空间上)和高度(能量上)的势垒,在势垒中出现了内建电场,载流子的扩散电流和漂移电流相互抵消,没有净电流通过PN结。本节讨论PN结的电流电压特性,就相当于是要讨论PN结在外加电压的非平衡状态下,将会如何工作?

简而言之,本节的目标是:\empx{研究外加电场作用下非平衡PN结的性质}。

\subsection{正向偏压下PN结的定性分析}
当对PN结外加正向偏压$V$时(即,P区接正极,N区接负极),能带变化如\xref{fig:正向偏压下的能带结构}所示。
\begin{Figure}[正向偏压下的能带结构]
    \includegraphics[scale=0.75]{build/Chapter06A_09.fig.pdf}
\end{Figure}
\begin{itemize}
    \item 势垒区内载流子浓度很小,电阻很大。
    \item 势垒区外载流子浓度很大,电阻很小。
\end{itemize}
因此,正向偏压$V$基本降落在势垒区,正向偏压在势垒区中产生了与内建电场方向相反的电场,从而削弱了内建电场,因此势垒区宽度减小,并且势垒高度由$qV_\text{D}$减小至$q(V_\text{D}-V)$。

势垒区的电场减弱,破坏了载流子的扩散运动和漂移运动原有的平衡,电场导致漂移,电场减弱,势垒区的扩散流就将大于漂移流。这样一来,当少子扩散通过势垒区后在势垒区边界出的浓度$n(x_\text{p})$和$p(x_\text{n})$,比相应的平衡少子浓度$n_\text{P0}, p_\text{P0}$要高。而这一部分多出来的少子就分别成为了P区和N区的非平衡载流子,它们将分别形成由势垒区边界$x_\text{p}$和$x_\text{n}$向P区和N区内部的电子扩散流和空穴扩散流。在扩散过程中,非平衡少子将逐渐与多子复合,最终将减小至$n_\text{P0}$和$p_\text{N0}$,理论上来说,扩散过程需要蔓延无限长的距离,但通常可以认为经过数倍扩散长度$L_\text{n}$和$L_\text{p}$的距离后,扩散流中的少子就被基本复合殆尽了,这一段区域即称为\uwave{扩散区}。

在\xref{chap:非平衡载流子}中,我们曾讨论过光注入的非平衡载流子,而在这里,这种通过对PN结外加正向偏压,使得PN结的势垒区边界出现的非平衡载流子,相应的就称为电注入的非平衡载流子。

\subsection{反向偏压下PN结的定性分析}
当对PN结外加反向偏压$V$时(即,P区接负极,N区接正极),能带变化如\xref{fig:反向偏压下的能带结构}所示。

\begin{Figure}[反向偏压下的能带结构]
    \includegraphics[scale=0.75]{build/Chapter06A_10.fig.pdf}
\end{Figure}

正向偏压和反相偏压下,PN结的特性有相似之处,我们下面对比着来看。

关于PN结势垒的变化,如\xref{fig:PN结的电势分布}所示
\begin{itemize}
    \item 正向偏压下,电场方向与内建电场相反,势垒宽度减小,势垒高度减小至$q(V_\text{D}-V)$。
    \item 反相偏压下,电场方向与内建电场相同,势垒宽度增大,势垒高度增大至$q(V_\text{D}+V)$。
\end{itemize}\goodbreak

关于PN结载流子浓度的变化,如\xref{fig:PN结的载流子浓度}所示
\begin{itemize}
    \item 正向偏压下,势垒区的扩散流强于漂移流,势垒边缘的少数载流子浓度$n(x_\text{p})$和$p(x_\text{n})$高于两端的少子浓度$n_\text{P0}$和$p_\text{N0}$,非平衡载流子将由PN结的势垒边缘向两端扩散。
    \item 反向偏压下,势垒区的扩散流弱于漂移流,势垒边缘的少数载流子浓度$n(x_\text{p})$和$p(x_\text{n})$低于两端的少子浓度$n_\text{P0}$和$p_\text{N0}$,非平衡载流子将由PN结的两端向势垒边缘扩散。
\end{itemize}
其实关于反向偏压下的载流子浓度,我们也可以这么理解,势垒区的扩散流弱于漂移流,漂移驱动的是少子,由于扩散流无法提供足够的少子,漂移将从PN结两端(即P区和N区的内部)进一步获得少子。这被形象的称为少数载流子的抽取或吸出。而在极端情况下,即假若反向电压很大,此时少数载流子几乎被抽取殆尽,在势垒区边界附近的少子浓度可以认为是零。

\subsection{非平衡PN结的能带图}
\xref{fig:正向偏压下的能带结构}和\xref{fig:反向偏压下的能带结构}中,注意到PN结在外加电压时,费米能级$E_\text{F}$分裂为了$E_\text{Fp},E_\text{Fn}$准费米能级\footnote{这里有必要澄清一下$E_\text{Fp},E_\text{Fn}$,在本章伊始,由\xref{fig:P型和N型半导体的能带图}引出\xref{fig:PN结的能带图}的过程中,记号$E_\text{Fn},E_\text{Fp}$是代表P区和N区的费米能级,我们看到,两块分离的P型和N型半导体$E_\text{Fn}\neq E_\text{Fp}$,两者形成PN结后则有$E_\text{Fn}=E_\text{Fp}$,即能带偏移,使P区和N区具有一致的费米能级。而当\xref{fig:PN结的能带图}作为平衡PN结引出正偏\xref{fig:正向偏压下的能带结构}和反偏\xref{fig:反向偏压下的能带结构}时,此时,记号$E_\text{Fn},E_\text{Fp}$的意义则分别转向表示电子和空穴的准费米能级,在平衡PN结中两者相等(平衡态的标志即统一的费米能级),在非平衡PN结中两者则不总是相等。}
\begin{itemize}
    \item 平衡区内有$E_\text{Fp}=E_\text{Fn}$,这很合理,平衡区和扩散区的分界即扩散的非平衡载流子的浓度可以忽略不计处,平衡区不存在非平衡载流子,而平衡态的标志即统一的费米能级。
    \item 平衡区中不妨记$E_\text{F}=E_\text{Fp}=E_\text{Fn}$,注意到P区和N区的费米能级$E_\text{F}$是不同的
    \begin{itemize}
        \item 正向偏压下,势垒减小为$V_\text{D}-V$,N区能带随之上移,N区$E_\text{F}$将大于P区。
        \item 反相偏压下,势垒增大为$V_\text{D}+V$,N区能带随之下移,N区$E_\text{F}$将小于P区。
    \end{itemize}
    \item 扩散区,由于有尚未复合完全的非平衡少数载流子的存在,费米能级发生分裂
    \begin{itemize}
        \item 正向偏压下,在P区,少子电子浓度因扩散增加,故$E_\text{Fn}$上移接近导带。
        \item 反向偏压下,在P区,少子电子浓度因抽取减小,故$E_\text{Fn}$下移远离导带。
        \item 正向偏压下,在N区,少子空穴浓度因扩散增加,故$E_\text{Fp}$下移接近价带。
        \item 反向偏压下,在N区,少子空穴浓度因抽取减小,故$E_\text{Fp}$上移远离价带。
    \end{itemize}
    \item 扩散区中准费米能级以线性方式变化,尚不清楚这是实际情况,还只是作图上的简化。
    \item 势垒区远小于扩散区\footnote{实际如此,\xref{fig:正向偏压下的能带结构}和\xref{fig:反向偏压下的能带结构}中势垒区和扩散区的长度未按此要求绘制。},故可以近似认为费米能级在势垒区不变。
\end{itemize}
这里我们可能会有疑问,注意到在反向偏压时,如\xref{fig:反向偏压下的能带结构}所示,准费米能级已经跑到价带或导带中了,此时难道不该适用\xref{sec:简并半导体}中简并半导体的理论了吗?这是图像观察不仔细所致的,这里,进入价带的是电子准费米能级,进入导带的是空穴准费米能级,反而是更加非简并了。\goodbreak

\subsection{非平衡PN结的载流子浓度}
在定性计算非平衡PN结的载流子浓度前,我们还要做出几点理想化的假设
\begin{enumerate}
    \item 小注入条件:注入的少数载流子浓度比平衡多数载流子浓度小得多。
    \item 突变耗尽层条件:耗尽层外的半导体是电中性的,即外加电压和接触电势差完全降落在耗尽层上,因此,注入的非平衡少数载流子在P区和N区的运动是纯扩散运动。
    \item 玻尔兹曼边界条件:耗尽层两端,载流子分布满足玻尔兹曼统计分布。
\end{enumerate}
在本小节,我们只需要计算载流子作为少子在扩散区的分布,因为\fancyref{fml:PN结的平衡载流子浓度}在载流子进入对方扩散区前都是仍然适用的,只不过要将$qV_D$替换\footnote{应指出的是,在\fancyref{fml:PN结多子浓度与少子浓度的关系}中的$qV_\text{D}$无需替换为$q(V_\text{D}-V)$,该关系不随外加电压而变。}为$q(V_\text{D}-V)$
\begin{Align}[10pt]
    n(x)&=\begin{cases}
        \mal{n_\text{N0}\exp[\frac{qV(x)-q(V_\text{D}-V)}{\kB T}]},&x\geq x_\text{p}\\
        \mal{n_\text{P}(x)},&x<x_\text{p}
    \end{cases}\\
    p(x)&=\begin{cases}
        \mal{p_\text{P0}\exp[\frac{-qV(x)}{\kB T}]},&\hspace{5.4em}x\leq x_\text{n}\\
        \mal{p_\text{N}(x)},&\hspace{5.4em}x>x_\text{n}
    \end{cases}
\end{Align}
这里我们用$V$的正负来表示正偏压和负偏压,若$V$为正即正偏压,若$V$为负即负偏压。

这里我们要计算的就是$n_\text{P}(x)$和$p_\text{N}(x)$,平衡时两者即$n_\text{P0}$和$p_\text{N0}$的常数,当外加偏压时,两者分别是由PN结势垒边界的浓度$n(x_\text{p})$和$p(x_\text{n})$向PN结两端的$n_\text{P0}$和$p_\text{N0}$的扩散过程。

\begin{BoxFormula}[PN结外加偏压时的少子浓度]
    PN结外加偏压$V$时,在N区,空穴的载流子浓度为
    \begin{Equation}
        p_\text{N}(x)=p_\text{N0}+p_\text{N0}\qty[\exp(\frac{qV}{\kB T})-1]\exp(\frac{x_\text{n}-x}{L_\text{p}})
    \end{Equation}
    PN结外加偏压$V$时,在P区,电子的载流子浓度为
    \begin{Equation}
        n_\text{P}(x)=n_\text{P0}+n_\text{P0}\qty[\exp(\frac{qV}{\kB T})-1]\exp(\frac{x-x_\text{p}}{L_\text{n}})
    \end{Equation}
\end{BoxFormula}

\begin{Proof}
    根据\fancyref{fml:非平衡态下的载流子浓度积}
    \begin{Equation}&[1]
        np=n_\text{i}^2\exp\qty(\frac{E_\text{Fn}-E_\text{Fp}}{\kB T})
    \end{Equation}
    势垒区P区边界$x=x_\text{P}$处,如\xref{fig:正向偏压下的能带结构}和\xref{fig:反向偏压下的能带结构}所示,有$E_\text{Fn}-E_\text{Fp}=qV$,故
    \begin{Equation}&[2]
        n_\text{P}(x_\text{p})p_\text{P}(x_\text{p})=n_\text{i}^2\exp(\frac{qV}{\kB T})
    \end{Equation}
    这里$p_\text{P}(x_\text{P})$作为P区多数载流子,可以作为常数$p_\text{P0}$代入
    \begin{Equation}&[3]
        n_\text{P}(x_\text{p})p_\text{P0}=n_\text{i}^2\exp(\frac{qV}{\kB T})
    \end{Equation}
    而考虑到$n_\text{i}^2=n_\text{P0}p_\text{P0}$
    \begin{Equation}&[5]
        n_\text{P}(x_\text{p})=n_\text{P0}\exp(\frac{qV}{\kB T})
    \end{Equation}
    所以说,注入P区边界的非平衡少数载流子浓度为
    \begin{Equation}&[6]
        \delt{n_\text{P}}(x_\text{p})=n_\text{P0}\qty[\exp(\frac{qV}{\kB T})-1]
    \end{Equation}
    类似的,注入N区边界的非平衡少数载流子浓度为
    \begin{Equation}&[7]
        \delt{p_\text{N}}(x_\text{n})=p_\text{N0}\qty[\exp(\frac{qV}{\kB T})-1]
    \end{Equation}
    这两式子就是连续性方程求解所需的边界条件了。
    
    根据\fancyref{eqt:连续性方程},在稳定态下,在N区中扩散的非平衡空穴的连续性方程为
    \begin{Equation}&[8]
        D_\text{p}\dv[2]{\delt{p_\text{N}}}{x}-\mu_\text{p}\Emf\dv{\delt{p_\text{N}}}{x}-\mu_\text{p}p_\text{N}\dv{\Emf}{x}-\frac{\delt{p_\text{N}}}{\tau_\text{p}}=0
    \end{Equation}
    在我们的假设下,外加电场和内建电场都仅存在于势垒区,扩散区无电场,故
    \begin{Equation}&[9]
        D_\text{P}\dv[2]{\delt{p_\text{N}}}{x}-\frac{\delt{p_\text{N}}}{\tau_\text{p}}=0
    \end{Equation}
    这是一个最简单的二阶微分方程,其通解为(其中$L_\text{p}=\sqrt{D_\text{p}\tau_\text{p}}$是扩散长度)
    \begin{Equation}&[10]
        \delt{p}_\text{N}(x)=A\exp(-\frac{x}{L_\text{p}})+B\exp(\frac{x}{L_\text{p}})
    \end{Equation}
    由于$\delt{p}_\text{N}$是在正半轴的N区上,故$B=0$舍去正指数项
    \begin{Equation}&[11]
        \delt{p}_\text{N}(x)=A\exp(-\frac{x}{L_\text{p}})
    \end{Equation}
    令$x=x_\text{n}$
    \begin{Equation}&[12]
        \delt{p}_\text{N}(x_\text{n})=A\exp(-\frac{x_\text{n}}{L_\text{p}})
    \end{Equation}
    将\xrefpeq{12}和\xrefpeq{7}联立
    \begin{Equation}&[13]
        A\exp(-\frac{x_\text{n}}{L_\text{p}})=p_\text{N0}\qty[\exp(\frac{qV}{\kB T})-1]
    \end{Equation}
    解得常数$A$
    \begin{Equation}&[14]
        A=p_\text{N0}\qty[\exp(\frac{qV}{\kB T})-1]\exp(\frac{x_\text{n}}{L_\text{p}})
    \end{Equation}
    将\xrefpeq{14}代回\xrefpeq{11}
    \begin{Equation}
        \delt{p_\text{N}}(x)=p_\text{N0}\qty[\exp(\frac{qV}{\kB T})-1]\exp(\frac{x_\text{n}-x}{L_\text{p}})
    \end{Equation}
    而$p_\text{N}(x)=p_\text{N0}+\delt{p}_\text{N}(x)$,故
    \begin{Equation}
        p_\text{N}(x)=p_\text{N0}+p_\text{N0}\qty[\exp(\frac{qV}{\kB T})-1]\exp(\frac{x_\text{n}-x}{L_\text{p}})
    \end{Equation}
    类似的可以得到
    \begin{Equation}
        n_\text{P}(x)=n_\text{P0}+n_\text{P0}\qty[\exp(\frac{qV}{\kB T}-1)]\exp(\frac{x-x_\text{p}}{L_\text{n}})
    \end{Equation}
    这里$x_\text{n}-x$变为$x-x_\text{p}$与\xrefpeq{10}舍去正指数项还是负指数项有关。
\end{Proof}
\xref{fig:PN结的载流子浓度}极为生动的展现本小节的工作,即正偏和反偏下PN结的载流子分布将如何变化?

\subsection{PN结的电流电压特性}
当在PN结上加一定电压时,在PN结上就会通过一定的电流,这些电流就来自\xref{subsec:非平衡PN结的载流子浓度}中我们计算的少子扩散流。很明显,通过PN结的总电流密度$J$就等于势垒边界$x_\text{n}$和$x_\text{p}$处的空穴扩散电流密度$J_\text{p}(x_\text{n})$和电子扩散电流密度$J_\text{n}(x_\text{p})$,当然,我们会说,为什么偏偏选取势垒边界处的电流密度来计算?实际上,扩散过程伴随着复合,扩散流其实是越来越弱的,但根据电流连续性原理,这些电流不可能凭空消失,实际上,随着载流子的复合,少子的扩散电流逐渐转化为多子的漂移电流,因此,选取哪个截面都一样,那就不妨选取最便捷的势垒边界了。

\begin{BoxEquation}[肖克利方程]
    \uwave{肖克利方程}(Shockley Equation)描述了理想PN结的电流电压特性
    \begin{Equation}
        J=J_\text{s}\qty[\exp\qty(\frac{qV}{\kB T})-1]
    \end{Equation}
    其中$J_\text{s}$为
    \begin{Equation}
        J_\text{s}=\frac{qD_\text{n}n_\text{P0}}{L_\text{n}}+\frac{qD_\text{p}p_\text{N0}}{L_\text{p}}
    \end{Equation}
\end{BoxEquation}

\begin{Proof}
    根据\xref{fml:载流子的漂移扩散}和\fancyref{fml:PN结外加偏压时的少子浓度},计算空穴扩散流$J_\text{p}$
    \begin{Equation}&[1]
        J_\text{p}(x)=-qD_\text{p}\dv{p_\text{N}}{x}=\frac{qD_\text{p}}{L_\text{p}}p_\text{N0}\qty[\exp(\frac{qV}{\kB T})-1]\exp(\frac{x_\text{n}-x}{L_\text{p}})
    \end{Equation}
    根据\xref{fml:载流子的漂移扩散}和\fancyref{fml:PN结外加偏压时的少子浓度},计算电子扩散流$J_\text{n}$
    \begin{Equation}&[2]
        J_\text{n}(x)=qD_\text{n}\dv{n_\text{P}}{x}=\frac{qD_\text{p}}{L_\text{p}}n_\text{P0}\qty[\exp(\frac{qV}{\kB T})-1]\exp(\frac{x-x_\text{p}}{L_\text{n}})
    \end{Equation}
    在\xrefpeq{1}中代入$x=x_\text{n}$
    \begin{Equation}
        J_\text{p}(x_\text{n})=\frac{qD_\text{p}p_\text{N0}}{L_\text{p}}\qty[\exp(\frac{qV}{\kB T})-1]
    \end{Equation}
    在\xrefpeq{2}中代入$x=x_\text{p}$
    \begin{Equation}
        J_\text{n}(x_\text{p})=\frac{qD_\text{n}n_\text{P0}}{L_\text{n}}\qty[\exp(\frac{qV}{\kB T})-1]
    \end{Equation}
    因此
    \begin{Equation}*
        J=J_\text{p}(x_\text{n})+J_\text{n}(x_\text{p})=\qty(\frac{qD_\text{p}p_\text{N0}}{L_\text{p}}+\frac{qD_\text{n}n_\text{P0}}{L_\text{n}})\qty[\exp(\frac{qV}{\kB T})-1]
    \end{Equation}
    若引入$J_\text{s}$作为代换变量
    \begin{Equation}*
        J=J_\text{s}\qty[\exp(\frac{qV}{\kB T})-1]\qedhere
    \end{Equation}
\end{Proof}

根据\fancyref{eqt:肖克利方程}
\begin{itemize}
    \item 正向导通:PN结在正向偏压下,电流将随电压指数增大,故称为正向导通。
    \item 反向截止:PN结在反向偏压下,电流将趋于$-J_\text{s}$的定值,这表明,反向电流密度为是与外界常量无关的常数(称为反向饱和电流)。由于$J_\text{s}$很小,因此称PN结反向截止。
\end{itemize}
这表明,\empx{PN结具有单向导电性},正向导通,反向截止,这种特性也称为\uwave{整流效应}。