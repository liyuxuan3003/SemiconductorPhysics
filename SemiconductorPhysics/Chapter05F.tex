\section{爱因斯坦关系式}
在\xref{chap:半导体的导电性}中,我们曾在\xref{sec:载流子的漂移运动}中详细讨论过载流子在电场作用下的漂移运动,这会产生漂移电流。在\xref{chap:非平衡载流子}中,我们亦在\xref{sec:载流子的扩散运动}中讨论过非平衡载流子在浓度梯度作用下的扩散运动,这会产生扩散电流。而本节要解决的中心问题是,\empx{载流子的漂移运动和扩散运动间有何种关系}?

根据\fancyref{fml:漂移电流密度}
\begin{Equation}
    (J_\text{n})_\text{漂}=qn\mu_\text{n}\Emf\qquad
    (J_\text{p})_\text{漂}=qp\mu_\text{p}\Emf
\end{Equation}
根据\fancyref{fml:扩散电流密度}
\begin{Equation}
    (J_\text{n})_\text{扩}=qD_\text{n}\dv{\delt{n}}{x}\qquad
    (J_\text{n})_\text{扩}=-qD_\text{p}\dv{\delt{p}}{x}
\end{Equation}
简而言之,电场会导致漂移电流,浓度梯度会导致扩散电流。\xref{fig:载流子的漂移和扩散}形象体现了这一点
\begin{itemize}
    \item 漂移电流$(J)_\text{漂}$是电场作用的结果,漂移电流$(J)_\text{漂}$的方向与电场$\Emf$方向一致,这其实是因为,尽管电子和空穴的运动方向分别与电场相反和相同,但是由于电子带负电,两者的电流方向都与电场方向相一致,或者可以换一个角度,因为$J=\sigma\Emf$总是成立的。
    \item 扩散电流$(J)_\text{扩}$是浓度梯度的结果,扩散方向与相应载流子的浓度负梯度是一致的,这是因为,载流子总是由浓度较高处流向浓度较低处,但是,就电流而言,考虑到电子和空穴分别带负电和正电,故$(J_\text{n})_\text{扩}$和$(J_\text{p})_\text{扩}$与浓度负梯度的方向分别是相反和相同的。
\end{itemize}\vspace{-1ex}
\begin{Figure}[载流子的漂移和扩散]
    \includegraphics[scale=0.9]{build/Chapter05F_01.fig.pdf}
\end{Figure}
而一般情况下,扩散和漂移是同时发生的。但在\xref{fig:载流子的漂移和扩散}的基础上,还需要认识到,扩散的关键在于载流子的浓度分布不均,然而,导致载流子的浓度分布不均的因素并不只有一个
\begin{itemize}
    \item 不均匀的注入(例如单侧光照)会导致非平衡载流子浓度$\delt{n},\delt{p}$分布不均。
    \item 不均匀的掺杂亦会导致平衡载流子浓度$n_0,p_0$分布不均。
\end{itemize}
由于$n=n_0+\delt{n},~p=p_0+\delt{p}$,故一般情况下,浓度梯度应写作$\dv*{n}{x}$和$\dv*{p}{x}$。
\begin{BoxFormula}[载流子的漂移扩散]*
    电子在漂移和扩散作用下的电流密度满足
    \begin{Equation}
        J_\text{n}=(J_\text{n})_\text{漂}+(J_\text{n})_\text{扩}=qn\mu_\text{n}\Emf+qD_\text{n}\dv{n}{x}
    \end{Equation}
    空穴在漂移和扩散作用下的电流密度满足
    \begin{Equation}
        J_\text{p}=(J_\text{p})_\text{漂}+(J_\text{p})_\text{扩}=qp\mu_\text{p}\Emf-qD_\text{p}\dv{p}{x}
    \end{Equation}
\end{BoxFormula}
我们说,本节要研究漂移和扩散的关系,本质上,就是要研究$\mu_\text{n},\mu_\text{p}$和$D_\text{n},D_\text{p}$,即漂移率和扩散系数间的关系。而令人惊讶的是,这两者的关系竟然非常简洁!这就是爱因斯坦关系式。

\begin{BoxLaw}[爱因斯坦关系式]
    \uwave{爱因斯坦关系式}(Einstein Relation)指出,漂移率与扩散系数成正比
    \begin{Equation}
        \frac{D_\text{n}}{\mu_\text{n}}=
        \frac{\kB T}{q}\qquad
        \frac{D_\text{p}}{\mu_\text{p}}=
        \frac{\kB T}{q}
    \end{Equation}
\end{BoxLaw}

\begin{Proof}
    让我们来考虑一块处于热平衡状态的N型半导体,没有光照,没有外加电场,但是其掺杂并不均匀,故平衡载流子浓度$n_0,p_0$均为关于$x$的函数$n_0(x),p_0(x)$,故存在扩散电流
    \begin{Equation}&[1]
        (J_\text{n})_\text{扩}=qD_\text{n}\dv{n_0}{x}\qquad
        (J_\text{p})_\text{扩}=-qD_\text{p}\dv{p_0}{x}
    \end{Equation}
    或许我们会想,既然没有外加电场,那此时就不应有漂移电流了?答案是否定的。
    
    掺杂不均匀与注入不均匀导致的扩散运动的最大区别在于,掺杂不均匀时,载流子可以发生扩散,但是,电离杂质是不能移动的,电离杂质的不均匀性将会破坏半导体的电中性,从而在半导体中形成内建电场$\Emf$,由于平衡条件下,不存在宏观电流,因此,内建电场$\Emf$的方向必然是反抗扩散电流$(J_\text{n})_\text{扩},(J_\text{p})_\text{扩}$的,内建电场形成的漂移电流$(J_\text{n})_\text{漂},(J_\text{p})_\text{漂}$应分别与之抵消
    \begin{Equation}&[2]
        (J_\text{n})_\text{漂}=n_0q\mu_\text{n}\Emf\qquad
        (J_\text{p})_\text{漂}=p_0q\mu_\text{p}\Emf
    \end{Equation}
    以及
    \begin{Equation}&[3]
        J_\text{n}=(J_\text{n})_\text{漂}+(J_\text{n})_\text{扩}=0\qquad
        J_\text{p}=(J_\text{p})_\text{漂}+(J_\text{p})_\text{扩}=0
    \end{Equation}
    我们以$J_\text{n}$部分为例求解,这是一个关于$n_0$的微分方程
    \begin{Equation}&[4]
        n_0q\mu_\text{n}\Emf+qD_\text{n}\dv{n_0}{x}=0
    \end{Equation}
    不过,我们的目的并不是求出$n_0$,而是利用$n_0$的性质解出$D_\text{n}$和$\mu_\text{n}$间的关系。

    注意到,当半导体内部出现电场时,半导体各处电势不相等,而电势为电场的负梯度
    \begin{Equation}&[5]
        \Emf=-\dv{V}{x}
    \end{Equation}
    在考虑电子的能量时,必须计入附加静电势能$-qV(x)$,因而,电子导带底的能量应由$E_\text{c}$修正为$E_\text{c}-qV(x)$,随$x$变化。因此,根据\fancyref{fml:导带电子浓度},此时$n_0$满足
    \begin{Equation}&[6]
        n_0(x)=N_\text{c}\exp[\frac{E_\text{F}-E_\text{c}+qV(x)}{\kB T}]
    \end{Equation}
    两端求导,注意上式右端仅$V(x)$与$x$有关
    \begin{Equation}&[7]
        \dv{n_0}{x}=\frac{q}{\kB T}N_\text{c}\exp[\frac{E_\text{F}-E_\text{c}+qV(x)}{\kB T}]\dv{V}{x}
    \end{Equation}
    将\xrefpeq{6}代入\xrefpeq{7},并根据\xrefpeq{5}用$-\Emf$代换$\dv*{V}{x}$
    \begin{Equation}&[8]
        \dv{n_0}{x}=-\frac{q}{\kB T}n_0\Emf
    \end{Equation}
    将\xrefpeq{8}代入\xrefpeq{4},代换其$\dv*{n_0}{x}$
    \begin{Equation}
        n_0q\mu_\text{n}\Emf-qD_\text{n}\frac{q}{\kB T}n_0\Emf=0
    \end{Equation}
    即
    \begin{Equation}
        \mu_\text{n}=\frac{qD_\text{n}}{\kB T}
    \end{Equation}
    或
    \begin{Equation}
        \frac{D_\text{n}}{\mu_\text{n}}=\frac{\kB T}{q}
    \end{Equation}
    类似的亦可以得到
    \begin{Equation}
        \frac{D_\text{p}}{\mu_\text{p}}=\frac{\kB T}{q}
    \end{Equation}
    由此,我们就得到了漂移率和扩散系数间的关系。
\end{Proof}