\section{复合理论}
现在的问题是,非平衡载流子到底是怎么复合的?如\xref{tab:复合的分类}所示
\begin{TableLong}[复合的分类]
<类别&类型&解释\\>
\mr{2}{复合的微观机构}
&\xgp[1ex]{直接复合}&电子从价带直接跃迁至导带,引起电子和空穴的直接复合。\\*
&\xgp[1ex]{间接复合}&电子和空穴通过禁带能级,即复合中心进行复合。\\ \hlinelig
\mr{2}{复合的发生位置}
&\xgp[1ex]{体内复合}&复合发生在半导体的内部。\\*
&\xgp[1ex]{间接复合}&复合发生在半导体的表面。\\ \hlinelig
\mr{3}[-0.5ex]{复合的能量过程}
&\xgp[1ex]{发射光子}&能量以光子形式释放,伴随着复合有发光现象。\\*
&\xgp[1ex]{发射声子}&能量以声子形式释放,伴随着复合晶格的振动被增强。\\*
&\xgp[1ex]{俄歇复合}&能量给予其他载流子,增加它们的动能。\\
\end{TableLong}
这里需要解释的是,什么是\uwave{复合中心}(Recombination Center)?复合中心代表的是禁带中的能级,通常记为$E_\text{t}$,实际上这些禁带能级就是杂质能级,只不过考虑复合时,我们并不再关心它到底是施主杂质能级还是受主杂质能级。复合中心的存在将使得复合变得更为容易,这是因为,直接复合时,载流子的跃迁需要跨越整个禁带,能量跨度较大。间接复合时,禁带中存在复合中心,因此只要电子和空穴跨越较小的能量分别到达复合中心,复合就可以发生。

\subsection{直接复合}
半导体中,总是存在载流子产生和复合两个相反的过程
\begin{itemize}
    \item 将单位时间单位体积内产生的电子--空穴对数$G$,称为\uwave{产生率}(Generation Rate)。
    \item 将单位时间单位体积内复合的电子--空穴对数$R$,称为\uwave{复合率}(Recombination Rate)。
\end{itemize}
直接复合的复合率$R$显然与电子浓度和空穴浓度的乘积$np$成正比,故有下式。
\begin{BoxFormula}[直接复合的复合率]
    直接复合的复合率,正比于载流子的浓度积
    \begin{Equation}
        R=rnp
    \end{Equation}
\end{BoxFormula}
直接复合的复合率$R$中,比例系数$r$称为\uwave{电子--空穴复合概率},复合概率与电子和空穴的热运动速度有关,这里$r$是一个平均的考量。复合概率$r$仅是温度的函数,与$n,p$无关。

直接复合的产生率$G$则有些不一样,在一定温度下,价带中的每个电子都有一定的概率被激发到导带,从而形成一对电子和空穴。但根据泡利不相容,如果导带和价带上已经有一些电子和空穴了,那产生率也就会相应小一些,因为电子不能被激发到已被电子占据的状态上。然而在非简并情况下,导带基本上是全空的,价带基本上的全满的,所以,可以认为激发概率不受载流子浓度$n,p$影响。因而,产生率$G$在非简并情况下,基本是相同的,是一个常数。
\begin{BoxFormula}[直接复合的产生率]
    直接复合的产生率,是一个常数
    \begin{Equation}
        G=rn_\text{i}^2        
    \end{Equation}
\end{BoxFormula}
\begin{Proof}
    在热平衡状态下,产生率与复合率是相同的
    \begin{Equation}
        G=R
    \end{Equation}
    代入\fancyref{fml:直接复合的复合率},热平衡时$n=n_0, p=p_0$
    \begin{Equation}
        G=R=rn_0p_0=rn_\text{i}^2
    \end{Equation}
    由于产生率$G$是一个常数,因此,在非热平衡态时产生率$G$也为
    \begin{Equation}*
        G=rn_\text{i}^2\qedhere
    \end{Equation}
\end{Proof}

在进一步讨论之前,我们先要阐明两个事实。首先,复合率$R$减去产生率$G$就等于非平衡载流子的净复合率,记为$U_\text{d}$。其次,在\xref{sec:非平衡载流子的寿命}中我们提到过,$1/\tau$表示复合概率,$\delt{p}/\tau$表示复合率,而这里复合率就是$U_\text{d}=R-G$,因此可以从中解出非平衡载流子的寿命$\tau=\delt{p}/U_\text{d}$。

\begin{BoxDefinition}[直接复合的净复合率]
    直接复合中,净复合率是复合率$R$和产生率$G$的差
    \begin{Equation}
        U_\text{d}=R-G
    \end{Equation}
\end{BoxDefinition}

\begin{BoxFormula}[净复合率与寿命的关系]
    净复合率与非平衡载流子寿命的关系是
    \begin{Equation}
        \tau=\frac{\delt{p}}{U_\text{d}}
    \end{Equation}
\end{BoxFormula}

现在,我们可以继续讨论直接复合中的净复合率和非平衡载流子寿命了。

\begin{BoxFormula}[直接复合的净复合率]
    直接复合的净复合率满足
    \begin{Equation}
        U_\text{d}=r(n_0+p_0)\delt{p}+r(\delt{p})^2
    \end{Equation}
\end{BoxFormula}
\begin{Proof}
    根据\fancyref{def:直接复合的净复合率}
    \begin{Equation}&[1]
        U_\text{d}=R-G
    \end{Equation}
    根据\fancyref{fml:直接复合的复合率}和\fancyref{fml:直接复合的产生率}
    \begin{Equation}&[2]
        U_\text{d}=rnp-rn_\text{i}^2=r(np-n_\text{i}^2)
    \end{Equation}
    在\xrefpeq{2}中,代入$n=n_0+\delt{n}$和$p=p_0+\delt{p}$,以及光注入的$\delt{n}=\delt{p}$
    \begin{Equation}&[3]
        U_\text{d}=r[(n_0+\delt{p})(p_0+\delt{p})-n_\text{i}^2]
    \end{Equation}
    化简得到
    \begin{Equation}&[4]
        U_\text{d}=r[n_0p_0+(n_0+p_0)\delt{p}+(\delt{p}^2-n_\text{i}^2)]
    \end{Equation}
    根据\fancyref{fml:载流子的浓度乘积},这里$n_0,p_0$和$n_\text{i}$可以约掉
    \begin{Equation}*
        U_\text{d}=r(n_0+p_0)+r(\delt{p})^2\qedhere
    \end{Equation}
\end{Proof}

\begin{BoxFormula}[直接复合的寿命]
    直接复合的寿命满足
    \begin{Equation}
        \tau=\frac{1}{r(n_0+p_0+\delt{p})}
    \end{Equation}
    在小注入条件下,即$\delt{p}\ll(n_0+p_0)$时,可以近似为
    \begin{Equation}
        \tau=\frac{1}{r(n_0+p_0)}
    \end{Equation}
    在大注入条件下,即$\delt{p}\gg(n_0+p_0)$时,可以近似为
    \begin{Equation}
        \tau=\frac{1}{r\delt{p}}
    \end{Equation}
\end{BoxFormula}

\begin{Proof}
    根据\fancyref{fml:净复合率与寿命的关系}和\fancyref{fml:直接复合的净复合率}
    \begin{Equation}
        \tau=\frac{\delt{p}}{U_\text{d}}=\frac{\delt{p}}{r(n_0+p_0)\delt{p}+r(\delt{p})^2}
    \end{Equation}
    即
    \begin{Equation}
        \tau=\frac{1}{r(n_0+p_0+\delt{p})}
    \end{Equation}
    两个近似很容易由此得到。
\end{Proof}

\fancyref{fml:直接复合的寿命}指出
\begin{itemize}
    \item 在小注入条件下,寿命是一个无关非平衡载流子浓度的常数,仅却决于半导体的温度和掺杂情况,考虑到,N型半导体中$n_0+p_0\approx n_0$,P型半导体中$n_0+p_0\approx p_0$,因此,可以说,寿命与多数载流子的浓度成反比,或者说,半导体电导率越高,寿命就越短。
    \item 在大注入条件下,寿命与非平衡载流子的浓度成反比,不再是常数。
\end{itemize}
复合概率$r$是可以通过实验测量的,然而,由此计算出的硅和锗的非平衡载流子的寿命竟达到了$\tau=3.5\si{s}$和$\tau=0.3\si{s}$,而实际上,两者的最大寿命都在$\si{ms}$量级。这是因为,硅和锗中起到主要作用的复合机构,不是直接复合,而是间接复合,后者我们会在下一小节具体讨论。

复合到底是由直接复合主导还是由间接复合主导?通常来说
\begin{itemize}
    \item 禁带宽度较小时,直接复合将主导复合过程。
    \item 禁带宽度较大时,间接复合将主导复合过程。
\end{itemize}
这是合理的,间接复合与禁带中的复合中心有关,这只有在禁带宽度较大时有意义。

\subsection{间接复合}
间接复合的关键,在于复合中心$E_\text{t}$的存在,其在复合过程中相当于起到了“台阶”的作用。

间接复合存在四个微观过程,如\xref{fig:间接复合的四个微观过程}所示
\begin{enumerate}
    \item 俘获电子过程,以\uwave{电子俘获率} $R_\text{n}$描述,代表复合中心$E_\text{t}$从导带俘获电子。
    \item 发射电子过程,以\uwave{电子产生率} $G_\text{n}$描述,代表复合中心$E_\text{t}$上的电子被激发到导带。
    \item 俘获空穴过程,以\uwave{空穴俘获率} $R_\text{p}$描述,代表复合中心$E_\text{t}$从价带伏虎空穴。
    \item 发射电子过程,以\uwave{空穴产生率} $G_\text{p}$描述,代表复合中心$E_\text{t}$上的空穴被激发到价带。
\end{enumerate}
\begin{Figure}[间接复合的四个微观过程]
    \includegraphics{build/Chapter05D_02.fig.pdf}
\end{Figure}
这里的“俘获”和“发射”都是以复合中心$E_\text{t}$的视角来谈的。

那么,现在的问题是,$R_\text{n},G_\text{n},R_\text{p},G_\text{p}$如何表达呢?关键在于四个浓度和两个模式。四个浓度是指,导带电子浓度$n$、价带空穴浓度$p$、复合中心的电子浓度$n_\text{t}$和空穴浓度$N_\text{t}-n_\text{t}$,这里的$N_\text{t}$即复合中心的掺杂浓度,相当于过去的$N_\text{D},N_\text{A}$。两个模式是指,俘获过程正比于两种载流子的浓度乘积,发射过程则只取决于复合中心上的相应载流子的浓度(非简并半导体)。

\begin{BoxFormula}[电子俘获率]
    电子俘获率$R_\text{n}$,正比于复合中心空穴浓度与导带电子浓度的积
    \begin{Equation}
        R_\text{n}=r_\text{n}n(N_\text{t}-n_\text{t})
    \end{Equation}
    这里的比例系数$r_\text{n}$\hspace{0.25em},称为\uwave{电子俘获系数}。
\end{BoxFormula}
\begin{BoxFormula}[电子发射率]
    电子发射率$G_\text{n}$,正比于复合中心的电子浓度
    \begin{Equation}
        G_\text{n}=s_{-}n_\text{t}
    \end{Equation}
    这里的比例系数$s_{-}$,称为\uwave{电子发射系数}。
\end{BoxFormula}
\begin{BoxFormula}[空穴俘获率]
    空穴俘获率$R_\text{p}$,正比于复合中心电子浓度与导带空穴浓度的积
    \begin{Equation}
        R_\text{p}=r_\text{p}pn_\text{t}
    \end{Equation}
    这里的比例系数$r_\text{p}$\hspace{0.25em},称为\uwave{空穴俘获系数}。
\end{BoxFormula}
\begin{BoxFormula}[空穴发射率]
    空穴发射率$G_\text{p}$,正比于复合中心的空穴浓度
    \begin{Equation}
        G_\text{p}=s_{+}(N_\text{t}-n_\text{t})
    \end{Equation}
    这里的比例系数$s_{+}$,称为\uwave{空穴发射系数}。
\end{BoxFormula}
现在我们要做这样一件事,简而言之,我们实在不太喜欢发射系数$s_{-},s_{+}$,我们希望将发射系数$s_{-},s_{+}$用俘获系数$r_\text{n},r_\text{p}$替代。为了达成这一目的,我们需要利用平衡时俘获率和发射率相等(平衡时俘获过程和发射过程必须相互抵消),即$R_\text{n}=G_\text{n}$和$R_\text{p}=G_\text{p}$来建立方程。

\begin{BoxFormula}[电子发射系数的再表示]
    电子发射系数可以用电子俘获系数表示为
    \begin{Equation}
        s_{-}=r_\text{n}n_1\qquad n_1=N_\text{c}\exp(\frac{E_\text{t}-E_\text{c}}{\kB T}) 
    \end{Equation}
    电子发射率相应可以表示为
    \begin{Equation}
        G_\text{n}=r_\text{n}n_1n_\text{t}
    \end{Equation}
\end{BoxFormula}

\begin{Proof}
    在平衡时,电子俘获率与电子发射率相等
    \begin{Equation}&[1]
        R_\text{n}=G_\text{n}
    \end{Equation}
    代入\fancyref{fml:电子俘获率}和\fancyref{fml:电子发射率}
    \begin{Equation}&[2]
        r_\text{n}n_0(N_\text{t}-n_\text{t0})=s_{-}n_\text{t0}
    \end{Equation}
    稍作整理,将$s_{-}$移至一端
    \begin{Equation}&[3]
        s_{-}=r_\text{n}n_0\qty(\frac{N_\text{t}}{n_\text{t0}}-1)
    \end{Equation}
    这里的关键在于$n_0$和$n_\text{t0}$的表达式是什么?根据\fancyref{fml:导带电子浓度},此处$n_0$应为
    \begin{Equation}&[4]
        n_0=N_\text{c}\exp(\frac{E_\text{F}-E_\text{c}}{\kB T})
    \end{Equation}
    根据\fancyref{fml:施主浓度}和\fancyref{fml:受主浓度},取决于$n_\text{t0}$是施主还是受主
    \begin{Equation}&[5]
        \qquad
        n_\text{t0}=N_\text{t}\frac{1}{1+g_\text{D}^{-1}\exp[(E_\text{t}-E_\text{F})/\kB T]}\qquad
        n_\text{t0}=N_\text{t}\frac{1}{1+g_\text{A}\vphantom{^{-1}}\exp[(E_\text{t}-E_\text{F})/\kB T]}
        \qquad
    \end{Equation}
    简单起见,在计算$n_\text{t0}$时,简并因子$g_\text{D}$或$g_\text{A}$可以忽略
    \begin{Equation}&[6]
        n_\text{t0}=\frac{N_\text{t}}{1+\exp[(E_\text{t}-E_\text{F})/\kB T]}
    \end{Equation}
    将\xrefpeq{4}和\xrefpeq{6}代入\xrefpeq{3}
    \begin{Equation}&[7]
        s_{-}=r_\text{n}N_\text{c}\exp(\frac{E_\text{F}-E_\text{c}}{\kB T})\exp(\frac{E_\text{t}-E_\text{F}}{\kB T})
    \end{Equation}
    化简得
    \begin{Equation}
        s_{-}=r_\text{n}N_\text{c}\exp(\frac{E_\text{t}-E_\text{c}}{\kB T})
    \end{Equation}
    如果我们引入
    \begin{Equation}
        n_1=N_\text{c}\exp(\frac{E_\text{t}-E_\text{c}}{\kB T})
    \end{Equation}
    则
    \begin{Equation}*
        s_{-}=r_\text{n}n_1\qedhere
    \end{Equation}
\end{Proof}
这里$n_1$没有明确的物理意义,只是一个代换变量。但从数学形式上看,这里的$n_1$相当于是费米能级$E_\text{F}$位于复合中心能级$E_\text{t}$时的价带电子浓度,之后的$p_1$也可以类似予以解释。

\begin{BoxFormula}[空穴发射系数的再表示]
    空穴的发射系数可以用空穴俘获系数表示为
    \begin{Equation}
        s_{+}=r_\text{p}p_1\qquad
        p_1=N_\text{v}\exp(\frac{E_\text{v}-E_\text{t}}{\kB T})
    \end{Equation}
    空穴发射率相应可以表示为
    \begin{Equation}
        G_\text{p}=r_\text{p}p_1(N_\text{t}-n_\text{t})
    \end{Equation}
\end{BoxFormula}

\begin{Proof}
    在平衡时,空穴俘获率和空穴发射率相等
    \begin{Equation}
        R_\text{p}=G_\text{p}
    \end{Equation}
    代入\fancyref{fml:空穴俘获率}和\fancyref{fml:空穴发射率}
    \begin{Equation}
        r_\text{p}p_0n_\text{t0}=s_{+}(N_\text{t}-n_\text{t0})
    \end{Equation}
    稍作整理,将$s_{+}$移至一端
    \begin{Equation}
        s_{+}=r_\text{p}p_0\qty(\frac{n_\text{t0}}{N_\text{t}-n_\text{t0}})
    \end{Equation}
    即
    \begin{Equation}
        s_{+}=r_\text{p}p_0\qty(\frac{N_\text{t}}{n_\text{t0}}-1)^{-1}
    \end{Equation}
    代入$p_0$和$n_\text{t0}$的表达式,根据\fancyref{fml:导带电子浓度}和前述经验
    \begin{Equation}
        s_{+}=r_\text{p}N_\text{v}\exp(\frac{E_\text{v}-E_\text{F}}{\kB T})\exp(\frac{E_\text{F}-E_\text{t}}{\kB T})
    \end{Equation}
    化简得
    \begin{Equation}
        s_{+}=r_\text{p}N_\text{v}\exp(\frac{E_\text{v}-E_\text{t}}{\kB T})
    \end{Equation}
    如果我们引入
    \begin{Equation}
        p_1=N_\text{v}\exp(\frac{E_\text{v}-E_\text{v}}{\kB T})
    \end{Equation}
    则
    \begin{Equation}*
        s_{+}=r_\text{p}p_1\qedhere
    \end{Equation}
\end{Proof}

至此,我们已经分别求出了四个微观过程的数学表达式,整理在\xref{tab:间接复合的四个微观过程}中。

\begin{Table}[间接复合的四个微观过程]
<物理量&表达式&说明(正比于?)\\>
电子俘获率&$R_\text{n}=r_\text{n}n(N_\text{t}-n_\text{t})$&复合中心空穴浓度与导带电子浓度的积\\
电子发射率&$G_\text{n}=r_\text{n}n_1n_\text{t}=s_{-}n_\text{t}$&复合中心电子浓度\\
空穴俘获率&$R_\text{p}=r_\text{p}pn_\text{t}$&复合中心电子浓度与价带空穴浓度的积\\
空穴发射率&$G_\text{p}=r_\text{p}p_1(N_\text{t}-n_\text{t})=s_{+}(N_\text{t}-n_\text{t})$&复合中心空穴浓度\\
\end{Table}
接下来,我们有两个任务
\begin{enumerate}
    \item 确定复合中心的电子浓度$n_\text{t}$的表达式。
    \item 确定净复合率$U$和非平衡载流子的寿命$\tau$。
\end{enumerate}

关于$n_\text{t}$的计算,其关键在于以下等式
\begin{Equation}
    \underbrace{R_\text{p}+G_\text{p}}_{\text{复合中心的空穴增加}}=\underbrace{R_\text{n}+G_\text{p}}_{\text{复合中心的电子增加}}
\end{Equation}
代入\xref{tab:间接复合的四个微观过程}的相关结论
\begin{Equation}
    r_\text{n}n_1n_\text{t}+r_\text{p}pn_\text{t}=r_\text{n}n(N_\text{t}-n_\text{t})+r_\text{p}p_1(N_\text{t}-n_\text{t})
\end{Equation}
整理得到
\begin{Equation}
    r_\text{n}(n+n_1)n_\text{t}+r_\text{p}(p+p_1)n_\text{t}=N_\text{t}(r_\text{n}n+r_\text{p}p_1)
\end{Equation}
由此很容易解出$n_\text{t}$为
\begin{BoxFormula}[间接复合的复合中心电子浓度]
    间接复合中,复合中心的电子浓度满足
    \begin{Equation}
        n_\text{t}=\frac{N_\text{t}(r_\text{n}n+r_\text{p}p_1)}{r_\text{n}(n+n_1)+r_\text{p}(p+p_1)}
    \end{Equation}
\end{BoxFormula}
关于净复合率的计算,其关键在于以下等式
\begin{Equation}
    \underbrace{R_\text{p}-G_\text{p}}_\text{价带的空穴减小}=
    \underbrace{R_\text{n}-G_\text{n}}_\text{导带的电子减小}
\end{Equation}
实际上,这就是间接复合中的净复合率,为了与直接复合区分,将$U_\text{d}$改记为$U$。
\begin{BoxDefinition}[间接复合的净复合率]
    间接复合中,净复合率是
    \begin{Equation}
        U=R_\text{n}-G_\text{n}=R_\text{p}-G_\text{p}
    \end{Equation}
\end{BoxDefinition}
\begin{BoxFormula}[间接复合的净复合率]
    间接复合的净复合率满足
    \begin{Equation}
        U=\frac{N_\text{t}r_\text{n}r_\text{p}[(n_0+p_0)\delt{p}+(\delt{p})^2]}{r_\text{n}(n_0+n_1+\delt{p})+r_\text{p}(p_0+p_1+\delt{p})}
    \end{Equation}
\end{BoxFormula}

\begin{Proof}
    根据\fancyref{def:间接复合的净复合率}
    \begin{Equation}&[1]
        U=R_\text{n}-G_\text{n}=R_\text{p}-G_\text{p}
    \end{Equation}
    使用其中任何一个等式都可以帮助我们得到正确的结论,这里选取$R_\text{n}-G_\text{n}$的等式。

    根据\fancyref{fml:电子俘获率}和\fancyref{fml:电子发射系数的再表示}
    \begin{Equation}&[2]
        U=r_\text{n}n(N_\text{t}-n_\text{t})-r_\text{n}n_1n_\text{t}
    \end{Equation}
    稍作整理
    \begin{Equation}&[3]
        U=r_\text{n}nN_\text{t}-r_\text{n}(n+n_1)n_\text{t}
    \end{Equation}
    在\xrefpeq{3}中代入\fancyref{fml:间接复合的复合中心电子浓度}
    \begin{Equation}&[4]
        U=r_\text{n}nN_\text{t}-\frac{r_n(n+n_1)N_\text{t}(r_\text{n}n+r_\text{p}p_1)}{r_\text{n}(n+n_1)+r_\text{p}(p+p_1)}
    \end{Equation}
    通分,式子有些庞大,但还请不要惊慌
    \begin{Equation}&[5]
        \qquad
        U=\frac{r_\text{n}^2n(n+n_1)N_\text{t}+r_\text{n}r_\text{p}n(p+p_1)N_\text{t}-r_\text{n}^2n(n+n_1)N_\text{t}-r_\text{n}r_\text{p}p_1(n+n_1)N_\text{t}}{r_\text{n}(n+n_1)+r_\text{p}(p+p_1)}
        \qquad
    \end{Equation}
    让我们惊喜的是,\xrefpeq{5}分子的第一项和第三项是可以约掉的
    \begin{Equation}&[6]
        U=\frac{r_\text{n}r_\text{p}n(p+p_1)N_\text{t}-r_\text{n}r_\text{p}p_1(n+n_1)N_\text{t}}{r_\text{n}(n+n_1)+r_\text{p}(p+p_1)}
    \end{Equation}
    再整理
    \begin{Equation}&[7]
        U=\frac{r_\text{n}r_\text{p}N_\text{t}\qty(np+np_1-np_1-n_1p_1)}{r_\text{n}(n+n_1)+r_\text{p}(p+p_1)}
    \end{Equation}
    约掉\xrefpeq{7}的第二第三项,并注意到$n_1p_1=n_\text{i}^2$
    \begin{Equation}&[8]
        U=\frac{r_\text{n}r_\text{p}N_\text{t}\qty(np-n_\text{i}^2)}{r_\text{n}(n+n_1)+r_\text{p}(p+p_1)}
    \end{Equation}
    最后,代入$n=n_0+\delt{n}, p=p_0+\delt{p}$,以及$\delt{n}=\delt{p}$
    \begin{Equation}*
        U=\frac{N_\text{t}r_\text{n}r_\text{p}[(n_0+p_0)\delt{p}+(\delt{p})^2]}{r_\text{n}(n_0+n_1+\delt{p})+r_\text{p}(p_0+p_1+\delt{p})}\qedhere
    \end{Equation}
\end{Proof}

\begin{BoxFormula}[间接复合的寿命]
    间接复合的寿命满足
    \begin{Equation}
        \tau=\frac{r_\text{n}(n_0+n_1+\delt{p})+r_\text{p}(p_0+p_1+\delt{p})}{N_\text{t}r_\text{p}r_\text{n}(n_0+p_0+\delt{p})}
    \end{Equation}
    在小注入条件下,即$\delt{p}\ll(n_0+p_0)$时,可以近似为
    \begin{Equation}
        \tau=\frac{r_\text{n}(n_0+n_1)+r_\text{p}(p_0+p_1)}{N_\text{t}r_\text{p}r_\text{n}(n_0+p_0)}
    \end{Equation}
\end{BoxFormula}
\begin{Proof}
    根据\fancyref{fml:净复合率与寿命的关系}和\fancyref{fml:间接复合的净复合率}
    \begin{Equation}&[A]
        \tau=\frac{\delt{p}}{U}=\delt{p}\frac{r_\text{n}(n_0+n_1+\delt{p})+r_\text{p}(p_0+p_1+\delt{p})}{N_\text{t}r_\text{n}r_\text{p}[(n_0+p_0)\delt{p}+(\delt{p})^2]}
    \end{Equation}
    即
    \begin{Equation}&[B]
        \tau=\frac{r_\text{n}(n_0+n_1+\delt{p})+r_\text{p}(p_0+p_1+\delt{p})}{N_\text{t}r_\text{p}r_\text{n}(n_0+p_0+\delt{p})}
    \end{Equation}
    在小注入近似下,所有的$\delt{p}$都可以忽略掉。
\end{Proof}

\fancyref{fml:间接复合的寿命}指出,在小注入近似下,寿命近取决于$n_0,n_1,p_0,p_1$的值,与非平衡载流子的浓度$\delt{p}$无关,由于$N_\text{c},N_\text{v}$通常具有相近的数值,那么
\begin{Gather}[6pt]
    n_0\sim\exp[-(E_\text{c}-E_\text{F})]\qquad
    n_1\sim\exp[-(E_\text{c}-E_\text{t})]\\
    p_0\sim\exp[-(E_\text{F}-E_\text{v})]\qquad
    p_1\sim\exp[-(E_\text{t}-E_\text{v})]
\end{Gather}
当$\kB T$比这些能量间隔小的多时,往往$n_0,n_1,p_0,p_1$之间是高低悬殊,有若干数量级之差,所以,在\xrefpeq{B}中,我们往往只需要考虑$n_0,n_1,p_0,p_1$的最大者,从而使得问题大为简化。

从分类讨论上,我们主要有以下三个维度,如\xref{fig:间接复合的近似原理}
\begin{itemize}
    \item 半导体是N型还是P型?
    \item 半导体的掺杂是重掺还是轻掺(即$E_\text{F}$距导带和价带的距离是近还是远)?
    \item 复合中心能级,是更靠近价带还是更靠近导带?
\end{itemize}
从结果上看,我们主要有三种结果
\begin{itemize}
    \item 若半导体重掺,使得费米能级$E_\text{F}$比复合中心能级$E_\text{t}$及其关于禁带中心的镜像$E_\text{t}'$更靠近导带(N型)或价带(P型),那么$E_\text{c}-E_\text{F}$或$E_\text{F}-E_\text{v}$是最小的,因此$n_0$或$p_0$是所有载流子浓度$n_0,n_1,p_0,p_1$中的最大者,此时半导体称为\uwave{强N型区}或\uwave{强P型区}。
    \item 若半导体轻掺,使费米能级比$E_\text{F}$比$E_\text{t},E_\text{t'}$更靠近中线,此时称为\uwave{高阻态}
    \item 高阻态中,若$E_\text{t}$与$E_\text{F}$在中线同侧,N型$n_1$最大,P型$p_1$最大。
    \item 高阻态中,若$E_\text{t}$与$E_\text{F}$在中线异侧,N型$p_1$最大,P型$n_1$最大。
\end{itemize}

\begin{Table}[间接复合的近似原理]{|c|c|c|c|c|}
<
\mr{2}{掺杂}&\mc{2}(c|){复合中心近价带}&\mc{2}(c|){复合中心近导带}\\\clinelig{2-5}
&重掺&轻掺&重掺&轻掺\\
>
\mr{3}{N型}&
\xcell[0.2cm][0.0cm]{\includegraphics[width=3cm]{build/Chapter05D_03.fig.pdf}}&
\xcell[0.2cm][0.0cm]{\includegraphics[width=3cm]{build/Chapter05D_04.fig.pdf}}&
\xcell[0.2cm][0.0cm]{\includegraphics[width=3cm]{build/Chapter05D_05.fig.pdf}}&
\xcell[0.2cm][0.0cm]{\includegraphics[width=3cm]{build/Chapter05D_06.fig.pdf}}\\
&强N型区&高阻区&强N型区&高阻区\\
&\xgp[0.0cm][0.3cm]{$n_0\gg(\,\cdots)$}&$p_1\gg(\,\cdots), n_0\gg p_0$
&$n_0\gg(\,\cdots)$&$n_1\gg(\,\cdots), n_0\gg p_0$\\ \hlinelig
\mr{3}{P型}&
\xcell[0.2cm][0.0cm]{\includegraphics[width=3cm]{build/Chapter05D_07.fig.pdf}}&
\xcell[0.2cm][0.0cm]{\includegraphics[width=3cm]{build/Chapter05D_08.fig.pdf}}&
\xcell[0.2cm][0.0cm]{\includegraphics[width=3cm]{build/Chapter05D_09.fig.pdf}}&
\xcell[0.2cm][0.0cm]{\includegraphics[width=3cm]{build/Chapter05D_10.fig.pdf}}\\
&强P型区&高阻区&强P型区&高阻区\\
&\xgp[0.0cm][0.3cm]{$p_0\gg(\,\cdots)$}&$p_1\gg(\,\cdots), p_0\gg n_0$
&$p_0\gg(\,\cdots)$&$n_1\gg(\,\cdots), p_0\gg n_0$\\
\end{Table}

这样一来,我们就有以下六个近似式
\begin{Table}[间接复合的寿命近似]{ccccll}
<掺杂&掺杂浓度&复合中心&工作区&近似&寿命\\>
N型&重掺&/&强N型区&$n_0\gg n_1,p_0,p_1$&\xgp[4pt]{$\mal{\tau=\tau_\text{p}=\frac{1}{N_\text{t}r_\text{p}}}$}\\
N型&轻掺&近导带&高阻区&$n_1\gg n_0,p_0,p_1\quad n_0\gg p_0$&\xgp[4pt]{$\mal{\tau=\frac{n_1}{N_\text{t}r_\text{p}}\frac{1}{n_0}}$}\\
N型&轻掺&近价带&高阻区&$p_1\gg p_0,n_0,n_1\quad n_0\gg p_0$&\xgp[4pt]{$\mal{\tau=\frac{p_1}{N_\text{t}r_\text{n}}\frac{1}{n_0}}$}\\
\hlinelig
P型&重掺&/&强P型区&$p_0\gg p_1,n_0,n_1$&\xgp[4pt]{$\mal{\tau=\tau_\text{n}=\frac{1}{N_\text{t}r_\text{n}}}$}\\
P型&轻掺&近导带&高阻区&$n_1\gg n_0,p_0,p_1\quad p_0\gg n_0$&\xgp[4pt]{$\mal{\tau=\frac{n_1}{N_\text{t}r_\text{p}}\frac{1}{p_0}}$}\\
P型&轻掺&近价带&高阻区&$p_1\gg p_0,n_0,n_1\quad p_0\gg n_0$&\xgp[4pt]{$\mal{\tau=\frac{p_1}{N_\text{t}r_\text{n}}\frac{1}{p_0}}$}\\
\end{Table}
特别注意的是,N型半导体对应的是$\tau_\text{p}$,P型半导体对应的是$\tau_\text{n}$,是反的!

现在,让我们反过来通过寿命研究净复合率和复合中心能级间的关系。

根据\fancyref{fml:间接复合的净复合率}推导过程中的\xrefpeq[间接复合的净复合率]{8},复合率$U$可以表示为\setpeq{复合率与复合中心能级}
\begin{Equation}&[1]
    U=\frac{r_\text{n}r_\text{p}N_\text{t}\qty(np-n_\text{i}^2)}{r_\text{n}(n+n_1)+r_\text{p}(p+p_1)}
\end{Equation}
上下同除$r_\text{n}r_\text{p}N_\text{t}$
\begin{Equation}&[2]
    U=\frac{np-n_\text{i}^2}{(1/r_\text{p}N_\text{t})(n+n_1)+(1/r_\text{n}N_\text{t})(p+p_1)}
\end{Equation}
代入\xref{tab:间接复合的寿命近似}中$\tau_\text{n},\tau_\text{p}$的表达式
\begin{Equation}&[3]
    U=\frac{np-n_\text{i}^2}{\tau_\text{p}(n+n_1)+\tau_\text{n}(p+p_1)}
\end{Equation}
根据\xref{fml:电子发射系数的再表示}和\xref{fml:空穴发射系数的再表示}中的$n_1,p_1$的表达式
\begin{Equation}&[5]
    n_1=N_\text{c}\exp(\frac{E_\text{t}-E_\text{c}}{\kB T})\qquad
    p_1=N_\text{v}\exp(\frac{E_\text{v}-E_\text{t}}{\kB T})
\end{Equation}
改写为
\begin{Equation}&[6]
    n_1=n_\text{i}\exp(\frac{E_\text{t}-E_\text{i}}{\kB T})\qquad
    p_1=n_\text{i}\exp(\frac{E_\text{i}-E_\text{t}}{\kB T})
\end{Equation}
将\xrefpeq{6}代入\xrefpeq{4}
\begin{Equation}&[7]
    \qquad\qquad\qquad
    U=
    \frac{np-n_\text{i}^2}
    {
        \tau_\text{p}[n+n_\text{i}\exp(E_\text{t}-E_\text{i}/\kB T)]+
        \tau_\text{n}[p+n_\text{i}\exp(E_\text{i}-E_\text{t}/\kB T)]}
    \qquad\qquad\qquad
\end{Equation}
简单起见,可以假定$r_\text{n}=r_\text{p}=r$,这样$\tau_\text{n}=\tau_\text{p}=1/N_\text{t}r$
\begin{Equation}
    U=\frac{N_\text{t}r(np-n_\text{i}^2)}{n+p+2n_\text{i}\cosh(E_\text{t}-E_\text{i}/\kB T)}
\end{Equation}
由此可见,当$E_\text{t}=E_\text{i}$时,净复合率$U$趋向于极大,因此,位于禁带中央附近的深能级杂质是最有效的复合中心,而浅能级杂质则是无效的复合中心,由此,也引出了两种杂质功能上的差异,即,\empx{浅能级杂质提供多数载流子},\empx{深能级杂质形成复合中心}。在硅中最常用的复合中心是金,在硅中掺入少量的金,就可以显著提高复合率,从而极大的缩短非平衡载流子的寿命。

\subsection{表面复合}
前面我们只考虑了半导体内部的复合过程,但实际上,非平衡载流子的寿命在很大程度上受表面状态和形状大小的影响。例如半导体表面的粗糙程度和样品的大小,都会显著影响寿命。

若体内复合的寿命为$\tau_\text{v}$,而表面复合的寿命为$\tau_\text{s}$,则总寿命为
\begin{Equation}
    \frac{1}{\tau}=\frac{1}{\tau_\text{v}}+\frac{1}{\tau_\text{s}}
\end{Equation}
实验表明,表面复合率$U_\text{s}$与表面处非平衡载流子浓度$(\delt{p})_\text{s}$成正比。
\begin{BoxFormula}[表面复合率]
    表面复合率正比于表面处的非平衡载流子浓度
    \begin{Equation}
        U_\text{s}=s(\delt{p})_\text{s}
    \end{Equation}
\end{BoxFormula}
这里$s$可以赋予一个直观意义,即表面处的非平衡载流子以垂直速度$s$流出表面。